\documentclass{article}
\usepackage{amsmath}
\usepackage{diagbox}
\usepackage{diagbox}
\usepackage{colortbl}
\usepackage{colortbl}
\usepackage{amssymb}
\usepackage{tikz}
\DeclareMathOperator*{\argmax}{arg\,max}
\usetikzlibrary{arrows.meta, positioning}
\usepackage{fancyhdr}
\usepackage{xcolor}

\definecolor{mycolumncolor}{rgb}{0.9, 0.9, 0.9}
\usepackage{pgfplots}
\usepackage{diagbox}
\usepackage{geometry}
\usepackage{hyperref}
\usepackage{verbatim}
\geometry{a4paper, margin=1in}
\usepackage[utf8]{inputenc}
\usepackage[T2A]{fontenc}
\usepackage[russian]{babel}
\usepackage{colortbl}
\usetikzlibrary{intersections}
\usepgfplotslibrary{fillbetween}
\pgfplotsset{compat=1.18}
\usepackage{graphicx} 
\usepackage{listings} 
\usepackage{float}
\usepackage{caption}
\usepackage{graphicx}
\hypersetup{
    colorlinks=true,   
    linkcolor=black,     
    urlcolor=blue,       
    citecolor=blue,      
    filecolor=blue,        
    pdfborder={0 0 0}     
}

\tikzset{
mybox/.style={
        rectangle, draw, rounded corners, align=center, minimum height=1cm, minimum width=2.5cm
    },
myarrow/.style={
-{Latex[length=3mm, width=2mm]}, thick
}
}

\pagestyle{fancy}
\fancyhf{}
\fancyhead[L]{ФИО: Зацепилин А.В.}
\fancyhead[R]{Группа: Б22-534}

\begin{document}

% Титульный лист
\begin{titlepage}
    \centering
    {\large Министерство науки и высшего образования Российской Федерации\\[0.5cm]}
    {\Large ФЕДЕРАЛЬНОЕ ГОСУДАРСТВЕННОЕ АВТОНОМНОЕ ОБРАЗОВАТЕЛЬНОЕ УЧРЕЖДЕНИЕ\\
        ВЫСШЕГО ОБРАЗОВАНИЯ}\\[0.5cm]
    {\Large Национальный исследовательский ядерный университет «МИФИ»\\
    (НИЯУ МИФИ)}\\[1cm]
    {\large ИНСТИТУТ ИНТЕЛЛЕКТУАЛЬНЫХ КИБЕРНЕТИЧЕСКИХ СИСТЕМ\\
    КАФЕДРА КИБЕРНЕТИКИ}\\[1cm]
    \includegraphics[width=0.3\textwidth]{IIKS.png}\\[1.5cm] % Логотип университета
    {\huge Отчет по курсу «Методы оптимизации»}\\[5cm]

    \begin{flushright}
        \textbf{Выполнил:}\\
        Студент группы Б22-534\\
        Зацепилин А.В.\\[0.5cm]
        \textbf{Вариант №55}
    \end{flushright}

    \vfill
    Москва, осень 2024
\end{titlepage}

\tableofcontents
\newpage

\section{Задание №1}

\subsection{Задача (a)}

Оптимизационная задача:
\[
    F = 3x_1 + 4x_2 \to \max, \ \min
\]
Ограничения:
\[
    \begin{cases}
        x_1 - 3 \cdot x_2 \leq 3  \\
        x_1 + x_2 \leq 10         \\
        -x_1 + 4 \cdot x_2 \leq 4 \\
        x_1 \geq 0, \ x_2 \geq 0
    \end{cases}
\]

\begin{figure}[h]
    \centering
    \begin{tikzpicture}
        \begin{axis}[
                xlabel={$x_1$},
                ylabel={$x_2$},
                axis lines=middle,
                xmin=0, xmax=10,
                ymin=0, ymax=10,
                grid=both,
                width=14cm,
                height=12cm,
                legend pos=north east
            ]
            % Линии ограничений с разными цветами
            \addplot[name path=A, domain=0:10, thick, blue] { (x - 3)/3 };
            \addplot[name path=B, domain=0:10, thick, red] { 10 - x };
            \addplot[name path=C, domain=0:10, thick, green] { (4 + x)/4 };

            % Область допустимых решений
            \addplot[fill opacity=0.2, color=gray] fill between[
                    of=A and B,
                    soft clip={domain=6:10}
                ];

            % Вектор {3, 4}
            \addplot[->, thick, black] coordinates {(0,0) (3,4)};
            \node at (axis cs: 3,4) [anchor=west] {$\vec{v} = \{3,4\}$};

            % Подписи
            \legend{
                $x_1 - 3x_2 = 3$,
                $x_1 + x_2 = 10$,
                $-x_1 + 4x_2 = 4$
            }

        \end{axis}
    \end{tikzpicture}

    \caption{Графическое решение задачи (a)}
\end{figure}

Вычисление точек пересечения

1. Минимум \(x_1 - 3x_2 = 3\) и \(x_2 = 0\):
\[
    \begin{cases}
        x_1 = 3 + 3x_2 \\
        x_2 = 0 \implies x_1 =  3
    \end{cases}
\]
Точка пересечения: \((3,0)\).

2. Максимум \(x_1 + x_2 = 10\) и \(-x_1 + 4x_2 = 4\):
\[
    \begin{cases}
        x_1 + x_2 = 10 \\
        -x_1 + 4x_2 = 4
    \end{cases}
    \implies
    \begin{cases}
        x_1 = \frac{36}{5} \\
        x_2 = \frac{14}{5}
    \end{cases}
\]
Точка пересечения: \(\left(\frac{36}{5}, \frac{14}{5}\right)\).

Вычисление значений целевой функции \(F = 3x_1 + 4x_2\) в найденных точках:

- В точке \((3, 0)\):
\[
    F(3, 0) = 9
\]

- В точке \(\left(\frac{36}{5}, \frac{14}{5}\right)\):
\[
    F\left(\frac{36}{5}, \frac{14}{5}\right) = \frac{164}{5}
\]

Ответ:
\begin{tabular}{l l}
    Максимум функции \(F = 3x_1 + 4x_2\) & достигается в точке \(\left(\frac{36}{5}, \frac{14}{5}\right)\), где \(F =\frac{164}{5}\). \\
    Минимум функции \(F = 3x_1 + 4x_2\)  & достигается в точке \((3, 0)\), где \(F = 9\).
\end{tabular}

\newpage

\subsection{Задача (b)}

Оптимизационная задача:
\[
    F = x_1 + 7x_2 \to \max, \ \min
\]
Ограничения:
\[
    \begin{cases}
        3 \cdot x_1 + 4 \cdot x_2 \geq 12 \\
        2 \cdot x_1 - x_2 \leq 6          \\
        x_1 \geq 0, \ x_2 \geq 0
    \end{cases}
\]

Построение графиков

\begin{figure}[h]
    \centering
    \begin{tikzpicture}
        \begin{axis}[
                xlabel={$x_1$},
                ylabel={$x_2$},
                axis lines=middle,
                xmin=0, xmax=10,
                ymin=0, ymax=10,
                grid=both,
                width=14cm,
                height=12cm,
                legend pos=north east
            ]
            % Линии ограничений с разными цветами
            \addplot[name path=A, domain=0:10, thick, blue] { (12 - 3*x)/4 };
            \addplot[name path=B, domain=0:10, thick, red] { 2*x - 6 };

            % Область допустимых решений
            \addplot[fill opacity=0.2, color=gray] fill between[
                    of=A and B,
                    soft clip={domain=0:10}
                ];

            \addplot[->, thick, black] coordinates {(0,0) (1,7)};
            \node at (axis cs: 1,7) [anchor=west] {$\vec{v} = \{1,7\}$};

            % Подписи
            \legend{
                $3x_1 + 4x_2 = 12$,
                $2x_1 - x_2 = 6$
            }
        \end{axis}
    \end{tikzpicture}
    \caption{Графическое решение задачи (b)}
\end{figure}

Вычисление точек пересечения

1. Минимум \(3x_1 + 4x_2 = 12\) и \(2x_1 - x_2 = 6\):
\[
    \begin{cases}
        3x_1 + 4x_2 = 12 \\
        2x_1 - x_2 = 6
    \end{cases} \implies
    \begin{cases}
        x_1 = \frac{36}{11} \\
        x_2 = \frac{6}{11}
    \end{cases}
\]
Точка пересечения: \(\left(\frac{36}{11}, \frac{6}{11}\right)\).

Вычисление значений целевой функции \(F = x_1 + 7x_2\) в найденной точке:

В точке \(\left(\frac{36}{11}, \frac{6}{11}\right)\):
\[
    F\left(\frac{36}{11}, \frac{6}{11}\right) = \frac{78}{11}
\]

Ответ:
\begin{tabular}{l l}
    Максимума функции не существует.                                                                                                  \\
    Минимум функции \(F = x_1 + 7x_2\) & достигается в точке \(\left(\frac{36}{11}, \frac{6}{11}\right)\), где \(F = \frac{78}{11}\).
\end{tabular}

\newpage

\subsection{Задача (c)}

Оптимизационная задача:
\[
    F = 2x_1 + x_2 \to \max, \ \min
\]
Ограничения:
\[
    \begin{cases}
        3 \cdot x_1 - x_2 \geq 9 \\
        x_1 + x_2 \leq 2         \\
        x_1 \geq 0, \ x_2 \geq 0
    \end{cases}
\]

Построение графиков

\begin{figure}[h]
    \centering
    \begin{tikzpicture}
        \begin{axis}[
                xlabel={$x_1$},
                ylabel={$x_2$},
                axis lines=middle,
                xmin=-5, xmax=10,
                ymin=-5, ymax=10,
                grid=both,
                width=14cm,
                height=12cm,
                legend pos=north east
            ]
            % Линии ограничений с разными цветами
            \addplot[name path=A, domain=0:10, thick, blue] { 3*x - 9 };
            \addplot[name path=B, domain=0:10, thick, red] { 2 - x };

            % Область допустимых решений
            \addplot[fill opacity=0.2, color=gray] fill between[
                    of=A and B,
                    soft clip={domain=0:10}
                ];

            \addplot[->, thick, black] coordinates {(0,0) (2,1)};
            \node at (axis cs: 2,1) [anchor=west] {$\vec{v} = \{2,1\}$};

            % Подписи
            \legend{
                $3x_1 - x_2 = 9$,
                $x_1 + x_2 = 2$
            }
        \end{axis}
    \end{tikzpicture}
    \caption{Графическое решение задачи (c)}
\end{figure}

Ответ:
\begin{tabular}{l l}
    Максимума функции не существует. \\
    Минимума функции не существует.
\end{tabular}

\section{Задание №2}

\subsection{Задача (a)}

Оптимизационная задача (из задачи 1.1):
\[
    F = 3x_1 + 4x_2 \to \max
\]
Ограничения:
\[
    \begin{cases}
        x_1 - 3 \cdot x_2 \leq 3  \\
        x_1 + x_2 \leq 10         \\
        -x_1 + 4 \cdot x_2 \leq 4 \\
        x_1 \geq 0, \ x_2 \geq 0
    \end{cases}
\]

Решение задачи с помощью симплекс-метода:
\[
    F - 3x_1 - 4x_2 = 0
\]

\[
    \begin{cases}
        x_1 - 3 \cdot x_2 + x_3 = 3  \\
        x_1 + x_2 + x_4 = 10         \\
        -x_1 + 4 \cdot x_2 + x_5 = 4 \\
        x_1 \geq 0, \ x_2 \geq 0, \ x_3 \geq 0, \ x_4 \geq 0, \ x_5 \geq 0
    \end{cases}
\]

\vspace{25pt}

\[
    \begin{array}{|c|c|>{\columncolor{green}}c|c|c|c|c|c|}
        \hline
             & x_1 & x_2 & x_3 & x_4 & x_5 & b_i & b_i/p.c. \geq 0 \\
        \hline
        x_3  & 1   & -3  & 1   & 0   & 0   & 3   & -1              \\
        x_4  & 1   & 1   & 0   & 1   & 0   & 10  & 10              \\
        \rowcolor{yellow}
        x_5  & -1  & 4   & 0   & 0   & 1   & 4   & 1               \\
        F(x) & -3  & -4  & 0   & 0   & 0   & 0   & 0               \\
        \hline
    \end{array}
\]

В последней строке есть элементы \(\leq 0\). Занулим элементы выше и ниже стоящие от разрешающего элемента.

\[
    \begin{array}{|c|>{\columncolor{green}}c|c|c|c|c|c|c|}
        \hline
             & x_1          & x_2 & x_3 & x_4 & x_5          & b_i & b_i/p.c. \geq 0 \\
        \hline
        x_2  & -\frac{1}{4} & 1   & 0   & 0   & \frac{1}{4}  & 1   & -4              \\
        x_3  & \frac{1}{4}  & 0   & 1   & 0   & \frac{3}{4}  & 6   & 24              \\
        \rowcolor{yellow}
        x_4  & \frac{5}{4}  & 0   & 0   & 1   & -\frac{1}{4} & 9   & \frac{36}{5}    \\
        F(x) & -4           & 0   & 0   & 0   & 1            & 4   &                 \\
        \hline
    \end{array}
\]

В последней строке есть элементы \(\leq 0\). Занулим элементы выше и ниже стоящие от разрешающего элемента.

\[
    \begin{array}{|c|c|c|c|c|c|c|c|}
        \hline
             & x_1 & x_2 & x_3 & x_4          & x_5          & b_i           & b_i/p.c. \geq 0 \\
        \hline
        x_1  & 1   & 0   & 0   & \frac{4}{5}  & -\frac{1}{5} & \frac{36}{5}  & \frac{36}{5}    \\
        x_2  & 0   & 1   & 0   & \frac{1}{5}  & \frac{4}{20} & \frac{14}{5}  & -               \\
        x_3  & 0   & 0   & 1   & -\frac{1}{5} & \frac{4}{5}  & \frac{21}{5}  & 24              \\
        F(x) & 0   & 0   & 0   & \frac{16}{5} & \frac{1}{5}  & \frac{164}{5} &                 \\
        \hline
    \end{array}
\]

В последней строке не осталось элементов \(\leq 0\). Мы пришли к конечной таблице.\\
Максимум функции достигается при \(x_1 = \frac{36}{5}, x_2 = \frac{14}{5}\), и значение целевой функции равно \(F(x) = \frac{164}{5}\).

\newpage

\subsection{Задача (b)}

Оптимизационная задача (из задачи 1.2):
\[
    F = x_1 + 7x_2 \to \max
\]
Ограничения:
\[
    \begin{cases}
        3 \cdot x_1 + 4 \cdot x_2 \geq 12 \\
        2 \cdot x_1 - x_2 \leq 6          \\
        x_1 \geq 0, \ x_2 \geq 0
    \end{cases}
\]

Решение задачи с помощью симплекс-метода:
\[
    F - x_1 - 7x_2 = 0
\]

\[
    \begin{cases}
        -3 \cdot x_1 - 4 \cdot x_2 + x_3 =  -12 \\
        2 \cdot x_1 - x_2 + x_4 = 6             \\
        x_1 \geq 0, \ x_2 \geq 0
    \end{cases}
\]

\vspace{25pt}

\[
    \begin{array}{|c|c|>{\columncolor{pink}}c|c|c|c|c|}
        \hline
             & x_1 & x_2 & x_3 & x_4 & b_i & b_i/p.c. \geq 0 \\
        \hline
        x_3  & -3  & -4  & 1   & 0   & -12 & -               \\
        x_4  & 2   & -1  & 0   & 1   & 6   & -               \\
        F(x) & -1  & -7  & 0   & 0   & 0   & -               \\
        \hline
    \end{array}
\]

В последней строке есть элементы \(\leq 0\). Минимальный из них -7, но т.к. все элементы этого столбца отрицательные, то область допустимых решений неограниченна.


Оптимизационная задача (из задачи 1.2):
\[
    F = x_1 + 7x_2 \to \min
\]
Переведём эту задачу в поиск максимума взяв обратную функцию от изначальной.
\[
    G = -x_1 - 7x_2 \to \max
\]
Ограничения:
\[
    \begin{cases}
        3 \cdot x_1 + 4 \cdot x_2 \geq 12 \\
        2 \cdot x_1 - x_2 \leq 6          \\
        x_1 \geq 0, \ x_2 \geq 0
    \end{cases}
\]

Решение задачи с помощью симплекс-метода:
\[
    G + x_1 + 7x_2 = 0
\]

\[
    \begin{cases}
        -3 \cdot x_1 - 4 \cdot x_2 + x_3 =  -12 \\
        2 \cdot x_1 - x_2 + x_4 = 6             \\
        x_1 \geq 0, \ x_2 \geq 0
    \end{cases}
\]

\vspace{25pt}

\[
    \begin{array}{|c|>{\columncolor{green}}c|c|c|c|c|c|}
        \hline
             & x_1 & x_2 & x_3 & x_4 & b_i & b_i/p.c. \geq 0 \\
        \hline
        x_3  & -3  & -4  & 1   & 0   & -12 & 4               \\
        \rowcolor{yellow}
        x_4  & 2   & -1  & 0   & 1   & 6   & 3               \\
        G(x) & 1   & 7   & 0   & 0   & 0   & 0               \\
        \hline
    \end{array}
\]

В последней строке есть элементы \(\leq 0\). Занулим элементы выше и ниже стоящие от разрешающего элемента.

\[
    \begin{array}{|c|c|>{\columncolor{green}}c|c|c|c|c|}
        \hline
             & x_1 & x_2           & x_3 & x_4          & b_i & b_i/p.c. \geq 0 \\
        \hline
        x_1  & 1   & -\frac{1}{2}  & 0   & \frac{1}{2}  & 3   & -6              \\
        \rowcolor{yellow}
        x_3  & 0   & -\frac{11}{2} & 1   & \frac{3}{2}  & -3  & \frac{6}{11}    \\
        G(x) & 0   & \frac{15}{2}  & 0   & -\frac{1}{2} & -3  & -\frac{6}{15}   \\
        \hline
    \end{array}
\]

В последней строке есть элементы \(\leq 0\). Занулим элементы выше и ниже стоящие от разрешающего элемента.

\[
    \begin{array}{|c|c|c|c|c|c|c|}
        \hline
             & x_1 & x_2 & x_3           & x_4           & b_i           & b_i/p.c. \geq 0 \\
        \hline
        x_2  & 0   & 1   & -\frac{2}{11} & -\frac{3}{11} & \frac{6}{11}  & -               \\
        x_1  & 1   & 0   & -\frac{1}{11} & \frac{4}{11}  & \frac{36}{11} & -               \\
        G(x) & 0   & 0   & \frac{4}{165} & \frac{17}{11} & \frac{78}{11} & -               \\
        \hline
    \end{array}
\]

В последней строке не осталось элементов \(\leq 0\). Мы пришли к конечной таблице.\\
Максимум функции достигается при \(x_1 = \frac{36}{11}, x_2 = \frac{6}{11}\), и значение целевой функции равно \(F(x) = \frac{78}{11}\).

\newpage

\subsection{Задача (с)}

Оптимизационная задача (из задачи 1.3):
\[
    F = 2x_1 + x_2 \to \max
\]
Ограничения:
\[
    \begin{cases}
        3 \cdot x_1 - x_2 \geq 9 \\
        x_1 + x_2 \leq 2         \\
        x_1 \geq 0, \ x_2 \geq 0
    \end{cases}
\]

Решение задачи с помощью симплекс-метода:
\[
    F - 2x_1 - x_2 = 0
\]

\[
    \begin{cases}
        -3 \cdot x_1 + x_2 + x_3 =  -9 \\
        x_1 + x_2 + x_4 = 2            \\
        x_1 \geq 0, \ x_2 \geq 0
    \end{cases}
\]

\vspace{25pt}

\[
    \begin{array}{|c|>{\columncolor{green}}c|c|c|c|c|c|}
        \hline
             & x_1 & x_2 & x_3 & x_4 & b_i & b_i/p.c. \geq 0 \\
        \hline
        x_3  & -3  & 1   & 1   & 0   & -9  & -               \\
        \rowcolor{yellow}
        x_4  & 1   & 1   & 0   & 1   & 2   & 2               \\
        F(x) & -2  & -1  & 0   & 0   & 0   & -               \\
        \hline
    \end{array}
\]
Первую и последнюю строки не вычисляем для последнего столбца т.к. элементы р.с. \(\leq 0\).
В последней строке есть элементы \(\leq 0\). Занулим элементы выше и ниже стоящие от разрешающего элемента.

\[
    \begin{array}{|c|c|c|c|c|>{\columncolor{pink}}c|c|}
        \hline
             & x_1 & x_2 & x_3 & x_4 & b_i & b_i/p.c. \geq 0 \\
        \hline
        x_1  & 1   & 1   & 0   & 1   & 2   & -               \\
        x_3  & 0   & 4   & 1   & 3   & -3  & -               \\
        F(x) & 0   & 1   & 0   & 2   & 4   & -               \\
        \hline
    \end{array}
\]

В последней строке не осталось элементов \(\leq 0\).\\
Мы пришли к конечной таблице. Т.к. не все \(b_i\geq 0\) \(\implies\) решения не существует.


\newpage

\section{Задание №3}
\subsection{Задача (a)}

Оптимизационная задача (из задачи 1.1):
\[
    F = 3x_1 + 4x_2 \to \max
\]
Ограничения:
\[
    \begin{cases}
        x_1 - 3 \cdot x_2 \leq 3  \\
        x_1 + x_2 \leq 10         \\
        -x_1 + 4 \cdot x_2 \leq 4 \\
        x_1 \geq 0, \ x_2 \geq 0
    \end{cases}
\]

Решение задачи с помощью симплекс-метода:

\[
    \begin{array}{|c|c|c|c|c|c|c|c|}
        \hline
             & x_1 & x_2 & x_3 & x_4          & x_5          & b_i           & b_i/p.c. \geq 0 \\
        \hline
        x_1  & 1   & 0   & 0   & \frac{4}{5}  & -\frac{1}{5} & \frac{36}{5}  & \frac{36}{5}    \\
        x_2  & 0   & 1   & 0   & \frac{1}{5}  & \frac{4}{20} & \frac{14}{5}  & -               \\
        x_3  & 0   & 0   & 1   & -\frac{1}{5} & \frac{4}{5}  & \frac{21}{5}  & 24              \\
        F(x) & 0   & 0   & 0   & \frac{16}{5} & \frac{1}{5}  & \frac{164}{5} &                 \\
        \hline
    \end{array}
\]

Максимум функции достигается при \(x_1 = \frac{36}{5}, x_2 = \frac{14}{5}\), и значение целевой функции равно \(F(x) = \frac{164}{5}\).\\

Составим двойственную задачу:

\[
    F^* = 3y_1 + 10x_2 + 4y_3 \to \min
\]

Ограничения:
\[
    \begin{cases}
        y_1 + y_2 - y_3 \geq 3                  \\
        -3 \cdot y_1 + y_2 + 4 \cdot y_3 \geq 4 \\
        y_1 \geq 0, \ y_2 \geq 0, \ y_3 \geq 0
    \end{cases}
\]\\

Решим задачу 1 способом для этого составим систему для нахождения \(y_1^*, y_2^*, y_3^*\):\\

\[
    \begin{cases}
        (\frac{36}{5} - 3\frac{14}{5} - 3) \cdot y_1^* = 0 \\
        (\frac{36}{5} + \frac{14}{5} - 10) \cdot y_2^* = 0 \\
        (-\frac{36}{5} + 4\frac{14}{5} - 4) \cdot y_3^* = 0
    \end{cases} \implies
    \begin{cases}
        -\frac{8}{5} \cdot y_1^* = 0 \implies y_1^* = 0 \\
        0 \cdot y_2^* = 0 \implies y_2^* \geq 0         \\
        0 \cdot y_3^* = 0 \implies y_3^* \geq 0
    \end{cases}
\]\\

Вычислим \(y_2^*, y_3^*\) с учётом что \(y_1^* = 0\)\\

\[
    \begin{cases}
        (y_1^* + y_2^* - y_3^* - 3) \cdot \frac{36}{5} = 0 \\
        (-3y_1^* + y_2^* + 4y_3^* - 4) \cdot \frac{14}{5} = 0
    \end{cases} \implies
    \begin{cases}
        y_2^* - y_3^* - 3 = 0 \implies y_2^* = \frac{15}{5} \\
        y_2^* + 4y_3^* - 4 = 0 \implies y_3^* = \frac{1}{5}
    \end{cases}
\]\\

Вектор решения: \[y^* = (0;\frac{15}{5};\frac{1}{5})\]\\
Подставим решение в \(F^*\) и сравним с тем что получалось в  \(F\):\\
\[F^* = 10 \cdot \frac{16}{5} + 4 \cdot \frac{1}{5} = \frac{164}{5}\]\\

Правильное решение найдено.\\

\vspace{25pt}

Решим задачу 2 способом для этого возьмём конечную симплекс-таблицу для базовой задачи:\\

\[
    \begin{array}{|c|c|c|c|c|c|c|c|}
        \hline
        \text{Базис} & A_1 & A_2 & A_3 & A_4          & A_5          & B_i          & C \\
        \hline
        A_1          & 1   & 0   & 0   & \frac{4}{5}  & -\frac{1}{5} & \frac{36}{5} & 3 \\
        A_2          & 0   & 1   & 0   & \frac{1}{5}  & \frac{4}{20} & \frac{14}{5} & 4 \\
        A_3          & 0   & 0   & 1   & -\frac{1}{5} & \frac{4}{5}  & \frac{21}{5} & 0 \\
        \hline
    \end{array}
\]\\

Считаем по формуле: \(y^* = C \cdot A^{-1}\)\\

Посчитаем значение \(y^*\):

\[
    y^* =\begin{pmatrix} 3 & 4 & 0 \end{pmatrix} \cdot \begin{pmatrix}
        0 & \frac{4}{5}  & -\frac{1}{5} \\
        0 & \frac{1}{5}  & \frac{4}{20} \\
        1 & -\frac{1}{5} & \frac{4}{5}
    \end{pmatrix} = \begin{pmatrix} 0 & \frac{16}{5} & \frac{1}{5} \end{pmatrix}
\]

Теперь посчитаем значение функции \(F^*\):

\[
    F^* = 10 \cdot \frac{16}{5} + 4 \cdot \frac{1}{5} = \frac{164}{5}
\]

Правильное решение найдено.



\newpage

\section{Задание №4(6)}

Условие задачи:

\[
    \begin{array}{|c|c|c|c|c|c|}
        \hline
        \text{Сырьё}      & A & B & C & D  & \text{Запасы} \\
        \hline
        \text{Металл}     & 1 & 6 & 4 & 5  & 800           \\
        \hline
        \text{Пластмасса} & 5 & 9 & 8 & 10 & 2500          \\
        \hline
        \text{Резина}     & 0 & 3 & 1 & 5  & 600           \\
        \hline
        \text{Прибыль}    & 2 & 7 & 8 & 4  & -             \\
        \hline
    \end{array}
\]\\
Математическая интерпретация задачи:\\
\[
    F = 2x_1 + 7x_2 + 8x_3 + 4x_4 \to \max
\]

\[
    \begin{cases}
        x_1 + 6 \cdot x_2 + 4 \cdot x_3 + 5 \cdot x_4 \leq 800           \\
        5 \cdot x_1 + 9 \cdot x_2 + 8 \cdot x_3 + 10 \cdot x_4 \leq 2500 \\
        3 \cdot x_2 + x_3 + 5 \cdot x_4 \leq 600                         \\
    \end{cases}
\]\\
Составим условие задачи для решения симплекс методом:\\

\[
    F - 2x_1 - 7x_2 - 8x_3 - 4x_4 = 0
\]

\[
    \begin{cases}
        x_1 + 6 \cdot x_2 + 4 \cdot x_3 + 5 \cdot x_4 + x_5 = 800           \\
        5 \cdot x_1 + 9 \cdot x_2 + 8 \cdot x_3 + 10 \cdot x_4 + x_6 = 2500 \\
        3 \cdot x_2 + x_3 + 5 \cdot x_4 \leq 600 + x_7 = 600                \\
    \end{cases}
\]

Составим начальную симплекс-таблицу:\\

\[
    \begin{array}{|c|c|c|>{\columncolor{green}}c|c|c|c|c|c|c|}
        \hline
             & x_1 & x_2 & x_3 & x_4 & x_5 & x_6 & x_7 & b_i  & b_i/p.c. \geq 0 \\
        \hline
        \rowcolor{yellow}
        x_5  & 1   & 6   & 4   & 5   & 1   & 0   & 0   & 800  & 200             \\
        x_6  & 5   & 9   & 8   & 10  & 0   & 1   & 0   & 2500 & \frac{625}{2}   \\
        x_7  & 0   & 3   & 1   & 5   & 0   & 0   & 1   & 600  & 600             \\
        F(x) & -2  & -7  & -8  & -4  & 0   & 0   & 0   & 0    & 0               \\
        \hline
    \end{array}
\]

Т.к. в последней строке есть элементы  \(\leq 0\) выбираем минимальный отрицательный элемент в последнем столбце и считаем последний столбец после чего выбираем разрешающий элемент.

Занулим все элементы выше и ниже разрешающего элемента:

\[
    \begin{array}{|c|c|c|c|c|c|c|c|c|c|}
        \hline
             & x_1          & x_2         & x_3 & x_4          & x_5          & x_6 & x_7 & b_i  & b_i/p.c. \geq 0 \\
        \hline
        x_3  & \frac{1}{4}  & \frac{3}{2} & 1   & \frac{5}{4}  & \frac{1}{4}  & 0   & 0   & 200  & 200             \\
        x_6  & 3            & -3          & 0   & 0            & -2           & 1   & 0   & 900  & \frac{625}{2}   \\
        x_7  & -\frac{1}{4} & \frac{3}{2} & 0   & \frac{15}{4} & -\frac{1}{4} & 0   & 1   & 400  & 600             \\
        F(x) & 0            & 5           & 0   & 6            & 2            & 0   & 0   & 1600 & -               \\
        \hline
    \end{array}
\]

В последней строке все элементы \(\geq 0 \implies\) оптимальный план найден.

Максимум функции достигается при \(x_1 = 0, x_2 = 0, x_3 = 200, x_4 = 0\), и значение целевой функции равно \(F(x) =1600\).

Составим двойственную задачу:

\[
    F^* = 800y_1 + 2500y_2 + 600y_3 \to \min
\]

Конечная симлекс-таблица с добавлением столбца C:

\[
    \begin{array}{|c|c|c|c|c|c|c|c|c|c|c|}
        \hline
             & x_1          & x_2         & x_3 & x_4          & x_5          & x_6 & x_7 & b_i  & b_i/p.c. \geq 0 & C \\
        \hline
        x_3  & \frac{1}{4}  & \frac{3}{2} & 1   & \frac{5}{4}  & \frac{1}{4}  & 0   & 0   & 200  & 200             & 8 \\
        x_6  & 3            & -3          & 0   & 0            & -2           & 1   & 0   & 900  & \frac{625}{2}   & 0 \\
        x_7  & -\frac{1}{4} & \frac{3}{2} & 0   & \frac{15}{4} & -\frac{1}{4} & 0   & 1   & 400  & 600             & 0 \\
        F(x) & 0            & 5           & 0   & 6            & 2            & 0   & 0   & 1600 & -               & - \\
        \hline
    \end{array}
\]

Считаем по формуле: \(y^* = C \cdot A^{-1}\)

Посчитаем значение \(y^*\):

\[
    y^* =\begin{pmatrix} 8 & 0 & 0 \end{pmatrix} \cdot \begin{pmatrix}
        \frac{1}{4}  & 0 & 0 \\
        -2           & 1 & 0 \\
        -\frac{1}{4} & 0 & 1
    \end{pmatrix} = \begin{pmatrix} 2 & 0 & 0 \end{pmatrix}
\]

Минимум функции достигается при \(y_1 = 2, y_2 = 0, y_3 = 0\).

Теперь посчитаем значение функции \(F^*\):

\[
    F^* = 800 \cdot 2 + 2500 \cdot 0 + 600 \cdot 0 = 1600
\]

Значения \(F \text{ и } F^*\) совпадают \(\implies\) задача решена правильно.

\textbf{Анализ результатов}

Подставим $ \mathbf{x^*} = \left(0; 0; 200; 0\right)$ в условия прямой задачи:

\[
    \begin{cases}
        0 + 6 \cdot 0 + 4 \cdot 200 + 5 \cdot 0 = 800                     \\
        5 \cdot 0 + 9 \cdot 0 + 8 \cdot 200 + 10 \cdot 0 = 1600 \leq 2500 \\
        3 \cdot 0 + 200 + 5 \cdot 0 = 200 \leq 600                        \\
    \end{cases}
\]

Второе и третье условия имеют строгий знак $<$, значит второй и третий ресурсы (пластмасса и резина) не являются дефицитными (остатки 900 и 400 соответственно).

Первое условие образует равенство $=$, значит первый ресурс (металл) дефицитен.

Подставим $ \mathbf{y^*} = \begin{pmatrix} 2 & 0 & 0 \end{pmatrix}$ в условия двойственной задачи:

\[
    \begin{cases}
        6 > 2  \\
        12 > 7 \\
        8 = 8  \\
        10 > 4
    \end{cases}
\]

Первое, второе и четвёртое условия имеют строгий знак $>$, следовательно, производить эти изделия экономически невыгодно.

Третье условие имеет равенство $=$, следовательно, двойственная оценка ресурса, используемого для изготовления продукта в точности равна доходам, а значит продукт выгодно производить.

Величина двойственных оценок показывает, насколько возрастает целевая функция при увеличении запасов дефицитного ресурса на единицу.
Увеличение запасов ресурса Р1 (металл) на единицу приведет к новому оптимальному плану.
Коэффициенты $A_B^{-1}$ показывают, что увеличение прибыли достигается засчет увеличения выпуска продукции $C$, при этом запасы пластмассы сократятся на $2$ единиц и запасы резины сократятся на $\frac{1}{2}$ единицы.


Ответ: $\mathbf{x^*} = \begin{pmatrix} 0 & 0 & 200 & 0 \end{pmatrix}, \mathbf{y^*} = \begin{pmatrix} 2 & 0 & 0 \end{pmatrix}$


\textbf{Анализ устойчивости двойственных оценок}

Определим интервалы устойчивости:
\[x^*_{B\text{нов}} = x_B + A_B^{-1} \cdot (b + \Delta b)\]
\[ A_B^{-1} \cdot (b + \Delta b) \geq 0 \]

\[
    A_B^{-1} \cdot (b + \Delta b) = \begin{pmatrix}
        \frac{1}{4}  & 0 & 0 \\
        -2           & 1 & 0 \\
        -\frac{1}{4} & 0 & 1
    \end{pmatrix} \cdot \begin{pmatrix}
        800 + \Delta b_1  \\
        2500 + \Delta b_2 \\
        600 + \Delta b_3
    \end{pmatrix} = \begin{pmatrix}
        200 + \frac{1}{4} \Delta b_1    \\
        900 - 2 \Delta b_1 + \Delta b_2 \\
        400 - \frac{1}{4} \Delta b_1 + \Delta b_3
    \end{pmatrix} \geq 0
\]

\newpage
Рассмотрим частные случаи:
\begin{enumerate}
    \item $\Delta b_1 \geq 0, \Delta b_2 = 0, \Delta b_3 = 0$:
          \[
              \begin{pmatrix}
                  200 + \frac{1}{4} \Delta b_1 \\
                  900 - 2 \Delta b_1           \\
                  400 - \frac{1}{4} \Delta b_1
              \end{pmatrix} \geq 0 \Leftrightarrow \begin{cases}
                  200 + \frac{1}{4} \Delta b_1 \geq 0 \\
                  900 - 2 \Delta b_1 \geq 0           \\
                  400 - \frac{1}{4} \Delta b_1 \geq 0
              \end{cases} \Leftrightarrow \begin{cases}
                  \Delta b_1 \geq -800 \\
                  \Delta b_1 \leq 450  \\
                  \Delta b_1 \leq 1600
              \end{cases} \Leftrightarrow -800 \leq \Delta b_1 \leq 450
          \]
          При увеличении запасов 1-го ресурса не более чем на 450 единиц и уменьшении его запасов не более чем на 800 единиц значение целевой функции не изменится.
    \item $\Delta b_1 = 0, \Delta b_2 \geq 0, \Delta b_3 = 0$:
          \[
              \begin{pmatrix}
                  200              \\
                  900 + \Delta b_2 \\
                  400
              \end{pmatrix} \geq 0
              \Leftrightarrow 900 + \Delta b_2 \geq 0 \Leftrightarrow \Delta b_2 \geq -900
          \]
          При уменьшении запасов 2-го ресурса не более чем на 900 единиц, при этом оптимальный план двойственной задачи не изменится.
    \item $\Delta b_1 = 0, \Delta b_2 = 0, \Delta b_3 \geq 0$:
          \[
              \begin{pmatrix}
                  200 \\
                  900 \\
                  400 + \Delta b_3
              \end{pmatrix} \geq 0
              \Leftrightarrow 400 + \Delta b_3 \geq 0 \Leftrightarrow \Delta b_3 \geq -400
          \]
          При уменьшении запасов 3-го ресурса не более чем на 400 единиц, при этом оптимальный план двойственной задачи не изменится.
\end{enumerate}

Предположим: $\Delta b_1 = 450, \Delta b_2 = -900, \Delta b_3 = 400$:

\[
    \begin{pmatrix}
        x_3^{\text{нов}} \\
        x_6^{\text{нов}} \\
        x_7^{\text{нов}}
    \end{pmatrix}
    =
    \begin{pmatrix}
        200 + \frac{1}{4} \cdot 450 \\
        900 - 2 \cdot 450 - 900     \\
        400 - \frac{1}{4} \cdot 450 + 400
    \end{pmatrix} = \begin{pmatrix}
        \frac{625}{2} \\
        0             \\
        \frac{1375}{2}
    \end{pmatrix} \geq 0
\]

Посчитаем новое значение целевой функции:

\[
    F = 8 \cdot \frac{625}{2} = 2500
\]

\newpage
\section{Задание №5}
\textbf{Условие задачи:}

Рассмотрим закрытую транспортную задачу размером \(5 \times 4\) с пятью поставщиками и четырьмя потребителями. Общий запас равен общему спросу.

\textbf{Данные задачи:}

\begin{itemize}
    \item \textbf{Запасы поставщиков} (в единицах товара):
          \[
              S_1 = 55, \quad S_2 = 75, \quad S_3 = 100, \quad S_4 = 60, \quad S_5 = 110
          \]
    \item \textbf{Потребности потребителей} (в единицах товара):
          \[
              D_1 = 90, \quad D_2 = 110, \quad D_3 = 80, \quad D_4 = 120
          \]
\end{itemize}

Проверим общий баланс:
\[
    S = S_1 + S_2 + S_3 + S_4 + S_5 = 400
\]
\[
    D = D_1 + D_2 + D_3 + D_4 = 400
\]

Так как общий запас равен общему спросу, задача является закрытой.

\textbf{Матрица стоимости транспортировки} (в таблице указана стоимость транспортировки единицы товара от поставщика \(S_i\) к потребителю \(D_j\)):

\[
    \begin{array}{c|ccccc|c}
                      & S_1 & S_2 & S_3 & S_4 & S_5 & \text{Потребности} \\
        \hline
        D_1           & 4   & 5   & 6   & 7   & 3   & 90                 \\
        D_2           & 8   & 1   & 3   & 4   & 6   & 110                \\
        D_3           & 6   & 4   & 9   & 3   & 5   & 80                 \\
        D_4           & 3   & 7   & 2   & 8   & 1   & 120                \\
        \hline
        \text{Запасы} & 55  & 75  & 100 & 60  & 110 &                    \\
    \end{array}
\]

Целевая функция:
\[
    F = 4x_{11} + 5x_{12} + 6x_{13} + ... + 2x_{44} + 8x_{45} + x_{46}
\]

Ограничения:

\[
    \begin{cases}
        x_{11} + x_{12} + x_{13} + x_{14} + x_{15} = 90  \\
        x_{21} + x_{22} + x_{23} + x_{24} + x_{25} = 110 \\
        x_{31} + x_{32} + x_{33} + x_{34} + x_{35} = 80  \\
        x_{41} + x_{42} + x_{43} + x_{44} + x_{45} = 120 \\
        x_{11} + x_{21} + x_{31} + x_{41} = 55           \\
        x_{12} + x_{22} + x_{32} + x_{42} = 75           \\
        x_{13} + x_{23} + x_{33} + x_{43} = 100          \\
        x_{14} + x_{24} + x_{34} + x_{44} = 60           \\
        x_{15} + x_{25} + x_{35} + x_{45} = 110
    \end{cases}
\]\\

Задача состоит в том, чтобы минимизировать общую стоимость (целевую функцию) транспортировки при соблюдении ограничений на запасы и потребности.

\textbf{Решение задачи}

\textbf{Метод северо-западного угла}

Метод северо-западного угла предполагает заполнение транспортной таблицы, начиная с левой верхней ячейки и двигаясь по строкам и столбцам. На каждом шаге распределяем максимум возможного количества товара в текущую ячейку, обновляя остатки.

\textbf{Шаги метода северо-западного угла:}

\begin{enumerate}
    \item Ячейка \((S_1, D_1)\): минимальное значение между \(55\) и \(90\) — это \(55\). Заполняем \(55\), обновляем \(S_1 = 0\), \(D_1 = 35\).
    \item Ячейка \((S_2, D_1)\): минимальное значение между \(75\) и \(35\) — это \(35\). Заполняем \(35\), обновляем \(S_2 = 40\), \(D_1 = 0\).
    \item Ячейка \((S_2, D_2)\): минимальное значение между \(40\) и \(110\) — это \(40\). Заполняем \(40\), обновляем \(S_2 = 0\), \(D_2 = 70\).
    \item Ячейка \((S_3, D_2)\): минимальное значение между \(100\) и \(70\) — это \(70\). Заполняем \(70\), обновляем \(S_3 = 30\), \(D_2 = 0\).
    \item Ячейка \((S_3, D_3)\): минимальное значение между \(30\) и \(80\) — это \(30\). Заполняем \(30\), обновляем \(S_3 = 0\), \(D_3 = 50\).
    \item Ячейка \((S_4, D_3)\): минимальное значение между \(60\) и \(50\) — это \(50\). Заполняем \(50\), обновляем \(S_4 = 10\), \(D_3 = 0\).
    \item Ячейка \((S_4, D_4)\): минимальное значение между \(10\) и \(120\) — это \(10\). Заполняем \(10\), обновляем \(S_4 = 0\), \(D_4 = 110\).
    \item Ячейка \((S_5, D_4)\): минимальное значение между \(110\) и \(110\) — это \(110\). Заполняем \(110\), обновляем \(S_5 = 0\), \(D_4 = 0\).
\end{enumerate}

\textbf{Итоговое распределение методом северо-западного угла:}

\[
    \begin{array}{c|ccccc|c}
                      & S_1  & S_2  & S_3  & S_4  & S_5   & \text{Потребности} \\
        \hline
        D_1           & 55^4 & 35^5 & 0^6  & 0^7  & 0^3   & 90                 \\
        D_2           & 0^8  & 40^1 & 70^3 & 0^4  & 0^6   & 110                \\
        D_3           & 0^6  & 0^4  & 30^9 & 50^3 & 0^5   & 80                 \\
        D_4           & 0^3  & 0^7  & 0^2  & 10^8 & 110^1 & 120                \\
        \hline
        \text{Запасы} & 55   & 75   & 100  & 60   & 110   &                    \\
    \end{array}
\]

\textbf{Вычисление общей стоимости}

Теперь рассчитаем общую стоимость транспортировки \(F\), используя полученное распределение:

\[
    F = 55 \cdot 4 + 35 \cdot 5 + 40 \cdot 1 + 70 \cdot 3 + 30 \cdot 9 + 50 \cdot 3 + 10 \cdot 8 + 110 \cdot 1 = 1255
\]

Итак, общая стоимость транспортировки составляет \(F = 1255\).

\textbf{Итоговое распределение \(X\)}

Итоговая матрица распределения \(X\):

\[
    X = \begin{pmatrix}
        55 & 35 & 0  & 0  & 0   \\
        0  & 40 & 70 & 0  & 0   \\
        0  & 0  & 30 & 50 & 0   \\
        0  & 0  & 0  & 10 & 110 \\
    \end{pmatrix}
\]

\textbf{Алгоритм метода минимального элемента}

Метод минимального элемента включает следующие шаги:

\begin{enumerate}
    \item Найти ячейку с наименьшей стоимостью в матрице \(C_{ij}\).

    \item Заполнить ячейку \((i, j)\) максимальным возможным количеством: \(\min(S_i, D_j)\).

    \item Обновить запасы и потребности, вычитая заполненное количество из соответствующих значений \(S_i\) и \(D_j\).

    \item Если потребность или запас равен нулю, вычеркнуть соответствующую строку или столбец.

    \item Повторить шаги 1-4, пока все потребности и запасы не будут удовлетворены.
\end{enumerate}

\textbf{Решение методом минимального элемента}

\[
    \begin{array}{c|ccccc|c}
                      & S_1  & S_2  & S_3  & S_4  & S_5  & \text{Потребности} \\
        \hline
        D_1           & 0^4  & 0^5  & 0^6  & 0^7  & 90^3 & 90                 \\
        D_2           & 0^8  & 75^1 & 35^3 & 0^4  & 0^6  & 110                \\
        D_3           & 0^6  & 0^4  & 0^9  & 60^3 & 20^5 & 80                 \\
        D_4           & 55^3 & 0^7  & 65^2 & 0^8  & 0^1  & 120                \\
        \hline
        \text{Запасы} & 55   & 75   & 100  & 60   & 110  &                    \\
    \end{array}
\]

\textbf{Вычисление общей стоимости}

Теперь рассчитаем общую стоимость транспортировки \(F\), используя полученное распределение:

\[
    F = 90 \cdot 3 + 75 + 35 \cdot 3 + 60 \cdot 3 + 20 \cdot 5 + 55 \cdot 3 + 65 \cdot 2 = 1025
\]

Итак, общая стоимость транспортировки составляет \(F = 1025\).

\textbf{Итоговое распределение \(X\)}

Итоговая матрица распределения \(X\):

\[
    X = \begin{pmatrix}
        0  & 0  & 0  & 0  & 90 \\
        0  & 75 & 35 & 0  & 0  \\
        0  & 0  & 0  & 60 & 20 \\
        55 & 0  & 65 & 0  & 0  \\
    \end{pmatrix}
\]

\textbf{Метод потенциалов}

Метод потенциалов используется для проверки оптимальности текущего распределения и нахождения улучшенного решения, если оно не оптимально.

Для базисных клеток используем условие \( U_i + V_j = C_{ij} \). Примем \( U_1 = 0 \) и вычислим остальные потенциалы.

Обозначение: \( C^x_y \), где С - количество поставляемого груза, x - цена за единицу, y - потенциал.

Составим таблицу:
\[
    \begin{array}{c|ccccc|c}
        \text{Потребности/Запасы} & 55_4     & 75_5     & 100_7                                 & 60_1                                  & 110_{-6} &     \\
        \hline
        90_0                      & 55^4_-   & 35^5_-   & 0^6_{-1}                              & 0^7_6                                 & 0^3_9    & D_1 \\
        110_{-4}                  & 0^8_8    & 40^1_-   & 70^3_-                                & 0^4_7                                 & 0^6_{16} & D_2 \\
        80_2                      & 0^6_0    & 0^4_{-3} & \cellcolor[RGB]{102, 205, 170} 30^9_- & \cellcolor[RGB]{102, 205, 170} 50^3_- & 0^5_9    & D_3 \\
        120_7                     & 0^3_{-8} & 0^7_{-5} & \cellcolor{orange} 0^2_{-14}          & \cellcolor[RGB]{102, 205, 170} 10^8_- & 110^1_-  & D_4 \\
        \hline
                                  & S_1      & S_2      & S_3                                   & S_4                                   & S_5      &     \\
    \end{array}
\]

Есть потенциалы \( (< 0) \).\\
Найдем элемент с наименьшим потенциалом: \( (S_3, D_3) \).\\
Построим цикл зелёным цветом.\\
Проделаем перераспределение товаров и построим новую таблицу:
\[
    \begin{array}{c|ccccc|c}
        \text{Потребности/Запасы} & 55_4   & 75_5     & 100_7                                 & 60_1     & 110_6                                  &     \\
        \hline
        90_0                      & 55^4_- & 35^5_-   & 0^6_{-1}                              & 0^7_6    & 0^3_{-3}                               & D_1 \\
        110_{-4}                  & 0^8_8  & 40^1_-   & 70^3_-                                & 0^4_7    & 0^6_4                                  & D_2 \\
        80_2                      & 0^6_0  & 0^4_{-3} & \cellcolor[RGB]{102, 205, 170} 20^9_- & 60^3_-   & \cellcolor{orange} 0^5_{-3}            & D_3 \\
        120_{-5}                  & 0^3_4  & 0^7_0    & \cellcolor[RGB]{102, 205, 170} 10^2_- & 0^8_{12} & \cellcolor[RGB]{102, 205, 170} 110^1_- & D_4 \\
        \hline
                                  & S_1    & S_2      & S_3                                   & S_4      & S_5                                    &     \\
    \end{array}
\]
Есть потенциалы \( (< 0) \).\\
Найдем элемент с наименьшим потенциалом: \( (S_5, D_3) \).\\
Построим цикл зелёным цветом.\\
Проделаем перераспределение товаров и построим новую таблицу:
\[
    \begin{array}{c|ccccc|c}
        \text{Потребности/Запасы} & 55_4   & 75_5                                  & 100_7                                 & 60_4   & 110_6                                 &     \\
        \hline
        90_0                      & 55^4_- & \cellcolor[RGB]{102, 205, 170} 35^5_- & 0^6_{-1}                              & 0^7_3  & \cellcolor{orange} 0^3_{-3}           & D_1 \\
        110_{-4}                  & 0^8_8  & \cellcolor[RGB]{102, 205, 170} 40^1_- & \cellcolor[RGB]{102, 205, 170} 70^3_- & 0^4_0  & 0^6_4                                 & D_2 \\
        80_{-1}                   & 0^6_3  & 0^4_0                                 & 0^9_3                                 & 60^3_- & 20^5_-                                & D_3 \\
        120_{-5}                  & 0^3_4  & 0^7_7                                 & \cellcolor[RGB]{102, 205, 170} 30^2_- & 0^8_9  & \cellcolor[RGB]{102, 205, 170} 90^1_- & D_4 \\
        \hline
                                  & S_1    & S_2                                   & S_3                                   & S_4    & S_5                                   &     \\
    \end{array}
\]
Есть потенциалы \( (< 0) \).\\
Найдем элемент с наименьшим потенциалом: \( (S_5, D_1) \).\\
Построим цикл зелёным цветом.\\
Проделаем перераспределение товаров и построим новую таблицу:

\[
    \begin{array}{c|ccccc|c}
        \text{Потребности/Запасы} & 55_4   & 75_5                                  & 100_7                                 & 60_1   & 110_3                                 &     \\
        \hline
        90_0                      & 55^4_- & 0^5_0                                 & 0^6_{-1}                              & 0^7_6  & 35^3_-                                & D_1 \\
        110_{-4}                  & 0^8_8  & \cellcolor[RGB]{102, 205, 170} 75^1_- & \cellcolor[RGB]{102, 205, 170} 35^3_- & 0^4_7  & 0^6_7                                 & D_2 \\
        80_2                      & 0^6_0  & \cellcolor{orange} 0^4_{-3}           & 0^9_0                                 & 60^3_- & \cellcolor[RGB]{102, 205, 170} 20^5_- & D_3 \\
        120_{-2}                  & 0^3_1  & 0^7_4                                 & \cellcolor[RGB]{102, 205, 170} 65^2_- & 0^8_9  & \cellcolor[RGB]{102, 205, 170} 55^1_0 & D_4 \\
        \hline
                                  & S_1    & S_2                                   & S_3                                   & S_4    & S_5                                   &     \\
    \end{array}
\]
Есть потенциалы \( (< 0) \).\\
Найдем элемент с наименьшим потенциалом: \( (S_5, D_1) \).\\
Построим цикл зелёным цветом.\\
Проделаем перераспределение товаров и построим новую таблицу:

\[
    \begin{array}{c|ccccc|c}
        \text{Потребности/Запасы} & 55_4   & 75_2   & 100_4  & 60_1   & 110_3  &     \\
        \hline
        90_0                      & 55^4_- & 0^5_3  & 0^6_2  & 0^7_6  & 35^3_- & D_1 \\
        110_{-1}                  & 0^8_5  & 55^1_- & 55^3_- & 0^4_4  & 0^6_4  & D_2 \\
        80_2                      & 0^6_0  & 20^4_- & 0^9_3  & 60^3_- & 0^5_0  & D_3 \\
        120_{-2}                  & 0^3_1  & 0^7_7  & 45^2_- & 0^8_9  & 75^1_- & D_4 \\
        \hline
                                  & S_1    & S_2    & S_3    & S_4    & S_5    &     \\
    \end{array}
\]

Все потенциалы \( (\geq 0) \) оптимальный план найден.\\

\[
    F = 55 \cdot 4 + 35 \cdot 3 + 55 + 55 \cdot 3 + 20 \cdot 4 + 60 \cdot 3 + 45 \cdot 2 + 75 = 970
\]

Итак, общая стоимость транспортировки составляет \(F = 970\).

Итоговая матрица распределения \(X\):

\[
    X = \begin{pmatrix}
        55 & 0  & 0  & 0  & 35 \\
        0  & 55 & 55 & 0  & 0  \\
        0  & 20 & 0  & 60 & 0  \\
        0  & 0  & 45 & 0  & 75 \\
    \end{pmatrix}
\]

Используя код на Python:

\begin{verbatim}
from cvxopt.modeling import variable, op
import time

start = time.time()

# Переменные
x = variable(20, 'x')

# Стоимости
c = [4, 8, 6, 3, 5, 1, 4, 7, 6, 3, 9, 2, 7, 4, 3, 8, 3, 6, 5, 1]

# Целевая функция
z = sum(c[i] * x[i] for i in range(20))

# Ограничения
supply = [55, 75, 100, 60, 110]
demand = [90, 110, 80, 120]

constraints = []
for i in range(5):
    constraints.append(sum(x[i * 4 + j] for j in range(4)) <= supply[i])

for j in range(4):
    constraints.append(sum(x[i * 4 + j] for i in range(5)) == demand[j])

x_non_negative = (x >= 0)
constraints.append(x_non_negative)

# Постановка задачи
problem = op(z, constraints)

# Решение задачи
problem.solve(solver='glpk')

# Вывод результатов
print("Результат Xopt:")
for i in x.value:
    print(i)

print("Стоимость доставки:")
print(problem.objective.value()[0])

stop = time.time()
print("Время:")
print(stop - start)
\end{verbatim}

Получаем такие же значения:
\begin{verbatim}
GLPK Simplex Optimizer 5.0
29 rows, 20 columns, 60 non-zeros
      0: obj =   0.000000000e+00 inf =   4.000e+02 (4)
      8: obj =   1.255000000e+03 inf =   0.000e+00 (0)
*    20: obj =   9.700000000e+02 inf =   0.000e+00 (0)
OPTIMAL LP SOLUTION FOUND
Result Xopt:
 55   0   0   0 
  0  55  20   0 
  0  55   0  45 
  0   0  60   0 
 35   0   0  75 
Cost: 970.0
Time:0.01
\end{verbatim}

\text{Результаты совапали.}

\newpage
\section{Задание №6}

Оптимизационная задача:
\[
    F = 3x_1 + 4x_2 \to \max
\]
Ограничения:
\[
    \begin{cases}
        x_1 - 3 \cdot x_2 \leq 3  \\
        x_1 + x_2 \leq 10         \\
        -x_1 + 4 \cdot x_2 \leq 4 \\
        x_1 \geq 0, \ x_2 \geq 0
    \end{cases}
\]
\subsection{Графическое решение задачи.}
\begin{figure}[h]
    \centering
    \begin{tikzpicture}
        \begin{axis}[
                xlabel={$x_1$},
                ylabel={$x_2$},
                axis lines=middle,
                xmin=0, xmax=10,
                ymin=0, ymax=10,
                grid=both,
                width=14cm,
                height=12cm,
                legend pos=north east
            ]
            % Линии ограничений с разными цветами
            \addplot[name path=A, domain=0:10, thick, blue] { (x - 3)/3 };
            \addplot[name path=B, domain=0:10, thick, red] { 10 - x };
            \addplot[name path=C, domain=0:10, thick, green] { (4 + x)/4 };

            % Область допустимых решений
            \addplot[fill opacity=0.2, color=gray] fill between[
                    of=A and B,
                    soft clip={domain=6:10}
                ];

            % Вектор {3, 4}
            \addplot[->, thick, black] coordinates {(0,0) (3,4)};
            \node at (axis cs: 3,4) [anchor=west] {$\vec{v} = \{3,4\}$};

            % Подписи
            \legend{
                $x_1 - 3x_2 = 3$,
                $x_1 + x_2 = 10$,
                $-x_1 + 4x_2 = 4$
            }

        \end{axis}
    \end{tikzpicture}

    \caption{Графическое решение задачи}
\end{figure}

Вычисление точек пересечения

\(x_1 + x_2 = 10\) и \(-x_1 + 4x_2 = 4\):
\[
    \begin{cases}
        x_1 + x_2 = 10 \\
        -x_1 + 4x_2 = 4
    \end{cases}
    \implies
    \begin{cases}
        x_1 = \frac{36}{5} \\
        x_2 = \frac{14}{5}
    \end{cases}
\]
Точка пересечения: \(\left(\frac{36}{5}, \frac{14}{5}\right)\).

Вычисление значений целевой функции \(F = 3x_1 + 4x_2\):

\(\left(\frac{36}{5}, \frac{14}{5}\right)\):
\[
    F\left(\frac{36}{5}, \frac{14}{5}\right) = \frac{164}{5}
\]

Ответ:
\begin{tabular}{l l}
    Максимум функции \(F = 3x_1 + 4x_2\) & достигается в точке \(\left(\frac{36}{5}, \frac{14}{5}\right)\), где \(F = \frac{164}{5}\).
\end{tabular}\\

\subsection{Решение задачи с помощью симплекс-метода:}

Оптимизационная задача:
\[
    F = 3x_1 + 4x_2 \to \max
\]
Ограничения:
\[
    \begin{cases}
        x_1 - 3 \cdot x_2 \leq 3  \\
        x_1 + x_2 \leq 10         \\
        -x_1 + 4 \cdot x_2 \leq 4 \\
        x_1 \geq 0, \ x_2 \geq 0
    \end{cases}
\]

Решение задачи с помощью симплекс-метода:
\[
    F - 3x_1 - 4x_2 = 0
\]

\[
    \begin{cases}
        x_1 - 3 \cdot x_2 + x_3 = 3  \\
        x_1 + x_2 + x_4 = 10         \\
        -x_1 + 4 \cdot x_2 + x_5 = 4 \\
        x_1 \geq 0, \ x_2 \geq 0, \ x_3 \geq 0, \ x_4 \geq 0, \ x_5 \geq 0
    \end{cases}
\]

\vspace{25pt}

\[
    \begin{array}{|c|c|>{\columncolor{green}}c|c|c|c|c|c|}
        \hline
             & x_1 & x_2 & x_3 & x_4 & x_5 & b_i & b_i/p.c. \geq 0 \\
        \hline
        x_3  & 1   & -3  & 1   & 0   & 0   & 3   & -1              \\
        x_4  & 1   & 1   & 0   & 1   & 0   & 10  & 10              \\
        \rowcolor{yellow}
        x_5  & -1  & 4   & 0   & 0   & 1   & 4   & 1               \\
        F(x) & -3  & -4  & 0   & 0   & 0   & 0   & 0               \\
        \hline
    \end{array}
\]

В последней строке есть элементы \(\leq 0\). Занулим элементы выше и ниже стоящие от разрешающего элемента.

\[
    \begin{array}{|c|>{\columncolor{green}}c|c|c|c|c|c|c|}
        \hline
             & x_1          & x_2 & x_3 & x_4 & x_5          & b_i & b_i/p.c. \geq 0 \\
        \hline
        x_2  & -\frac{1}{4} & 1   & 0   & 0   & \frac{1}{4}  & 1   & -4              \\
        x_3  & \frac{1}{4}  & 0   & 1   & 0   & \frac{3}{4}  & 6   & 24              \\
        \rowcolor{yellow}
        x_4  & \frac{5}{4}  & 0   & 0   & 1   & -\frac{1}{4} & 9   & \frac{36}{5}    \\
        F(x) & -4           & 0   & 0   & 0   & 1            & 4   &                 \\
        \hline
    \end{array}
\]

В последней строке есть элементы \(\leq 0\). Занулим элементы выше и ниже стоящие от разрешающего элемента.

\[
    \begin{array}{|c|c|c|c|c|c|c|c|}
        \hline
             & x_1 & x_2 & x_3 & x_4          & x_5          & b_i           & b_i/p.c. \geq 0 \\
        \hline
        x_1  & 1   & 0   & 0   & \frac{4}{5}  & -\frac{1}{5} & \frac{36}{5}  & \frac{36}{5}    \\
        x_2  & 0   & 1   & 0   & \frac{1}{5}  & \frac{4}{20} & \frac{14}{5}  & -               \\
        x_3  & 0   & 0   & 1   & -\frac{1}{5} & \frac{4}{5}  & \frac{21}{5}  & 24              \\
        F(x) & 0   & 0   & 0   & \frac{16}{5} & \frac{1}{5}  & \frac{164}{5} &                 \\
        \hline
    \end{array}
\]

В последней строке не осталось элементов \(\leq 0\). Мы пришли к конечной таблице.\\
Максимум функции достигается при \(x_1 = \frac{36}{5}, x_2 = \frac{14}{5}\), и значение целевой функции равно \(F(x) = \frac{164}{5}\).

\newpage
\subsection{Решение задачи методом отсечения Гомори:}

\subsubsection{Геометрическим методом:}

\begin{figure}[h]
    \centering
    \begin{tikzpicture}
        \begin{axis}[
                xlabel={$x_1$},
                ylabel={$x_2$},
                axis lines=middle,
                xmin=0, xmax=10,
                ymin=0, ymax=10,
                grid=both,
                width=14cm,
                height=12cm,
                legend pos=north east
            ]
            % Линии ограничений с разными цветами
            \addplot[name path=A, domain=0:10, thick, blue] { (x - 3)/3 };
            \addplot[name path=B, domain=0:10, thick, red] { 10 - x };
            \addplot[name path=C, domain=0:10, thick, green] { (4 + x)/4 };

            % Область допустимых решений
            \addplot[fill opacity=0.2, color=gray] fill between[
                    of=A and B,
                    soft clip={domain=6:10}
                ];

            % Вектор {3, 4}
            \addplot[->, thick, black] coordinates {(0,0) (3,4)};
            \node at (axis cs: 3,4) [anchor=west] {$\vec{v} = \{3,4\}$};

            % Подписи
            \legend{
                $x_1 - 3x_2 = 3$,
                $x_1 + x_2 = 10$,
                $-x_1 + 4x_2 = 4$
            }

            \addplot[mark=*, red] coordinates {(8, 2)};
            \node at (axis cs:8,2) [anchor=west] {$(8, 2)$};
            \addplot[mark=*, red] coordinates {(7, 2)};
            \addplot[mark=*, red] coordinates {(6, 2)};
            \addplot[mark=*, red] coordinates {(6, 1)};
            \addplot[mark=*, red] coordinates {(5, 2)};
            \addplot[mark=*, red] coordinates {(5, 1)};
            \addplot[mark=*, red] coordinates {(4, 2)};
            \addplot[mark=*, red] coordinates {(4, 1)};
            \addplot[mark=*, red] coordinates {(3, 0)};
            \addplot[mark=*, red] coordinates {(3, 1)};
            \addplot[mark=*, red] coordinates {(2, 0)};
            \addplot[mark=*, red] coordinates {(2, 1)};
            \addplot[mark=*, red] coordinates {(1, 0)};
            \addplot[mark=*, red] coordinates {(1, 1)};

        \end{axis}
    \end{tikzpicture}

    \caption{Графическое решение задачи}
\end{figure}

Решение:
\[
    \begin{cases}
        x_1 = \frac{36}{5}, \\
        x_2 = \frac{14}{5}.
    \end{cases}
\]

Целевая функция:
\[
    F\left(\frac{36}{5}, \frac{14}{5}\right) = 3 \cdot \frac{36}{5} + 4 \cdot \frac{14}{5} = \frac{164}{5}.
\]

Решение:
\[
    \begin{cases}
        x_1 = 8, \\
        x_2 = 2.
    \end{cases}
\]

Целевая функция:
\[
    F(8, 2) = 3 \cdot 8 + 4 \cdot 2 = 32.
\]

\textbf{Ответ:} Максимум функции \(F = 3x_1 + 4x_2\) с учетом целочисленных ограничений достигается в точке \((8, 2)\), где \(F = 32\).

\subsubsection{Симплекс-методом:}

Добавляем дополнительные переменные \(x_3, x_4, x_5\) для приведения ограничений к равенствам:
\[
    \begin{cases}
        x_1 - 3x_2 + x_3        & = 3    \\
        x_1 + x_2 + x_4         & = 10   \\
        -x_1 + 4x_2 + x_5       & = 4    \\
        x_1, x_2, x_3, x_4, x_5 & \geq 0
    \end{cases}
\]

Целевая функция:
\[
    F = -3x_1 - 4x_2 \to \min.
\]


\textbf{Конечная симплекс-таблица:}
\[
    \begin{array}{|c|c|c|c|c|c|c|}
        \hline
            & x_1 & x_2 & x_3 & x_4          & x_5          & b             \\ \hline
        x_1 & 1   & 0   & 0   & \frac{4}{5}  & -\frac{1}{5} & \frac{36}{5}  \\ \hline
        x_2 & 0   & 1   & 0   & \frac{1}{5}  & \frac{1}{5}  & \frac{14}{5}  \\ \hline
        x_3 & 0   & 0   & 1   & -\frac{1}{5} & \frac{4}{5}  & \frac{21}{5}  \\ \hline
        F   & 0   & 0   & 0   & \frac{16}{5} & \frac{1}{5}  & \frac{164}{5} \\ \hline
    \end{array}
\]

Найдено нецелочисленное решение: \(x_1 = \frac{36}{5}\), \(x_2 = \frac{14}{5}\), \(F = \frac{164}{5}\).

Найдено оптимальное нецелочисленное решение. Среди свободных членов находим переменную с максимальным дробным числом:

\[
    x_1 = \frac{36}{5} = 1\frac{1}{5}, \quad x_2 = \frac{14}{5} = 2\frac{4}{5}
\]

Переменная \(x_2\) имеет максимальное дробное значение. Поэтому вводим дополнительное ограничение по 2 строке:

\[
    \begin{array}{|c|c|c|c|c|c|c|}
        \hline
            & x_1 & x_2 & x_3 & x_4          & x_5          & b             \\ \hline
        x_1 & 1   & 0   & 0   & \frac{4}{5}  & -\frac{1}{5} & \frac{36}{5}  \\ \hline
        \rowcolor{yellow}
        x_2 & 0   & 1   & 0   & \frac{1}{5}  & \frac{1}{5}  & \frac{14}{5}  \\ \hline
        x_3 & 0   & 0   & 1   & -\frac{1}{5} & \frac{4}{5}  & \frac{21}{5}  \\ \hline
        F   & 0   & 0   & 0   & \frac{16}{5} & \frac{1}{5}  & \frac{164}{5} \\ \hline
    \end{array}
\]

Записываем новое ограничение:

\[
    -\frac{4}{5} = -0x_1 - 0x_2 - 0x_3 - \frac{1}{5}x_4 - \frac{1}{5}x_5 + x_6
\]

\textbf{Обновлённая таблица:}
\[
    \begin{array}{|c|c|c|c|c|c|c|c|}
        \hline
                 & b             & x_1 & x_2 & x_3 & x_4          & x_5          & x_6 \\ \hline
        x_1      & \frac{36}{5}  & 1   & 0   & 0   & \frac{4}{5}  & -\frac{1}{5} & 0   \\ \hline
        x_2      & \frac{14}{5}  & 0   & 1   & 0   & \frac{1}{5}  & \frac{1}{5}  & 0   \\ \hline
        x_3      & \frac{21}{5}  & 0   & 0   & 1   & -\frac{1}{5} & \frac{4}{5}  & 0   \\ \hline
        x_1      & -\frac{4}{5}  & 0   & 0   & 0   & -\frac{1}{5} & -\frac{1}{5} & 1   \\ \hline
        F_{\max} & \frac{164}{5} & 0   & 0   & 0   & \frac{16}{5} & \frac{1}{5}  & 0   \\ \hline
    \end{array}
\]



Т.к. среди свободных членов есть отрицательные значения, то решение недопустимое, и сначала нужно перейти к допустимому решению. Для этого находим среди свободных членов максимальное отрицательное число по модулю. Это число будет задавать разрешающую (ведущую) строку.

В этой строке так же находим максимальный по модулю отрицательный элемент, который будет разрешающим (ведущим) столбцом.\\

\textbf{Разрешающий столбец:} \(x_4\)

\textbf{Разрешающая строка:} \(x_1\)\\


\textbf{Пересчитываем таблицу:}

\[
    \begin{array}{|c|c|c|c|c|>{\columncolor{green}}c|c|c|c|}
        \hline
                 & b             & x_1 & x_2 & x_3 & x_4          & x_5          & x_6 & \frac{b}{x_4} \\ \hline
        x_1      & \frac{36}{5}  & 1   & 0   & 0   & \frac{4}{5}  & -\frac{1}{5} & 0   & 9             \\ \hline
        x_2      & \frac{14}{5}  & 0   & 1   & 0   & \frac{1}{5}  & \frac{1}{5}  & 0   & 14            \\ \hline
        x_3      & \frac{21}{5}  & 0   & 0   & 1   & -\frac{1}{5} & \frac{4}{5}  & 0   & -21           \\ \hline
        \rowcolor{yellow}
        x_1      & -\frac{4}{5}  & 0   & 0   & 0   & -\frac{1}{5} & -\frac{1}{5} & 1   & 4             \\ \hline
        F_{\max} & \frac{164}{5} & 0   & 0   & 0   & \frac{16}{5} & \frac{1}{5}  & 0   &               \\ \hline
    \end{array}
\]

\textbf{Пересчитываем таблицу:}

\[
    \begin{array}{|c|c|c|c|c|c|c|c|}
        \hline
                 & b  & x_1 & x_2 & x_3 & x_4 & x_5 & x_6 \\ \hline
        x_1      & 4  & 1   & 0   & 0   & 0   & -1  & 4   \\ \hline
        x_2      & 2  & 0   & 1   & 0   & 0   & 0   & 1   \\ \hline
        x_3      & 5  & 0   & 0   & 1   & 0   & 1   & -1  \\ \hline
        x_4      & 4  & 0   & 0   & 0   & 1   & 1   & -5  \\ \hline
        F_{\max} & 20 & 0   & 0   & 0   & 0   & -3  & 16  \\ \hline
    \end{array}
\]

\textbf{Правило выбора разрешающего элемента:}

Среди коэффициентов целевой функции выбираем максимальный по модулю отрицательный элемент. Этот элемент определяет разрешающий столбец.

Разрешающая строка выбирается так, чтобы отношение свободного члена к элементу, находящемуся на пересечении разрешающего столбца и строки, было минимальным и неотрицательным.\\
Разрешающий столбец: \( x_5 \)\\
Разрешающая строка: \( x_4 \)\\

\[
    \begin{array}{|c|c|c|c|c|c|>{\columncolor{green}}c|c|c|}
        \hline
                 & b  & x_1 & x_2 & x_3 & x_4 & x_5 & x_6 & \frac{b}{x_5} \\ \hline
        x_1      & 4  & 1   & 0   & 0   & 0   & -1  & 4   & -4            \\ \hline
        x_2      & 2  & 0   & 1   & 0   & 0   & 0   & 1   & -             \\ \hline
        x_3      & 5  & 0   & 0   & 1   & 0   & 1   & -1  & 5             \\ \hline
        \rowcolor{yellow}
        x_4      & 4  & 0   & 0   & 0   & 1   & 1   & -5  & 4             \\ \hline
        F_{\max} & 20 & 0   & 0   & 0   & 0   & -3  & 16  &               \\ \hline
    \end{array}
\]

\textbf{Пересчитываем таблицу:}

\[
    \begin{array}{|c|c|c|c|c|c|c|c|}
        \hline
                 & b  & x_1 & x_2 & x_3 & x_4 & x_5 & x_6 \\ \hline
        x_1      & 8  & 1   & 0   & 0   & 1   & 0   & -1  \\ \hline
        x_2      & 2  & 0   & 1   & 0   & 0   & 0   & 1   \\ \hline
        x_3      & 1  & 0   & 0   & 1   & -1  & 0   & 4   \\ \hline
        x_5      & 4  & 0   & 0   & 0   & 1   & 1   & -5  \\ \hline
        F_{\max} & 32 & 0   & 0   & 0   & 3   & 0   & 1   \\ \hline
    \end{array}
\]

Так как все коэффициенты при целевой функции неотрицательны, решение оптимально.

\textbf{Значения переменных:}
\[
    x_1 = 8, \quad x_2 = 2
\]

\textbf{Значение целевой функции:}
\[
    F_{\max}(x) = 32
\]

\section{Задание №7}
\subsection{Условие}
Придумать задачу коммивояжера размерности $10 \times 10$. Значения в матрице расстояний должны быть любыми целыми числами от 1 до 100.
Решить задачу методом ветвей и границ. Полный перебор не использовать.
После выполнения задания добавить в отчёт граф решения, добавить решение задачи с помощью программных средств.

\section*{Постановка задачи}

Рассмотрим задачу коммивояжера для $10$ городов. Пусть города обозначены номерами от $1$ до $10$. Задана матрица расстояний $C = (c_{ij})$, где $c_{ij}$ — расстояние между городами $i$ и $j$. Требуется найти минимальный замкнутый путь, проходящий через каждый город ровно один раз.

\textbf{Матрица расстояний:}
\[
    C = \begin{pmatrix}
        \infty & 29     & 20     & 21     & 16     & 31     & 100    & 12     & 4      & 31     \\
        29     & \infty & 15     & 29     & 28     & 40     & 72     & 21     & 29     & 41     \\
        20     & 15     & \infty & 15     & 14     & 25     & 81     & 9      & 23     & 27     \\
        21     & 29     & 15     & \infty & 4      & 12     & 92     & 12     & 25     & 13     \\
        16     & 28     & 14     & 4      & \infty & 16     & 94     & 9      & 20     & 16     \\
        31     & 40     & 25     & 12     & 16     & \infty & 95     & 24     & 36     & 3      \\
        100    & 72     & 81     & 92     & 94     & 95     & \infty & 90     & 101    & 99     \\
        12     & 21     & 9      & 12     & 9      & 24     & 90     & \infty & 15     & 25     \\
        4      & 29     & 23     & 25     & 20     & 36     & 101    & 15     & \infty & 35     \\
        31     & 41     & 27     & 13     & 16     & 3      & 99     & 25     & 35     & \infty
    \end{pmatrix}.
\]

Здесь $\infty$ обозначает отсутствие дуги между городом $i$ и самим собой.

\section*{Метод решения: ветви и границы}

Метод ветвей и границ используется для эффективного решения задач дискретной оптимизации. Основная идея заключается в построении дерева решений, где каждая ветвь представляет собой подзадачу, а границы (оценки) позволяют исключить невыгодные подзадачи.

1. Для исходной матрицы $C$ выполните \textbf{редукцию строк и столбцов}:
- Для каждой строки вычтем минимальный элемент этой строки из всех её элементов.
- Для каждого столбца вычтите минимальный элемент этого столбца из всех его элементов.

\[
    \begin{array}{c|cccccccccc|c}
        i \setminus j & 1      & 2      & 3      & 4      & 5      & 6      & 7      & 8      & 9      & 10     & d_i \\
        \hline
        1             & \infty & 29     & 20     & 21     & 16     & 31     & 100    & 12     & 4      & 31     & 4   \\
        2             & 29     & \infty & 15     & 29     & 28     & 40     & 72     & 21     & 29     & 41     & 15  \\
        3             & 20     & 15     & \infty & 15     & 14     & 25     & 81     & 9      & 23     & 27     & 9   \\
        4             & 21     & 29     & 15     & \infty & 4      & 12     & 92     & 12     & 25     & 13     & 4   \\
        5             & 16     & 28     & 14     & 4      & \infty & 16     & 94     & 9      & 20     & 16     & 4   \\
        6             & 31     & 40     & 25     & 12     & 16     & \infty & 95     & 24     & 36     & 3      & 3   \\
        7             & 100    & 72     & 81     & 92     & 94     & 95     & \infty & 90     & 101    & 99     & 72  \\
        8             & 12     & 21     & 9      & 12     & 9      & 24     & 90     & \infty & 15     & 25     & 9   \\
        9             & 4      & 29     & 23     & 25     & 20     & 36     & 101    & 15     & \infty & 35     & 4   \\
        10            & 31     & 41     & 27     & 13     & 16     & 3      & 99     & 25     & 35     & \infty & 3   \\
    \end{array}
\]
Затем вычитаем $d_i$ из элементов рассматриваемой строки. В связи с этим во вновь полученной матрице в каждой строке будет как минимум один ноль.

\[
    \begin{array}{c|cccccccccc}
        i \setminus j & 1      & 2      & 3      & 4      & 5      & 6      & 7      & 8      & 9      & 10     \\
        \hline
        1             & \infty & 25     & 16     & 17     & 12     & 27     & 96     & 8      & 0      & 27     \\
        2             & 14     & \infty & 0      & 14     & 13     & 25     & 57     & 6      & 14     & 26     \\
        3             & 11     & 6      & \infty & 6      & 5      & 16     & 72     & 0      & 14     & 18     \\
        4             & 17     & 25     & 11     & \infty & 0      & 8      & 88     & 8      & 21     & 9      \\
        5             & 12     & 24     & 10     & 0      & \infty & 12     & 90     & 5      & 16     & 12     \\
        6             & 28     & 37     & 22     & 9      & 13     & \infty & 92     & 21     & 33     & 0      \\
        7             & 28     & 0      & 9      & 20     & 22     & 23     & \infty & 18     & 29     & 27     \\
        8             & 3      & 12     & 0      & 3      & 0      & 15     & 81     & \infty & 6      & 16     \\
        9             & 0      & 25     & 19     & 21     & 16     & 32     & 97     & 11     & \infty & 31     \\
        10            & 28     & 38     & 24     & 10     & 13     & 0      & 96     & 22     & 32     & \infty \\
    \end{array}
\]

Такую же операцию редукции проводим по столбцам, для чего в каждом столбце находим минимальный элемент:
\[ d_j = \min_i d_{ij} \]

\[
    \begin{array}{c|cccccccccc}
        i \setminus j & 1      & 2      & 3      & 4      & 5      & 6      & 7      & 8      & 9      & 10     \\
        \hline
        1             & \infty & 25     & 16     & 17     & 12     & 27     & 96     & 8      & 0      & 27     \\
        2             & 14     & \infty & 0      & 14     & 13     & 25     & 57     & 6      & 14     & 26     \\
        3             & 11     & 6      & \infty & 6      & 5      & 16     & 72     & 0      & 14     & 18     \\
        4             & 17     & 25     & 11     & \infty & 0      & 8      & 88     & 8      & 21     & 9      \\
        5             & 12     & 24     & 10     & 0      & \infty & 12     & 90     & 5      & 16     & 12     \\
        6             & 28     & 37     & 22     & 9      & 13     & \infty & 92     & 21     & 33     & 0      \\
        7             & 28     & 0      & 9      & 20     & 22     & 23     & \infty & 18     & 29     & 27     \\
        8             & 3      & 12     & 0      & 3      & 0      & 15     & 81     & \infty & 6      & 16     \\
        9             & 0      & 25     & 19     & 21     & 16     & 32     & 97     & 11     & \infty & 31     \\
        10            & 28     & 38     & 24     & 10     & 13     & 0      & 96     & 22     & 32     & \infty \\
        \hline
        d_j           & 0      & 0      & 0      & 0      & 0      & 0      & 57     & 0      & 0      & 0      \\
    \end{array}
\]

Получаем:

\[
    \begin{array}{c|cccccccccc}
        i \setminus j & 1      & 2      & 3      & 4      & 5      & 6      & 7      & 8      & 9      & 10     \\
        \hline
        1             & \infty & 25     & 16     & 17     & 12     & 27     & 96     & 8      & 0      & 27     \\
        2             & 14     & \infty & 0      & 14     & 13     & 25     & 57     & 6      & 14     & 26     \\
        3             & 11     & 6      & \infty & 6      & 5      & 16     & 72     & 0      & 14     & 18     \\
        4             & 17     & 25     & 11     & \infty & 0      & 8      & 88     & 8      & 21     & 9      \\
        5             & 12     & 24     & 10     & 0      & \infty & 12     & 90     & 5      & 16     & 12     \\
        6             & 28     & 37     & 22     & 9      & 13     & \infty & 92     & 21     & 33     & 0      \\
        7             & 28     & 0      & 9      & 20     & 22     & 23     & \infty & 18     & 29     & 27     \\
        8             & 3      & 12     & 0      & 3      & 0      & 15     & 81     & \infty & 6      & 16     \\
        9             & 0      & 25     & 19     & 21     & 16     & 32     & 97     & 11     & \infty & 31     \\
        10            & 28     & 38     & 24     & 10     & 13     & 0      & 96     & 22     & 32     & \infty \\
    \end{array}
\]

Сумма констант приведения определяет нижнюю границу $H$:
\[ H = \sum d_i + \sum d_j \]
\[ H = 4 + 15 + 9 + 4 + 4 + 3 + 72 + 9 + 4 + 3 + 0 + 0 + 0 + 0 + 0 + 0 + 57 + 0 + 0 + 0 = 184 \]

Элементы матрицы $d_{ij}$ соответствуют расстоянию от пункта $i$ до пункта $j$. Поскольку в матрице $n$ городов, то $D$ является матрицей $n \times n$ с неотрицательными элементами $d_{ij} \geq 0$. Каждый допустимый маршрут представляет собой цикл, по которому коммивояжер посещает город только один раз и возвращается в исходный город. Длина маршрута определяется выражением:
\[ F(M_k) = \sum d_{ij} \]
Причем каждая строка и столбец входят в маршрут только один раз с элементом $d_{ij}$.


Шаг №1.
Определяем ребро ветвления и разбиваем все множество маршрутов относительно этого ребра на два подмножества $(i,j)$ и $(i^*,j^*)$.
С этой целью для всех клеток матрицы с нулевыми элементами заменяем поочередно нули на $\infty$ и определяем для них сумму образовавшихся констант приведения, они приведены в скобках.

\[
    \begin{array}{c|cccccccccc|c}
        i \setminus j & 1      & 2      & 3      & 4      & 5      & 6      & 7      & 8      & 9      & 10     & d_i \\
        \hline
        1             & \infty & 25     & 16     & 17     & 12     & 27     & 39     & 8      & 0(14)  & 27     & 8   \\
        2             & 14     & \infty & 0(0)   & 14     & 13     & 25     & 0(15)  & 6      & 14     & 26     & 0   \\
        3             & 11     & 6      & \infty & 6      & 5      & 16     & 15     & 0(10)  & 14     & 18     & 5   \\
        4             & 17     & 25     & 11     & \infty & 0(8)   & 8      & 31     & 8      & 21     & 9      & 8   \\
        5             & 12     & 24     & 10     & 0(8)   & \infty & 12     & 33     & 5      & 16     & 12     & 5   \\
        6             & 28     & 37     & 22     & 9      & 13     & \infty & 35     & 21     & 33     & 0(18)  & 9   \\
        7             & 28     & 0(15)  & 9      & 20     & 22     & 23     & \infty & 18     & 29     & 27     & 9   \\
        8             & 3      & 12     & 0(0)   & 3      & 0(0)   & 15     & 24     & \infty & 6      & 16     & 0   \\
        9             & 0(14)  & 25     & 19     & 21     & 16     & 32     & 40     & 11     & \infty & 31     & 11  \\
        10            & 28     & 38     & 24     & 10     & 13     & 0(18)  & 39     & 22     & 32     & \infty & 10  \\
        \hline
        d_j           & 3      & 6      & 0      & 3      & 0      & 8      & 15     & 5      & 6      & 9      & 0   \\
    \end{array}
\]

\[
    d(1,9) = 8 + 6 = 14; \quad d(2,3) = 0 + 0 = 0; \quad d(2,7) = 0 + 15 = 15; \quad d(3,8) = 5 + 5 = 10; \quad d(4,5) = 8 + 0 = 8; \quad d(5,4) = 5 + 3 = 8; \quad d(6,10) = 9 + 9 = 18; \quad d(7,2) = 9 + 6 = 15; \quad d(8,3) = 0 + 0 = 0; \quad d(8,5) = 0 + 0 = 0; \quad d(9,1) = 11 + 3 = 14; \quad d(10,6) = 10 + 8 = 18;
\]

Наибольшая сумма констант приведения равна $(9 + 9) = 18$ для ребра $(6,10)$, следовательно, множество разбивается на два подмножества $(6,10)$ и $(6^*,10^*)$.


Исключение ребра $(6,10)$ проводим путем замены элемента $d_{6,10} = 0$ на $\infty$, после чего осуществляем очередное приведение матрицы расстояний для образовавшегося подмножества $(6^*,10^*)$, в результате получим редуцированную матрицу.

\[
    \begin{array}{c|cccccccccc|c}
        i \setminus j & 1      & 2      & 3      & 4      & 5      & 6      & 7      & 8      & 9      & 10     & d_i \\
        \hline
        1             & \infty & 25     & 16     & 17     & 12     & 27     & 39     & 8      & 0      & 27     & 0   \\
        2             & 14     & \infty & 0      & 14     & 13     & 25     & 0      & 6      & 14     & 26     & 0   \\
        3             & 11     & 6      & \infty & 6      & 5      & 16     & 15     & 0      & 14     & 18     & 0   \\
        4             & 17     & 25     & 11     & \infty & 0      & 8      & 31     & 8      & 21     & 9      & 0   \\
        5             & 12     & 24     & 10     & 0      & \infty & 12     & 33     & 5      & 16     & 12     & 0   \\
        6             & 28     & 37     & 22     & 9      & 13     & \infty & 35     & 21     & 33     & \infty & 9   \\
        7             & 28     & 0      & 9      & 20     & 22     & 23     & \infty & 18     & 29     & 27     & 0   \\
        8             & 3      & 12     & 0      & 3      & 0      & 15     & 24     & \infty & 6      & 16     & 0   \\
        9             & 0      & 25     & 19     & 21     & 16     & 32     & 40     & 11     & \infty & 31     & 0   \\
        10            & 28     & 38     & 24     & 10     & 13     & 0      & 39     & 22     & 32     & \infty & 0   \\
        \hline
        d_j           & 0      & 0      & 0      & 0      & 0      & 0      & 0      & 0      & 0      & 9      & 18  \\
    \end{array}
\]

Нижняя граница гамильтоновых циклов этого подмножества:
\[
    H(6^*,10^*) = 184 + 18 = 202
\]

Включение ребра $(6,10)$ проводится путем исключения всех элементов 6-ой строки и 10-го столбца, в которой элемент $d_{6,10}$ заменяем на $\infty$, для исключения образования негамильтонова цикла.
В результате получим другую сокращенную матрицу $(9 \times 9)$, которая подлежит операции приведения.
После операции приведения сокращенная матрица будет иметь вид:

\[
    \begin{array}{c|ccccccccc|c}
        i \setminus j & 1      & 2      & 3      & 4      & 5      & 6      & 7      & 8      & 9      & d_i \\
        \hline
        1             & \infty & 25     & 16     & 17     & 12     & 27     & 39     & 8      & 0      & 0   \\
        2             & 14     & \infty & 0      & 14     & 13     & 25     & 0      & 6      & 14     & 0   \\
        3             & 11     & 6      & \infty & 6      & 5      & 16     & 15     & 0      & 14     & 0   \\
        4             & 17     & 25     & 11     & \infty & 0      & 8      & 31     & 8      & 21     & 0   \\
        5             & 12     & 24     & 10     & 0      & \infty & 12     & 33     & 5      & 16     & 0   \\
        7             & 28     & 0      & 9      & 20     & 22     & 23     & \infty & 18     & 29     & 0   \\
        8             & 3      & 12     & 0      & 3      & 0      & 15     & 24     & \infty & 6      & 0   \\
        9             & 0      & 25     & 19     & 21     & 16     & 32     & 40     & 11     & \infty & 0   \\
        10            & 28     & 38     & 24     & 10     & 13     & \infty & 39     & 22     & 32     & 10  \\
        \hline
        d_j           & 0      & 0      & 0      & 0      & 0      & 8      & 0      & 0      & 0      & 18  \\
    \end{array}
\]

Сумма констант приведения сокращенной матрицы:
\[
    \sum d_i + \sum d_j = 18
\]

Нижняя граница подмножества $(6,10)$ равна:
\[
    H(6,10) = 184 + 18 = 202 \leq 202
\]\\
Поскольку нижние границы подмножества $(6,10)$ и подмножества $(6^*,10^*)$ равны, то ребро $(6,10)$ включаем в маршрут с новой границей $H=202$.

Шаг №2.
Определяем ребро ветвления.

\[
    \begin{array}{c|ccccccccc|c}
        i \setminus j & 1      & 2      & 3      & 4      & 5      & 6      & 7      & 8      & 9      & d_i \\
        \hline
        1             & \infty & 25     & 16     & 17     & 12     & 19     & 39     & 8      & 0(14)  & 8   \\
        2             & 14     & \infty & 0(0)   & 14     & 13     & 17     & 0(15)  & 6      & 14     & 0   \\
        3             & 11     & 6      & \infty & 6      & 5      & 8      & 15     & 0(10)  & 14     & 5   \\
        4             & 17     & 25     & 11     & \infty & 0(0)   & 0(4)   & 31     & 8      & 21     & 0   \\
        5             & 12     & 24     & 10     & 0(4)   & \infty & 4      & 33     & 5      & 16     & 4   \\
        7             & 28     & 0(15)  & 9      & 20     & 22     & 15     & \infty & 18     & 29     & 9   \\
        8             & 3      & 12     & 0(0)   & 3      & 0(0)   & 7      & 24     & \infty & 6      & 0   \\
        9             & 0(14)  & 25     & 19     & 21     & 16     & 24     & 40     & 11     & \infty & 11  \\
        10            & 18     & 28     & 14     & 0(3)   & 3      & \infty & 29     & 12     & 22     & 3   \\
        \hline
        d_j           & 3      & 6      & 0      & 0      & 0      & 4      & 15     & 5      & 6      & 0   \\
    \end{array}
\]

\[
    d(1,9) = 8 + 6 = 14; \quad d(2,3) = 0 + 0 = 0; \quad d(2,7) = 0 + 15 = 15; \quad d(3,8) = 5 + 5 = 10; \quad d(4,5) = 0 + 0 = 0; \quad d(4,6) = 0 + 4 = 4; \quad d(5,4) = 4 + 0 = 4; \quad d(7,2) = 9 + 6 = 15; \quad d(8,3) = 0 + 0 = 0; \quad d(8,5) = 0 + 0 = 0; \quad d(9,1) = 11 + 3 = 14; \quad d(10,4) = 3 + 0 = 3;
\]

Максимум: $d(2,7)=15$.

Исключение ребра $(2,7)$: $d_{2,7}=\infty$.

\[
    \begin{array}{c|ccccccccc|c}
        i \setminus j & 1      & 2      & 3      & 4      & 5      & 6      & 7      & 8      & 9      & d_i \\
        \hline
        1             & \infty & 25     & 16     & 17     & 12     & 19     & 39     & 8      & 0      & 0   \\
        2             & 14     & \infty & 0      & 14     & 13     & 17     & \infty & 6      & 14     & 0   \\
        3             & 11     & 6      & \infty & 6      & 5      & 8      & 15     & 0      & 14     & 0   \\
        4             & 17     & 25     & 11     & \infty & 0      & 0      & 31     & 8      & 21     & 0   \\
        5             & 12     & 24     & 10     & 0      & \infty & 4      & 33     & 5      & 16     & 0   \\
        7             & 28     & 0      & 9      & 20     & 22     & 15     & \infty & 18     & 29     & 0   \\
        8             & 3      & 12     & 0      & 3      & 0      & 7      & 24     & \infty & 6      & 0   \\
        9             & 0      & 25     & 19     & 21     & 16     & 24     & 40     & 11     & \infty & 0   \\
        10            & 18     & 28     & 14     & 0      & 3      & \infty & 29     & 12     & 22     & 0   \\
        \hline
        d_j           & 0      & 0      & 0      & 0      & 0      & 0      & 15     & 0      & 0      & 15  \\
    \end{array}
\]

Нижняя граница гамильтоновых циклов этого подмножества:
\[
    H(2^*,7^*) = 202 + 15 = 217
\]

Включение ребра $(2,7)$: $d_{7,2}=\infty$.

\[
    \begin{array}{c|cccccccc|c}
        i \setminus j & 1      & 2      & 3      & 4      & 5      & 6      & 8      & 9      & d_i \\
        \hline
        1             & \infty & 25     & 16     & 17     & 12     & 19     & 8      & 0      & 0   \\
        3             & 11     & 6      & \infty & 6      & 5      & 8      & 0      & 14     & 0   \\
        4             & 17     & 25     & 11     & \infty & 0      & 0      & 8      & 21     & 0   \\
        5             & 12     & 24     & 10     & 0      & \infty & 4      & 5      & 16     & 0   \\
        7             & 28     & \infty & 9      & 20     & 22     & 15     & 18     & 29     & 9   \\
        8             & 3      & 12     & 0      & 3      & 0      & 7      & \infty & 6      & 0   \\
        9             & 0      & 25     & 19     & 21     & 16     & 24     & 11     & \infty & 0   \\
        10            & 18     & 28     & 14     & 0      & 3      & \infty & 12     & 22     & 0   \\
        \hline
        d_j           & 0      & 6      & 0      & 0      & 0      & 0      & 0      & 0      & 15  \\
    \end{array}
\]

Сумма констант приведения сокращенной матрицы:
\[
    \sum d_i + \sum d_j = 15
\]

Нижняя граница подмножества $(2,7)$ равна:
\[
    H(2,7) = 202 + 15 = 217 \leq 217
\]

Ребро $(2,7)$ включаем в маршрут с новой границей $H=217$.\\
Шаг №3.
Определяем ребро ветвления.

\[
    \begin{array}{c|cccccccc|c}
        i \setminus j & 1      & 2      & 3      & 4      & 5      & 6      & 8      & 9      & d_i \\
        \hline
        1             & \infty & 19     & 16     & 17     & 12     & 19     & 8      & 0(14)  & 8   \\
        3             & 11     & 0(6)   & \infty & 6      & 5      & 8      & 0(5)   & 14     & 0   \\
        4             & 17     & 19     & 11     & \infty & 0(0)   & 0(4)   & 8      & 21     & 0   \\
        5             & 12     & 18     & 10     & 0(4)   & \infty & 4      & 5      & 16     & 4   \\
        7             & 19     & \infty & 0(6)   & 11     & 13     & 6      & 9      & 20     & 6   \\
        8             & 3      & 6      & 0(0)   & 3      & 0(0)   & 7      & \infty & 6      & 0   \\
        9             & 0(14)  & 19     & 19     & 21     & 16     & 24     & 11     & \infty & 11  \\
        10            & 18     & 22     & 14     & 0(3)   & 3      & \infty & 12     & 22     & 3   \\
        \hline
        d_j           & 3      & 6      & 0      & 0      & 0      & 4      & 5      & 6      & 0   \\
    \end{array}
\]

\[
    d(1,9) = 8 + 6 = 14; \quad d(3,2) = 0 + 6 = 6; \quad d(3,8) = 0 + 5 = 5; \quad d(4,5) = 0 + 0 = 0; \quad d(4,6) = 0 + 4 = 4; \quad d(5,4) = 4 + 0 = 4; \quad d(7,3) = 6 + 0 = 6; \quad d(8,3) = 0 + 0 = 0; \quad d(8,5) = 0 + 0 = 0; \quad d(9,1) = 11 + 3 = 14; \quad d(10,4) = 3 + 0 = 3;
\]

Максимум: $d(1,9)=14$.

Исключение ребра $(1,9)$: $d_{1,9}=\infty$.

\[
    \begin{array}{c|cccccccc|c}
        i \setminus j & 1      & 2      & 3      & 4      & 5      & 6      & 8      & 9      & d_i \\
        \hline
        1             & \infty & 19     & 16     & 17     & 12     & 19     & 8      & \infty & 8   \\
        3             & 11     & 0      & \infty & 6      & 5      & 8      & 0      & 14     & 0   \\
        4             & 17     & 19     & 11     & \infty & 0      & 0      & 8      & 21     & 0   \\
        5             & 12     & 18     & 10     & 0      & \infty & 4      & 5      & 16     & 0   \\
        7             & 19     & \infty & 0      & 11     & 13     & 6      & 9      & 20     & 0   \\
        8             & 3      & 6      & 0      & 3      & 0      & 7      & \infty & 6      & 0   \\
        9             & 0      & 19     & 19     & 21     & 16     & 24     & 11     & \infty & 0   \\
        10            & 18     & 22     & 14     & 0      & 3      & \infty & 12     & 22     & 0   \\
        \hline
        d_j           & 0      & 0      & 0      & 0      & 0      & 0      & 0      & 6      & 14  \\
    \end{array}
\]

Нижняя граница гамильтоновых циклов этого подмножества:
\[
    H(1^*,9^*) = 217 + 14 = 231
\]

Включение ребра $(1,9)$: $d_{9,1}=\infty$.

\[
    \begin{array}{c|ccccccc|c}
        i \setminus j & 1      & 2      & 3      & 4      & 5      & 6      & 8      & d_i \\
        \hline
        3             & 11     & 0      & \infty & 6      & 5      & 8      & 0      & 0   \\
        4             & 17     & 19     & 11     & \infty & 0      & 0      & 8      & 0   \\
        5             & 12     & 18     & 10     & 0      & \infty & 4      & 5      & 0   \\
        7             & 19     & \infty & 0      & 11     & 13     & 6      & 9      & 0   \\
        8             & 3      & 6      & 0      & 3      & 0      & 7      & \infty & 0   \\
        9             & \infty & 19     & 19     & 21     & 16     & 24     & 11     & 11  \\
        10            & 18     & 22     & 14     & 0      & 3      & \infty & 12     & 0   \\
        \hline
        d_j           & 3      & 0      & 0      & 0      & 0      & 0      & 0      & 14  \\
    \end{array}
\]

Сумма констант приведения сокращенной матрицы:
\[
    \sum d_i + \sum d_j = 14
\]

Нижняя граница подмножества $(1,9)$ равна:
\[
    H(1,9) = 217 + 14 = 231 \leq 231
\]

Ребро $(1,9)$ включаем в маршрут с новой границей $H=231$.

Шаг №4.
Определяем ребро ветвления.

\[
    \begin{array}{c|ccccccc|c}
        i \setminus j & 1      & 2      & 3      & 4      & 5      & 6      & 8      & d_i \\
        \hline
        3             & 8      & 0(6)   & \infty & 6      & 5      & 8      & 0(0)   & 0   \\
        4             & 14     & 19     & 11     & \infty & 0(0)   & 0(4)   & 8      & 0   \\
        5             & 9      & 18     & 10     & 0(4)   & \infty & 4      & 5      & 4   \\
        7             & 16     & \infty & 0(6)   & 11     & 13     & 6      & 9      & 6   \\
        8             & 0(8)   & 6      & 0(0)   & 3      & 0(0)   & 7      & \infty & 0   \\
        9             & \infty & 8      & 8      & 10     & 5      & 13     & 0(5)   & 5   \\
        10            & 15     & 22     & 14     & 0(3)   & 3      & \infty & 12     & 3   \\
        \hline
        d_j           & 8      & 6      & 0      & 0      & 0      & 4      & 0      & 0   \\
    \end{array}
\]

\[
    d(3,2) = 0 + 6 = 6; \quad d(3,8) = 0 + 0 = 0; \quad d(4,5) = 0 + 0 = 0; \quad d(4,6) = 0 + 4 = 4; \quad d(5,4) = 4 + 0 = 4; \quad d(7,3) = 6 + 0 = 6; \quad d(8,1) = 0 + 8 = 8; \quad d(8,3) = 0 + 0 = 0; \quad d(8,5) = 0 + 0 = 0; \quad d(9,8) = 5 + 0 = 5; \quad d(10,4) = 3 + 0 = 3;
\]

Максимум: $d(8,1)=8$.

Исключение ребра $(8,1)$: $d_{8,1}=\infty$.

\[
    \begin{array}{c|ccccccc|c}
        i \setminus j & 1      & 2      & 3      & 4      & 5      & 6      & 8      & d_i \\
        \hline
        3             & 8      & 0      & \infty & 6      & 5      & 8      & 0      & 0   \\
        4             & 14     & 19     & 11     & \infty & 0      & 0      & 8      & 0   \\
        5             & 9      & 18     & 10     & 0      & \infty & 4      & 5      & 0   \\
        7             & 16     & \infty & 0      & 11     & 13     & 6      & 9      & 0   \\
        8             & \infty & 6      & 0      & 3      & 0      & 7      & \infty & 0   \\
        9             & \infty & 8      & 8      & 10     & 5      & 13     & 0      & 0   \\
        10            & 15     & 22     & 14     & 0      & 3      & \infty & 12     & 0   \\
        \hline
        d_j           & 8      & 0      & 0      & 0      & 0      & 0      & 0      & 8   \\
    \end{array}
\]

Нижняя граница гамильтоновых циклов этого подмножества:
\[
    H(8^*,1^*) = 231 + 8 = 239
\]

Включение ребра $(8,1)$: $d_{1,8}=\infty$.

\[
    \begin{array}{c|cccccc|c}
        i \setminus j & 2      & 3      & 4      & 5      & 6      & 8  & d_i \\
        \hline
        3             & 0      & \infty & 6      & 5      & 8      & 0  & 0   \\
        4             & 19     & 11     & \infty & 0      & 0      & 8  & 0   \\
        5             & 18     & 10     & 0      & \infty & 4      & 5  & 0   \\
        7             & \infty & 0      & 11     & 13     & 6      & 9  & 0   \\
        9             & 8      & 8      & 10     & 5      & 13     & 0  & 0   \\
        10            & 22     & 14     & 0      & 3      & \infty & 12 & 0   \\
        \hline
        d_j           & 0      & 0      & 0      & 0      & 0      & 0  & 0   \\
    \end{array}
\]

Сумма констант приведения сокращенной матрицы:
\[
    \sum d_i + \sum d_j = 0
\]

Нижняя граница подмножества $(8,1)$ равна:
\[
    H(8,1) = 231 + 0 = 231 \leq 239
\]

Запрещаем переходы: $(9,8)$.

Ребро $(8,1)$ включаем в маршрут с новой границей $H=231$.

Шаг №5.
Определяем ребро ветвления.

\[
    \begin{array}{c|cccccc|c}
        i \setminus j & 2      & 3      & 4      & 5      & 6      & 8      & d_i \\
        \hline
        3             & 0(8)   & \infty & 6      & 5      & 8      & 0(5)   & 0   \\
        4             & 19     & 11     & \infty & 0(3)   & 0(4)   & 8      & 0   \\
        5             & 18     & 10     & 0(4)   & \infty & 4      & 5      & 4   \\
        7             & \infty & 0(14)  & 11     & 13     & 6      & 9      & 6   \\
        9             & 8      & 8      & 10     & 5      & 13     & \infty & 0   \\
        10            & 22     & 14     & 0(3)   & 3      & \infty & 12     & 3   \\
        \hline
        d_j           & 8      & 8      & 0      & 3      & 4      & 5      & 0   \\
    \end{array}
\]

\[
    d(3,2) = 0 + 8 = 8; \quad d(3,8) = 0 + 5 = 5; \quad d(4,5) = 0 + 3 = 3; \quad d(4,6) = 0 + 4 = 4; \quad d(5,4) = 4 + 0 = 4; \quad d(7,3) = 6 + 8 = 14; \quad d(10,4) = 3 + 0 = 3;
\]

Максимум: $d(7,3)=14$.

Исключение ребра $(7,3)$: $d_{7,3}=\infty$.

\[
    \begin{array}{c|cccccc|c}
        i \setminus j & 2      & 3      & 4      & 5      & 6      & 8      & d_i \\
        \hline
        3             & 0      & \infty & 6      & 5      & 8      & 0      & 0   \\
        4             & 19     & 11     & \infty & 0      & 0      & 8      & 0   \\
        5             & 18     & 10     & 0      & \infty & 4      & 5      & 0   \\
        7             & \infty & \infty & 11     & 13     & 6      & 9      & 6   \\
        9             & 8      & 8      & 10     & 5      & 13     & \infty & 5   \\
        10            & 22     & 14     & 0      & 3      & \infty & 12     & 0   \\
        \hline
        d_j           & 0      & 8      & 0      & 0      & 0      & 0      & 19  \\
    \end{array}
\]

Нижняя граница гамильтоновых циклов этого подмножества:
\[
    H(7^*,3^*) = 231 + 19 = 250
\]

Включение ребра $(7,3)$: $d_{3,7}=\infty$.

\[
    \begin{array}{c|ccccc|c}
        i \setminus j & 2  & 4      & 5      & 6      & 8      & d_i \\
        \hline
        3             & 0  & 6      & 5      & 8      & 0      & 0   \\
        4             & 19 & \infty & 0      & 0      & 8      & 0   \\
        5             & 18 & 0      & \infty & 4      & 5      & 0   \\
        9             & 8  & 10     & 5      & 13     & \infty & 5   \\
        10            & 22 & 0      & 3      & \infty & 12     & 0   \\
        \hline
        d_j           & 0  & 0      & 0      & 0      & 0      & 5   \\
    \end{array}
\]

Сумма констант приведения сокращенной матрицы:
\[
    \sum d_i + \sum d_j = 5
\]

Нижняя граница подмножества $(7,3)$ равна:
\[
    H(7,3) = 231 + 5 = 236 \leq 250
\]

Запрещаем переходы: $(3,2)$, $(9,8)$.

Ребро $(7,3)$ включаем в маршрут с новой границей $H=236$.

Шаг №6.
Определяем ребро ветвления.

\[
    \begin{array}{c|ccccc|c}
        i \setminus j & 2      & 4      & 5      & 6      & 8      & d_i \\
        \hline
        3             & \infty & 6      & 5      & 8      & 0(10)  & 5   \\
        4             & 19     & \infty & 0(0)   & 0(4)   & 8      & 0   \\
        5             & 18     & 0(4)   & \infty & 4      & 5      & 4   \\
        9             & 3      & 5      & 0(3)   & 8      & \infty & 3   \\
        10            & 22     & 0(3)   & 3      & \infty & 12     & 3   \\
        \hline
        d_j           & 0      & 0      & 0      & 4      & 5      & 0   \\
    \end{array}
\]

\[
    d(3,8) = 5 + 5 = 10; \quad d(4,5) = 0 + 0 = 0; \quad d(4,6) = 0 + 4 = 4; \quad d(5,4) = 4 + 0 = 4; \quad d(9,5) = 3 + 0 = 3; \quad d(10,4) = 3 + 0 = 3;
\]

Максимум: $d(3,8)=10$.

Исключение ребра $(3,8)$: $d_{3,8}=\infty$.

\[
    \begin{array}{c|ccccc|c}
        i \setminus j & 2      & 4      & 5      & 6      & 8      & d_i \\
        \hline
        3             & \infty & 6      & 5      & 8      & \infty & 5   \\
        4             & 19     & \infty & 0      & 0      & 8      & 0   \\
        5             & 18     & 0      & \infty & 4      & 5      & 0   \\
        9             & 3      & 5      & 0      & 8      & \infty & 0   \\
        10            & 22     & 0      & 3      & \infty & 12     & 0   \\
        \hline
        d_j           & 3      & 0      & 0      & 0      & 5      & 13  \\
    \end{array}
\]

Нижняя граница гамильтоновых циклов этого подмножества:
\[
    H(3^*,8^*) = 236 + 13 = 249
\]

Включение ребра $(3,8)$: $d_{8,3}=\infty$.

\[
    \begin{array}{c|cccc|c}
        i \setminus j & 2  & 4      & 5      & 6      & d_i \\
        \hline
        4             & 19 & \infty & 0      & 0      & 0   \\
        5             & 18 & 0      & \infty & 4      & 0   \\
        9             & 3  & 5      & 0      & 8      & 0   \\
        10            & 22 & 0      & 3      & \infty & 0   \\
        \hline
        d_j           & 3  & 0      & 0      & 0      & 3   \\
    \end{array}
\]

Сумма констант приведения сокращенной матрицы:
\[
    \sum d_i + \sum d_j = 3
\]

Нижняя граница подмножества $(3,8)$ равна:
\[
    H(3,8) = 236 + 3 = 239 \leq 249
\]

Запрещаем переходы: $(9,2)$, $(9,8)$, $(9,7)$, $(9,3)$.

Ребро $(3,8)$ включаем в маршрут с новой границей $H=239$.

Шаг №7.
Определяем ребро ветвления.

\[
    \begin{array}{c|cccc|c}
        i \setminus j & 2      & 4      & 5      & 6      & d_i \\
        \hline
        4             & 16     & \infty & 0(0)   & 0(4)   & 0   \\
        5             & 15     & 0(4)   & \infty & 4      & 4   \\
        9             & \infty & 5      & 0(5)   & 8      & 5   \\
        10            & 19     & 0(3)   & 3      & \infty & 3   \\
        \hline
        d_j           & 3      & 0      & 0      & 4      & 0   \\
    \end{array}
\]

\[
    d(4,5) = 0 + 0 = 0; \quad d(4,6) = 0 + 4 = 4; \quad d(5,4) = 4 + 0 = 4; \quad d(9,5) = 5 + 0 = 5; \quad d(10,4) = 3 + 0 = 3;
\]

Максимум: $d(9,5)=5$.

Исключение ребра $(9,5)$: $d_{9,5}=\infty$.

\[
    \begin{array}{c|cccc|c}
        i \setminus j & 2      & 4      & 5      & 6      & d_i \\
        \hline
        4             & 16     & \infty & 0      & 0      & 0   \\
        5             & 15     & 0      & \infty & 4      & 0   \\
        9             & \infty & 5      & \infty & 8      & 5   \\
        10            & 19     & 0      & 3      & \infty & 0   \\
        \hline
        d_j           & 15     & 0      & 0      & 0      & 20  \\
    \end{array}
\]

Нижняя граница гамильтоновых циклов этого подмножества:
\[
    H(9^*,5^*) = 239 + 20 = 259
\]

Включение ребра $(9,5)$: $d_{5,9}=\infty$.

\[
    \begin{array}{c|ccc|c}
        i \setminus j & 2  & 4      & 6      & d_i \\
        \hline
        4             & 16 & \infty & 0      & 0   \\
        5             & 15 & 0      & 4      & 0   \\
        10            & 19 & 0      & \infty & 0   \\
        \hline
        d_j           & 15 & 0      & 0      & 15  \\
    \end{array}
\]

Сумма констант приведения сокращенной матрицы:
\[
    \sum d_i + \sum d_j = 15
\]

Нижняя граница подмножества $(9,5)$ равна:
\[
    H(9,5) = 239 + 15 = 254 \leq 259
\]

Запрещаем переходы: $(5,2)$, $(5,1)$, $(5,8)$, $(5,7)$, $(5,3)$.

Ребро $(9,5)$ включаем в маршрут с новой границей $H=254$.

Шаг №8.
Определяем ребро ветвления.

\[
    \begin{array}{c|ccc|c}
        i \setminus j & 2      & 4      & 6      & d_i \\
        \hline
        4             & 1      & \infty & 0(5)   & 1   \\
        5             & \infty & 0(4)   & 4      & 4   \\
        10            & 4      & 0(4)   & \infty & 4   \\
        \hline
        d_j           & 15     & 0      & 4      & 0   \\
    \end{array}
\]

\[
    d(4,6) = 1 + 4 = 5; \quad d(5,4) = 4 + 0 = 4; \quad d(10,4) = 4 + 0 = 4;
\]

Максимум: $d(4,6)=5$.

Исключение ребра $(4,6)$: $d_{4,6}=\infty$.

\[
    \begin{array}{c|ccc|c}
        i \setminus j & 2      & 4      & 6      & d_i \\
        \hline
        4             & 1      & \infty & \infty & 1   \\
        5             & \infty & 0      & 4      & 0   \\
        10            & 4      & 0      & \infty & 0   \\
        \hline
        d_j           & 1      & 0      & 4      & 6   \\
    \end{array}
\]

Нижняя граница гамильтоновых циклов этого подмножества:
\[
    H(4^*,6^*) = 254 + 6 = 260
\]

Включение ребра $(4,6)$: $d_{6,4}=\infty$.

\[
    \begin{array}{c|cc|c}
        i \setminus j & 2      & 4 & d_i \\
        \hline
        5             & \infty & 0 & 0   \\
        10            & 4      & 0 & 0   \\
        \hline
        d_j           & 4      & 0 & 4   \\
    \end{array}
\]

Сумма констант приведения сокращенной матрицы:
\[
    \sum d_i + \sum d_j = 4
\]

Нижняя граница подмножества $(4,6)$ равна:
\[
    H(4,6) = 254 + 4 = 258 \leq 260
\]

Ребро $(4,6)$ включаем в маршрут с новой границей $H=258$.

В соответствии с этой матрицей включаем в гамильтонов маршрут ребра $(5,4)$ и $(10,2)$.

В результате по дереву ветвлений гамильтонов цикл образуют ребра:
\[
    (6,10), (10,2), (2,7), (7,3), (3,8), (8,1), (1,9), (9,5), (5,4), (4,6)
\]

Длина маршрута равна $F(M_k) = 258$

\begin{tikzpicture}[node distance=1.5cm and 2.5cm]

    % Узлы
    \node[mybox] (start) {1\\Начало};
    \node[mybox, right=of start] (start1) {(6*, 10*)\\$H=202$};
    \node[mybox, below=of start] (a) {(6, 10)\\$H=202$};
    \node[mybox, right=of a] (a1) {(2*, 7*)\\$H=217$};
    \node[mybox, below=of a] (b) {(2, 7)\\$H=217$};
    \node[mybox, right=of b] (b1) {(1*, 9*)\\$H=231$};
    \node[mybox, below=of b] (c) {(1, 9)\\$H=231$};
    \node[mybox, right=of c] (c1) {(8*, 1*)\\$H=239$};
    \node[mybox, below=of c] (d) {(8, 1)\\$H=231$};
    \node[mybox, right=of d] (d1) {(7*, 3*)\\$H=250$};
    \node[mybox, below=of d] (e) {(7, 3)\\$H=236$};
    \node[mybox, right=of e] (e1) {(3*, 8*)\\$H=249$};
    \node[mybox, below=of e] (f) {(3, 8)\\$H=239$};
    \node[mybox, right=of f] (f1) {(9*, 5*)\\$H=259$};
    \node[mybox, below=of f] (g) {(9, 5)\\$H=254$};
    \node[mybox, right=of g] (g1) {(4*, 6*)\\$H=260$};
    \node[mybox, below=of g] (h) {(4, 16)\\$H=258$};
    \node[mybox, right=of h] (h1) {(5*, 4*)\\$H=258$};
    \node[mybox, below=of h] (i) {(5, 4)\\$H=258$};
    \node[mybox, right=of i] (i1) {(10*, 2*)\\$H=258$};
    \node[mybox, below=of i] (j) {(10, 2)\\$H=258$};

    % Стрелки
    \draw[myarrow] (start) -- (a);
    \draw[myarrow] (a) -- (b);
    \draw[myarrow] (b) -- (c);
    \draw[myarrow] (c) -- (d);
    \draw[myarrow] (d) -- (e);
    \draw[myarrow] (e) -- (f);
    \draw[myarrow] (f) -- (g);
    \draw[myarrow] (g) -- (h);
    \draw[myarrow] (h) -- (i);
    \draw[myarrow] (i) -- (j);
    \draw[myarrow] (start) -- (start1);
    \draw[myarrow] (a) -- (a1);
    \draw[myarrow] (b) -- (b1);
    \draw[myarrow] (c) -- (c1);
    \draw[myarrow] (d) -- (d1);
    \draw[myarrow] (e) -- (e1);
    \draw[myarrow] (f) -- (f1);
    \draw[myarrow] (g) -- (g1);
    \draw[myarrow] (h) -- (h1);
    \draw[myarrow] (i) -- (i1);



\end{tikzpicture}

Используя код на Python:

\begin{verbatim}
    from sys import maxsize
    from itertools import permutations
    
    # Количество вершин в графе
    V = 10
    
    # Матрица стоимостей
    graph = [
        [maxsize, 29, 20, 21, 16, 31, 100, 12, 4, 31],
        [29, maxsize, 15, 29, 28, 40, 72, 21, 29, 41],
        [20, 15, maxsize, 15, 14, 25, 81, 9, 23, 27],
        [21, 29, 15, maxsize, 4, 12, 92, 12, 25, 13],
        [16, 28, 14, 4, maxsize, 16, 94, 9, 20, 16],
        [31, 40, 25, 12, 16, maxsize, 95, 24, 36, 3],
        [100, 72, 81, 92, 94, 95, maxsize, 90, 101, 99],
        [12, 21, 9, 12, 9, 24, 90, maxsize, 15, 25],
        [4, 29, 23, 25, 20, 36, 101, 15, maxsize, 35],
        [31, 41, 27, 13, 16, 3, 99, 25, 35, maxsize]
    ]
    
    def tsp(graph, s):
        # Список вершин, исключая стартовую
        vertex = []
        for i in range(V):
            if i != s:
                vertex.append(i)
        
        # Инициализация минимальной стоимости
        min_cost = maxsize
        next_permutation = permutations(vertex)
        
        # Перебор всех возможных маршрутов
        for i in next_permutation:
            current_cost = 0
            k = s
            
            # Считаем стоимость текущего маршрута
            for j in i:
                current_cost += graph[k][j]
                k = j
            current_cost += graph[k][s]
            
            # Обновляем минимальную стоимость
            min_cost = min(min_cost, current_cost)
        
        return min_cost
    
    # Стартовая вершина
    s = 0
    
    # Выводим результат
    print("Минимальная стоимость:", tsp(graph, s))
\end{verbatim}

Получаем такие же значения:
\begin{verbatim}
    python3 main.py
    Минимальная стоимость: 258
\end{verbatim}

\text{Результаты совапали.}
\newpage
\section{Задание №8}

\subsection{Условие}

Придумать задачу о назначениях размерности $10 \times 10$ на поиск max. Диапазон значений элементов матрицы от 1 до 9.\\
Решить задачу, используя венгерский алгоритм.\\
В отчёт добавить решение задачи с помощью программных средств.

\subsection{Постановка задачи}

$n = 10$ ресурсов и объектов. Матрица стоимостей $C$ размерности $n \times n$:

\[
    C = \begin{bmatrix}
        7 & 2 & 5 & 9 & 3 & 8 & 6 & 4 & 9 & 2 \\
        4 & 5 & 9 & 6 & 8 & 1 & 2 & 7 & 3 & 5 \\
        4 & 8 & 3 & 7 & 2 & 3 & 5 & 6 & 4 & 8 \\
        6 & 3 & 7 & 4 & 5 & 2 & 9 & 1 & 8 & 7 \\
        2 & 9 & 1 & 8 & 6 & 5 & 4 & 3 & 7 & 9 \\
        5 & 4 & 6 & 2 & 7 & 9 & 3 & 4 & 9 & 6 \\
        8 & 1 & 4 & 5 & 9 & 3 & 7 & 2 & 6 & 4 \\
        9 & 7 & 3 & 3 & 4 & 6 & 8 & 5 & 5 & 5 \\
        3 & 6 & 8 & 7 & 6 & 4 & 2 & 9 & 5 & 3 \\
        6 & 3 & 9 & 6 & 2 & 8 & 8 & 4 & 8 & 2
    \end{bmatrix}
\]

1. Если задача решается на максимум (как в нашем случае), то в каждой строке матрицы необходимо найти максимальный элемент, вычесть его из каждого элемента соответствующей строки и умножить всю матрицу на $-1$. Если задача решается на минимум, то этот шаг необходимо пропустить.
\[
    \begin{tabular}{|c|c|c|c|c|c|c|c|c|c|}
        \hline
        2 & 7 & 4 & 0 & 6 & 1 & 3 & 5 & 0 & 7 \\
        \hline
        5 & 4 & 0 & 3 & 1 & 8 & 7 & 2 & 6 & 4 \\
        \hline
        5 & 1 & 6 & 2 & 7 & 6 & 4 & 3 & 5 & 1 \\
        \hline
        3 & 6 & 2 & 5 & 4 & 7 & 0 & 8 & 1 & 2 \\
        \hline
        7 & 0 & 8 & 1 & 3 & 4 & 5 & 6 & 2 & 0 \\
        \hline
        4 & 5 & 3 & 7 & 2 & 0 & 6 & 5 & 0 & 3 \\
        \hline
        1 & 8 & 5 & 4 & 0 & 6 & 2 & 7 & 3 & 5 \\
        \hline
        0 & 2 & 6 & 6 & 5 & 3 & 1 & 4 & 4 & 4 \\
        \hline
        6 & 3 & 1 & 2 & 3 & 5 & 7 & 0 & 4 & 6 \\
        \hline
        3 & 6 & 0 & 3 & 7 & 1 & 1 & 5 & 1 & 7 \\
        \hline
    \end{tabular}
\]

2. Проводим редукцию матрицы по строкам. В связи с этим во вновь полученной матрице в каждой строке будет как минимум один ноль.
\[
    \begin{tabular}{|c|c|c|c|c|c|c|c|c|c|c|}
        \hline
        2 & 7 & 4 & 0 & 6 & 1 & 3 & 5 & 0 & 7 & 0 \\
        \hline
        5 & 4 & 0 & 3 & 1 & 8 & 7 & 2 & 6 & 4 & 0 \\
        \hline
        4 & 0 & 5 & 1 & 6 & 5 & 3 & 2 & 4 & 0 & 1 \\
        \hline
        3 & 6 & 2 & 5 & 4 & 7 & 0 & 8 & 1 & 2 & 0 \\
        \hline
        7 & 0 & 8 & 1 & 3 & 4 & 5 & 6 & 2 & 0 & 0 \\
        \hline
        4 & 5 & 3 & 7 & 2 & 0 & 6 & 5 & 0 & 3 & 0 \\
        \hline
        1 & 8 & 5 & 4 & 0 & 6 & 2 & 7 & 3 & 5 & 0 \\
        \hline
        0 & 2 & 6 & 6 & 5 & 3 & 1 & 4 & 4 & 4 & 0 \\
        \hline
        6 & 3 & 1 & 2 & 3 & 5 & 7 & 0 & 4 & 6 & 0 \\
        \hline
        3 & 6 & 0 & 3 & 7 & 1 & 1 & 5 & 1 & 7 & 0 \\
        \hline
    \end{tabular}
\]

Затем такую же операцию редукции проводим по столбцам, для чего в каждом столбце находим минимальный элемент.
\[
    \begin{tabular}{|c|c|c|c|c|c|c|c|c|c|}
        \hline
        2 & 7 & 4 & 0 & 6 & 1 & 3 & 5 & 0 & 7 \\
        \hline
        5 & 4 & 0 & 3 & 1 & 8 & 7 & 2 & 6 & 4 \\
        \hline
        4 & 0 & 5 & 1 & 6 & 5 & 3 & 2 & 4 & 0 \\
        \hline
        3 & 6 & 2 & 5 & 4 & 7 & 0 & 8 & 1 & 2 \\
        \hline
        7 & 0 & 8 & 1 & 3 & 4 & 5 & 6 & 2 & 0 \\
        \hline
        4 & 5 & 3 & 7 & 2 & 0 & 6 & 5 & 0 & 3 \\
        \hline
        1 & 8 & 5 & 4 & 0 & 6 & 2 & 7 & 3 & 5 \\
        \hline
        0 & 2 & 6 & 6 & 5 & 3 & 1 & 4 & 4 & 4 \\
        \hline
        6 & 3 & 1 & 2 & 3 & 5 & 7 & 0 & 4 & 6 \\
        \hline
        3 & 6 & 0 & 3 & 7 & 1 & 1 & 5 & 1 & 7 \\
        \hline
        0 & 0 & 0 & 0 & 0 & 0 & 0 & 0 & 0 & 0 \\
        \hline
    \end{tabular}
\]

После вычитания минимальных элементов получаем полностью редуцированную матрицу.

3. Смотрим чтобы в каждом столбце и в каждой строке был только один выбранный ноль. Как видно ниже, в данном случае это сделать невозможно.

В итоге получаем следующую матрицу:
\[
    \begin{tabular}{|>{\columncolor{brown}}c|>{\columncolor{red}}c|>{\columncolor{green}}c|c|>{\columncolor{olive}}c|>{\columncolor{purple}}c|>{\columncolor{pink}}c|>{\columncolor{magenta}}c|>{\columncolor{yellow}}c|>{\columncolor{orange}}c|}
        \hline
        \rowcolor{yellow}
        2 & 7 & 4 & 0 & 6 & 1 & 3 & 5 & 0 & 7 \\
        \hline
        \rowcolor{green}
        5 & 4 & 0 & 3 & 1 & 8 & 7 & 2 & 6 & 4 \\
        \hline
        \rowcolor{orange}
        4 & 0 & 5 & 1 & 6 & 5 & 3 & 2 & 4 & 0 \\
        \hline
        \rowcolor{pink}
        3 & 6 & 2 & 5 & 4 & 7 & 0 & 8 & 1 & 2 \\
        \hline
        \rowcolor{red}
        7 & 0 & 8 & 1 & 3 & 4 & 5 & 6 & 2 & 0 \\
        \hline
        \rowcolor{purple}
        4 & 5 & 3 & 7 & 2 & 0 & 6 & 5 & 0 & 3 \\
        \hline
        \rowcolor{olive}
        1 & 8 & 5 & 4 & 0 & 6 & 2 & 7 & 3 & 5 \\
        \hline
        \rowcolor{brown}
        0 & 2 & 6 & 6 & 5 & 3 & 1 & 4 & 4 & 4 \\
        \hline
        \rowcolor{magenta}
        6 & 3 & 1 & 2 & 3 & 5 & 7 & 0 & 4 & 6 \\
        \hline
        3 & 6 & 0 & 3 & 7 & 1 & 1 & 5 & 1 & 7 \\
        \hline
    \end{tabular}
\]

Поскольку расположение нулевых элементов в матрице не позволяет образовать систему из 10-х независимых нулей (в матрице их только 9), то решение недопустимое.

4. Проводим модификацию матрицы. Вычеркиваем строки и столбцы с возможно
большим количеством нулевых элементов: строку 1, столбец 2, строку 6, столбец 3, столбец 10, строку 4,
строку 7, столбец 1, строку 9. Получаем сокращенную матрицу:

\[
    \begin{tabular}{|c|c|c|c|c|c|c|c|c|c|}
        \hline
        2 & 7 & 4 & 0                    & 6                    & 1                    & 3                    & 5                    & 0                    & 7 \\
        \hline
        5 & 4 & 0 & \cellcolor{yellow} 3 & \cellcolor{yellow} 1 & \cellcolor{yellow} 8 & \cellcolor{yellow} 7 & \cellcolor{yellow} 2 & \cellcolor{yellow} 6 & 4 \\
        \hline
        4 & 0 & 5 & \cellcolor{yellow} 1 & \cellcolor{yellow} 6 & \cellcolor{yellow} 5 & \cellcolor{yellow} 3 & \cellcolor{yellow} 2 & \cellcolor{yellow} 4 & 0 \\
        \hline
        3 & 6 & 2 & 5                    & 4                    & 7                    & 0                    & 8                    & 1                    & 2 \\
        \hline
        7 & 0 & 8 & \cellcolor{yellow} 1 & \cellcolor{yellow} 3 & \cellcolor{yellow} 4 & \cellcolor{yellow} 5 & \cellcolor{yellow} 6 & \cellcolor{yellow} 2 & 0 \\
        \hline
        4 & 5 & 3 & 7                    & 2                    & 0                    & 6                    & 5                    & 0                    & 3 \\
        \hline
        1 & 8 & 5 & 4                    & 0                    & 6                    & 2                    & 7                    & 3                    & 5 \\
        \hline
        0 & 2 & 6 & \cellcolor{yellow} 6 & \cellcolor{yellow} 5 & \cellcolor{yellow} 3 & \cellcolor{yellow} 1 & \cellcolor{yellow} 4 & \cellcolor{yellow} 4 & 4 \\
        \hline
        6 & 3 & 1 & 2                    & 3                    & 5                    & 7                    & 0                    & 4                    & 6 \\
        \hline
        3 & 6 & 0 & \cellcolor{yellow} 3 & \cellcolor{yellow} 7 & \cellcolor{yellow} 1 & \cellcolor{yellow} 1 & \cellcolor{yellow} 5 & \cellcolor{yellow} 1 & 7 \\
        \hline
    \end{tabular}
\]

Минимальный элемент сокращенной матрицы (min(3, 1, 8, 7, 2, 6, 1, 6, 5, 3, 2, 4, 1, 3, 4, 5, 6, 2, 6, 5, 3, 1, 4, 4, 3, 7, 1, 1, 5, 1) = 1) вычитаем из всех её элементов:

\[
    \begin{tabular}{|c|c|c|c|c|c|c|c|c|c|}
        \hline
        2 & 7 & 4 & 0                   & 6                   & 1                   & 3                   & 5                   & 0                   & 7 \\
        \hline
        5 & 4 & 0 & \cellcolor{yellow}2 & \cellcolor{yellow}0 & \cellcolor{yellow}7 & \cellcolor{yellow}6 & \cellcolor{yellow}1 & \cellcolor{yellow}5 & 4 \\
        \hline
        4 & 0 & 5 & \cellcolor{yellow}0 & \cellcolor{yellow}5 & \cellcolor{yellow}4 & \cellcolor{yellow}2 & \cellcolor{yellow}1 & \cellcolor{yellow}3 & 0 \\
        \hline
        3 & 6 & 2 & 5                   & 4                   & 7                   & 0                   & 8                   & 1                   & 2 \\
        \hline
        7 & 0 & 8 & \cellcolor{yellow}0 & \cellcolor{yellow}2 & \cellcolor{yellow}3 & \cellcolor{yellow}4 & \cellcolor{yellow}5 & \cellcolor{yellow}1 & 0 \\
        \hline
        4 & 5 & 3 & 7                   & 2                   & 0                   & 6                   & 5                   & 0                   & 3 \\
        \hline
        1 & 8 & 5 & 4                   & 0                   & 6                   & 2                   & 7                   & 3                   & 5 \\
        \hline
        0 & 2 & 6 & \cellcolor{yellow}5 & \cellcolor{yellow}4 & \cellcolor{yellow}2 & \cellcolor{yellow}0 & \cellcolor{yellow}3 & \cellcolor{yellow}3 & 4 \\
        \hline
        6 & 3 & 1 & 2                   & 3                   & 5                   & 7                   & 0                   & 4                   & 6 \\
        \hline
        3 & 6 & 0 & \cellcolor{yellow}2 & \cellcolor{yellow}6 & \cellcolor{yellow}0 & \cellcolor{yellow}0 & \cellcolor{yellow}4 & \cellcolor{yellow}0 & 7 \\
        \hline
    \end{tabular}
\]

Затем складываем минимальный элемент с элементами, расположенными на пересечениях вычеркнутых строк и столбцов:

\[
    \begin{tabular}{|c|c|c|c|c|c|c|c|c|c|}
        \hline
        \cellcolor{green} 3 & \cellcolor{green} 8 & \cellcolor{green} 5 & 0 & 6 & 1 & 3 & 5 & 0 & \cellcolor{green} 8 \\
        \hline
        5                   & 4                   & 0                   & 2 & 0 & 7 & 6 & 1 & 5 & 4                   \\
        \hline
        4                   & 0                   & 5                   & 0 & 5 & 4 & 2 & 1 & 3 & 0                   \\
        \hline
        \cellcolor{green} 4 & \cellcolor{green} 7 & \cellcolor{green} 3 & 5 & 4 & 7 & 0 & 8 & 1 & \cellcolor{green} 3 \\
        \hline
        7                   & 0                   & 8                   & 0 & 2 & 3 & 4 & 5 & 1 & 0                   \\
        \hline
        \cellcolor{green} 5 & \cellcolor{green} 6 & \cellcolor{green} 4 & 7 & 2 & 0 & 6 & 5 & 0 & \cellcolor{green} 4 \\
        \hline
        \cellcolor{green} 2 & \cellcolor{green} 9 & \cellcolor{green} 6 & 4 & 0 & 6 & 2 & 7 & 3 & \cellcolor{green} 6 \\
        \hline
        0                   & 2                   & 6                   & 5 & 4 & 2 & 0 & 3 & 3 & 4                   \\
        \hline
        \cellcolor{green} 7 & \cellcolor{green} 4 & \cellcolor{green} 2 & 2 & 3 & 5 & 7 & 0 & 4 & \cellcolor{green} 7 \\
        \hline
        3                   & 6                   & 0                   & 2 & 6 & 0 & 0 & 4 & 0 & 7                   \\
        \hline
    \end{tabular}
\]

5. Проводим редукцию матрицы по строкам. В связи с этим во вновь полученной матрице в каждой строке будет как минимум один ноль.
Затем такую же операцию редукции проводим по столбцам, для чего в каждом столбце находим минимальный элемент.
После вычитания минимальных элементов получаем полностью редуцированную матрицу.

\[
    \begin{tabular}{|c|c|c|c|c|c|c|c|c|c|}
        \hline
        \rowcolor{yellow}
        3 & 8 & 5 & 0 & 6 & 1 & 3 & 5 & 0 & 8 \\
        \hline
        \rowcolor{green}
        5 & 4 & 0 & 2 & 0 & 7 & 6 & 1 & 5 & 4 \\
        \hline
        \rowcolor{orange}
        4 & 0 & 5 & 0 & 5 & 4 & 2 & 1 & 3 & 0 \\
        \hline
        \rowcolor{pink}
        4 & 7 & 3 & 5 & 4 & 7 & 0 & 8 & 1 & 3 \\
        \hline
        \rowcolor{red}
        7 & 0 & 8 & 0 & 2 & 3 & 4 & 5 & 1 & 0 \\
        \hline
        \rowcolor{purple}
        5 & 6 & 4 & 7 & 2 & 0 & 6 & 5 & 0 & 4 \\
        \hline
        \rowcolor{olive}
        2 & 9 & 6 & 4 & 0 & 6 & 2 & 7 & 3 & 6 \\
        \hline
        \rowcolor{brown}
        0 & 2 & 6 & 5 & 4 & 2 & 0 & 3 & 3 & 4 \\
        \hline
        \rowcolor{magenta}
        7 & 4 & 2 & 2 & 3 & 5 & 7 & 0 & 4 & 7 \\
        \hline
        \rowcolor{teal}
        3 & 6 & 0 & 2 & 6 & 0 & 0 & 4 & 0 & 7 \\
        \hline
    \end{tabular}
\]

Количество найденных нулей равно $k = 10$. Таким образом, найдено оптимальное решение задачи о назначениях.\\

6. Посчитаем сумму элементов, стоящих на пересечениях строк и столбцов с нулевыми элементами:

\[
    \begin{tabular}{|c|c|c|c|c|c|c|c|c|c|}
        \hline
        7                  & 2 & 5                  & \cellcolor{teal} 9 & 3                  & 8                  & 6                  & 4                  & 9                  & 2                  \\
        \hline
        4                  & 5 & \cellcolor{teal} 9 & 6                  & 8                  & 1                  & 2                  & 7                  & 3                  & 5                  \\
        \hline
        4                  & 8 & 3                  & 7                  & 2                  & 3                  & 5                  & 6                  & 4                  & \cellcolor{teal} 8 \\
        \hline
        6                  & 3 & 7                  & 4                  & 5                  & 2                  & \cellcolor{teal} 9 & 1                  & 8                  & 7                  \\
        \hline
        2                  & 9 & 1                  & 8                  & 6                  & 5                  & 4                  & 3                  & 7                  & \cellcolor{teal} 9 \\
        \hline
        5                  & 4 & 6                  & 2                  & 7                  & \cellcolor{teal} 9 & 3                  & 4                  & 9                  & 6                  \\
        \hline
        8                  & 1 & 4                  & 5                  & \cellcolor{teal} 9 & 3                  & 7                  & 2                  & 6                  & 4                  \\
        \hline
        \cellcolor{teal} 9 & 7 & 3                  & 3                  & 4                  & 6                  & 8                  & 5                  & 5                  & 5                  \\
        \hline
        3                  & 6 & 8                  & 7                  & 6                  & 4                  & 2                  & \cellcolor{teal} 9 & 5                  & 3                  \\
        \hline
        6                  & 3 & 9                  & 6                  & 2                  & 8                  & 8                  & 4                  & \cellcolor{teal} 8 & 2                  \\
        \hline
    \end{tabular}
\]\\

\[ C_{\text{max}} = 9 + 9 + 9 + 9 + 9 + 9 + 9 + 8 + 9 + 8 = 88 \]

Используя код на Go:

\begin{verbatim}
from hungarian_algorithm import algorithm

G = {
'01': {'1': 7, '2': 2, '3': 5, '4': 9, '5': 3, '6': 8, '7': 6, '8': 4, '9': 9, '10': 2},
'02': {'1': 4, '2': 5, '3': 9, '4': 6, '5': 8, '6': 1, '7': 2, '8': 7, '9': 3, '10': 5},
'03': {'1': 4, '2': 8, '3': 3, '4': 7, '5': 2, '6': 3, '7': 5, '8': 6, '9': 4, '10': 8},
'04': {'1': 6, '2': 3, '3': 7, '4': 4, '5': 5, '6': 2, '7': 9, '8': 1, '9': 8, '10': 7},
'05': {'1': 2, '2': 9, '3': 1, '4': 8, '5': 6, '6': 5, '7': 4, '8': 3, '9': 7, '10': 9},
'06': {'1': 5, '2': 4, '3': 6, '4': 2, '5': 7, '6': 9, '7': 3, '8': 4, '9': 9, '10': 6},
'07': {'1': 8, '2': 1, '3': 4, '4': 5, '5': 9, '6': 3, '7': 7, '8': 2, '9': 6, '10': 4},
'08': {'1': 9, '2': 7, '3': 3, '4': 3, '5': 4, '6': 6, '7': 8, '8': 5, '9': 5, '10': 5},
'09': {'1': 3, '2': 6, '3': 8, '4': 7, '5': 6, '6': 4, '7': 2, '8': 9, '9': 5, '10': 3},
'010': {'1': 6, '2': 3, '3': 9, '4': 6, '5': 2, '6': 8, '7': 8, '8': 4, '9': 8, '10': 2},
}

print(algorithm.find_matching(G, matching_type="max", return_type="list"))

print(algorithm.find_matching(G, matching_type="max", return_type="total"))
\end{verbatim}

Получаем такие же значения:
\begin{verbatim}
python3 main.py
[(('05', '10'), 9), (('010', '6'), 8), (('01', '4'), 9), (('09', '8'), 9), (('04', '7'), 9), (('08', '1'), 9), (('03', '2'), 8), (('07', '5'), 9), (('06', '9'), 9), (('02', '3'), 9)]
88
\end{verbatim}

\text{Результаты совапали.}
\section{Задание №9}

\subsection{Условие}

Придумать задачу о распределении ресурсов размерности $6 \times 6$.
То есть в задаче имеется ресурс в количестве 6 единиц, который должен быть распределен между 6 предприятиями.
Диапазон значений в матрице доходности не ограничен(но не может быть отрицательных элементов).\\
Решить задачу динамическим программированием.\\
В отчёт добавить решение задачи с помощью программных средств.

\subsection{Постановка задачи}

$S = 6$ - количество имеющихся ресурсов.\\
$n = 6$ - количество предприятий.\\
Использование $j$-ым предприятием $i$ единиц
ресурса дает доход, определяемый значением нелинейной функции $f_j(i) = f_{ij}$. Обычно значения функции $f_j(i)$ задаются в виде матрицы доходности $F = \|f_{ij}\|_{(S+1) \times n}$.
Зададим матрицу доходности $F$ в виде таблицы:

\[
    \begin{tabular}{|c|c|c|c|c|c|c|}
        \hline
          & 1 & 2 & 3 & 4 & 5 & 6 \\
        \hline
        0 & 0 & 0 & 0 & 0 & 0 & 0 \\
        \hline
        1 & 4 & 2 & 5 & 3 & 7 & 1 \\
        \hline
        2 & 6 & 3 & 2 & 5 & 4 & 8 \\
        \hline
        3 & 3 & 7 & 4 & 2 & 6 & 5 \\
        \hline
        4 & 5 & 1 & 3 & 4 & 2 & 7 \\
        \hline
        5 & 2 & 4 & 6 & 1 & 5 & 3 \\
        \hline
        6 & 7 & 5 & 1 & 6 & 3 & 4 \\
        \hline
    \end{tabular}
\]

1. Для первого предприятия:
\[
    \begin{aligned}
        \phi_1(x) & = \max \left[ f_1(x_1) \right], \quad 0 \leq x_1 \leq x \\
        \phi_1(0) & = 0, \quad x_1^0 = 0                                    \\
        \phi_1(1) & = \max\{0, 4\} = 4, \quad x_0^1 = 1                     \\
        \phi_1(2) & = \max\{0, 4, 6\} = 6, \quad x_1^2 = 2                  \\
        \phi_1(3) & = \max\{0, 4, 6, 3\} = 6, \quad x_1^3 = 2               \\
        \phi_1(4) & = \max\{0, 4, 6, 3, 5\} = 6, \quad x_1^4 = 2            \\
        \phi_1(5) & = \max\{0, 4, 6, 3, 5, 2\} = 6, \quad x_1^5 = 2         \\
        \phi_1(6) & = \max\{0, 4, 6, 3, 5, 2, 7\} = 7, \quad x_1^6 = 6      \\
    \end{aligned}
\]

2. Для второго предприятия:
\[
    \begin{aligned}
        \phi_2(x) & = \max \left[ f_2(x_2) + \phi_1(x - x_2) \right], \quad 0 \leq x_2 \leq x \\
        \phi_2(0) & = 0, \quad x_2^0 = 0                                                      \\
        \phi_2(1) & = \max \begin{cases}
                               f_2(1) + \phi_1(0) \\
                               f_2(0) + \phi_1(1)
                           \end{cases} = \max \begin{cases}
                                                  2 + 0 \\
                                                  0 + 4
                                              \end{cases} = 4, \quad x_2^1 = 0                \\
        \phi_2(2) & = \max \begin{cases}
                               f_2(2) + \phi_1(0) \\
                               f_2(1) + \phi_1(1) \\
                               f_2(0) + \phi_1(2)
                           \end{cases} = \max \begin{cases}
                                                  3 + 0 \\
                                                  2 + 4 \\
                                                  0 + 6
                                              \end{cases} = 6, \quad x_2^2 = 1                \\
        \phi_2(3) & = \max \begin{cases}
                               f_2(3) + \phi_1(0) \\
                               f_2(2) + \phi_1(1) \\
                               f_2(1) + \phi_1(2) \\
                               f_2(0) + \phi_1(3)
                           \end{cases} = \max \begin{cases}
                                                  7 + 0 \\
                                                  3 + 4 \\
                                                  2 + 6 \\
                                                  0 + 6
                                              \end{cases} = 8, \quad x_2^3 = 1                \\
        \phi_2(4) & = \max \begin{cases}
                               f_2(4) + \phi_1(0) \\
                               f_2(3) + \phi_1(1) \\
                               f_2(2) + \phi_1(2) \\
                               f_2(1) + \phi_1(3) \\
                               f_2(0) + \phi_1(4)
                           \end{cases} = \max \begin{cases}
                                                  1 + 0 \\
                                                  7 + 4 \\
                                                  3 + 6 \\
                                                  2 + 6 \\
                                                  0 + 6
                                              \end{cases} = 11, \quad x_2^4 = 3               \\
        \phi_2(5) & = \max \begin{cases}
                               f_2(5) + \phi_1(0) \\
                               f_2(4) + \phi_1(1) \\
                               f_2(3) + \phi_1(2) \\
                               f_2(2) + \phi_1(3) \\
                               f_2(1) + \phi_1(4) \\
                               f_2(0) + \phi_1(5)
                           \end{cases} = \max \begin{cases}
                                                  4 + 0 \\
                                                  1 + 4 \\
                                                  7 + 6 \\
                                                  3 + 6 \\
                                                  2 + 6 \\
                                                  0 + 6
                                              \end{cases} = 13, \quad x_2^4 = 3               \\
        \phi_2(6) & = \max \begin{cases}
                               f_2(6) + \phi_1(0) \\
                               f_2(5) + \phi_1(1) \\
                               f_2(4) + \phi_1(2) \\
                               f_2(3) + \phi_1(3) \\
                               f_2(2) + \phi_1(4) \\
                               f_2(1) + \phi_1(5) \\
                               f_2(0) + \phi_1(6)
                           \end{cases} = \max \begin{cases}
                                                  5 + 0 \\
                                                  4 + 4 \\
                                                  1 + 6 \\
                                                  7 + 6 \\
                                                  3 + 6 \\
                                                  2 + 6 \\
                                                  0 + 7
                                              \end{cases} = 13, \quad x_2^6 = 3               \\
    \end{aligned}
\]

3. Для третьего предприятия:
\[
    \begin{aligned}
        \phi_3(x) & = \max \left[ f_3(x_3) + \phi_2(x - x_3) \right], \quad 0 \leq x_3 \leq x \\
        \phi_3(0) & = 0, \quad x_3^0 = 0                                                      \\
        \phi_3(1) & = \max \begin{cases}
                               f_3(1) + \phi_2(0) \\
                               f_3(0) + \phi_2(1)
                           \end{cases} = \max \begin{cases}
                                                  5 + 0 \\
                                                  0 + 4
                                              \end{cases} = 5, \quad x_3^1 = 1                \\
        \phi_3(2) & = \max \begin{cases}
                               f_3(2) + \phi_2(0) \\
                               f_3(1) + \phi_2(1) \\
                               f_3(0) + \phi_2(2)
                           \end{cases} = \max \begin{cases}
                                                  2 + 0 \\
                                                  5 + 4 \\
                                                  0 + 6
                                              \end{cases} = 9, \quad x_3^1 = 1                \\
        \phi_3(3) & = \max \begin{cases}
                               f_3(3) + \phi_2(0) \\
                               f_3(2) + \phi_2(1) \\
                               f_3(1) + \phi_2(2) \\
                               f_3(0) + \phi_2(3)
                           \end{cases} = \max \begin{cases}
                                                  4 + 0 \\
                                                  2 + 4 \\
                                                  5 + 6 \\
                                                  0 + 8
                                              \end{cases} = 11, \quad x_3^3 = 1               \\
        \phi_3(4) & = \max \begin{cases}
                               f_3(4) + \phi_2(0) \\
                               f_3(3) + \phi_2(1) \\
                               f_3(2) + \phi_2(2) \\
                               f_3(1) + \phi_2(3) \\
                               f_3(0) + \phi_2(4)
                           \end{cases} = \max \begin{cases}
                                                  3 + 0 \\
                                                  4 + 4 \\
                                                  2 + 6 \\
                                                  5 + 8 \\
                                                  0 + 11
                                              \end{cases} = 13, \quad x_3^4 = 1               \\
        \phi_3(5) & = \max \begin{cases}
                               f_3(5) + \phi_2(0) \\
                               f_3(4) + \phi_2(1) \\
                               f_3(3) + \phi_2(2) \\
                               f_3(2) + \phi_2(3) \\
                               f_3(1) + \phi_2(4) \\
                               f_3(0) + \phi_2(5)
                           \end{cases} = \max \begin{cases}
                                                  6 + 0  \\
                                                  3 + 4  \\
                                                  4 + 6  \\
                                                  2 + 8  \\
                                                  5 + 11 \\
                                                  0 + 13
                                              \end{cases} = 16, \quad x_3^4 = 1               \\
        \phi_3(6) & = \max \begin{cases}
                               f_3(6) + \phi_2(0) \\
                               f_3(5) + \phi_2(1) \\
                               f_3(4) + \phi_2(2) \\
                               f_3(3) + \phi_2(3) \\
                               f_3(2) + \phi_2(4) \\
                               f_3(1) + \phi_2(5) \\
                               f_3(0) + \phi_2(6)
                           \end{cases} = \max \begin{cases}
                                                  1 + 0  \\
                                                  6 + 4  \\
                                                  3 + 6  \\
                                                  4 + 8  \\
                                                  2 + 11 \\
                                                  5 + 13 \\
                                                  0 + 13
                                              \end{cases} = 18, \quad x_3^6 = 1               \\
    \end{aligned}
\]

4. Для четвертого предприятия:
\[
    \begin{aligned}
        \phi_4(x) & = \max \left[ f_4(x_4) + \phi_3(x - x_4) \right], \quad 0 \leq x_4 \leq x \\
        \phi_4(0) & = 0, \quad x_4^0 = 0                                                      \\
        \phi_4(1) & = \max \begin{cases}
                               f_4(1) + \phi_3(0) \\
                               f_4(0) + \phi_3(1)
                           \end{cases} = \max \begin{cases}
                                                  3 + 0 \\
                                                  0 + 5
                                              \end{cases} = 5, \quad x_4^1 = 0                \\
        \phi_4(2) & = \max \begin{cases}
                               f_4(2) + \phi_3(0) \\
                               f_4(1) + \phi_3(1) \\
                               f_4(0) + \phi_3(2)
                           \end{cases} = \max \begin{cases}
                                                  5 + 0 \\
                                                  3 + 5 \\
                                                  0 + 9
                                              \end{cases} = 9, \quad x_4^1 = 0                \\
        \phi_4(3) & = \max \begin{cases}
                               f_4(3) + \phi_3(0) \\
                               f_4(2) + \phi_3(1) \\
                               f_4(1) + \phi_3(2) \\
                               f_4(0) + \phi_3(3)
                           \end{cases} = \max \begin{cases}
                                                  2 + 0 \\
                                                  5 + 5 \\
                                                  3 + 9 \\
                                                  0 + 11
                                              \end{cases} = 12, \quad x_4^3 = 1               \\
        \phi_4(4) & = \max \begin{cases}
                               f_4(4) + \phi_3(0) \\
                               f_4(3) + \phi_3(1) \\
                               f_4(2) + \phi_3(2) \\
                               f_4(1) + \phi_3(3) \\
                               f_4(0) + \phi_3(4)
                           \end{cases} = \max \begin{cases}
                                                  4 + 0  \\
                                                  2 + 5  \\
                                                  5 + 9  \\
                                                  3 + 11 \\
                                                  0 + 13
                                              \end{cases} = 14, \quad x_4^4 = 1               \\
        \phi_4(5) & = \max \begin{cases}
                               f_4(5) + \phi_3(0) \\
                               f_4(4) + \phi_3(1) \\
                               f_4(3) + \phi_3(2) \\
                               f_4(2) + \phi_3(3) \\
                               f_4(1) + \phi_3(4) \\
                               f_4(0) + \phi_3(5)
                           \end{cases} = \max \begin{cases}
                                                  1 + 0  \\
                                                  4 + 5  \\
                                                  2 + 9  \\
                                                  5 + 11 \\
                                                  3 + 13 \\
                                                  0 + 16
                                              \end{cases} = 16, \quad x_4^4 = 0               \\
        \phi_4(6) & = \max \begin{cases}
                               f_4(6) + \phi_3(0) \\
                               f_4(5) + \phi_3(1) \\
                               f_4(4) + \phi_3(2) \\
                               f_4(3) + \phi_3(3) \\
                               f_4(2) + \phi_3(4) \\
                               f_4(1) + \phi_3(5) \\
                               f_4(0) + \phi_3(6)
                           \end{cases} = \max \begin{cases}
                                                  6 + 0  \\
                                                  1 + 5  \\
                                                  4 + 9  \\
                                                  2 + 11 \\
                                                  5 + 13 \\
                                                  3 + 16 \\
                                                  0 + 18
                                              \end{cases} = 19, \quad x_4^6 = 1               \\
    \end{aligned}
\]

5. Для пятого предприятия:
\[
    \begin{aligned}
        \phi_5(x) & = \max \left[ f_5(x_5) + \phi_4(x - x_5) \right], \quad 0 \leq x_5 \leq x \\
        \phi_5(0) & = 0, \quad x_5^0 = 0                                                      \\
        \phi_5(1) & = \max \begin{cases}
                               f_5(1) + \phi_4(0) \\
                               f_5(0) + \phi_4(1)
                           \end{cases} = \max \begin{cases}
                                                  7 + 0 \\
                                                  0 + 5
                                              \end{cases} = 7, \quad x_5^1 = 1                \\
        \phi_5(2) & = \max \begin{cases}
                               f_5(2) + \phi_4(0) \\
                               f_5(1) + \phi_4(1) \\
                               f_5(0) + \phi_4(2)
                           \end{cases} = \max \begin{cases}
                                                  4 + 0 \\
                                                  7 + 5 \\
                                                  0 + 9
                                              \end{cases} = 12, \quad x_5^1 = 1               \\
        \phi_5(3) & = \max \begin{cases}
                               f_5(3) + \phi_4(0) \\
                               f_5(2) + \phi_4(1) \\
                               f_5(1) + \phi_4(2) \\
                               f_5(0) + \phi_4(3)
                           \end{cases} = \max \begin{cases}
                                                  6 + 0 \\
                                                  4 + 5 \\
                                                  7 + 9 \\
                                                  0 + 12
                                              \end{cases} = 16, \quad x_5^3 = 1               \\
        \phi_5(4) & = \max \begin{cases}
                               f_5(4) + \phi_4(0) \\
                               f_5(3) + \phi_4(1) \\
                               f_5(2) + \phi_4(2) \\
                               f_5(1) + \phi_4(3) \\
                               f_5(0) + \phi_4(4)
                           \end{cases} = \max \begin{cases}
                                                  2 + 0  \\
                                                  6 + 5  \\
                                                  4 + 9  \\
                                                  7 + 12 \\
                                                  0 + 14
                                              \end{cases} = 19, \quad x_5^4 = 1               \\
        \phi_5(5) & = \max \begin{cases}
                               f_5(5) + \phi_4(0) \\
                               f_5(4) + \phi_4(1) \\
                               f_5(3) + \phi_4(2) \\
                               f_5(2) + \phi_4(3) \\
                               f_5(1) + \phi_4(4) \\
                               f_5(0) + \phi_4(5)
                           \end{cases} = \max \begin{cases}
                                                  5 + 0  \\
                                                  2 + 5  \\
                                                  6 + 9  \\
                                                  4 + 12 \\
                                                  7 + 14 \\
                                                  0 + 16
                                              \end{cases} = 21, \quad x_5^4 = 1               \\
        \phi_5(6) & = \max \begin{cases}
                               f_5(6) + \phi_4(0) \\
                               f_5(5) + \phi_4(1) \\
                               f_5(4) + \phi_4(2) \\
                               f_5(3) + \phi_4(3) \\
                               f_5(2) + \phi_4(4) \\
                               f_5(1) + \phi_4(5) \\
                               f_5(0) + \phi_4(6)
                           \end{cases} = \max \begin{cases}
                                                  3 + 0  \\
                                                  5 + 5  \\
                                                  2 + 9  \\
                                                  6 + 12 \\
                                                  4 + 14 \\
                                                  7 + 16 \\
                                                  0 + 19
                                              \end{cases} = 23, \quad x_5^6 = 1               \\
    \end{aligned}
\]

6. Для шестого предприятия:
\[
    \begin{aligned}
        \phi_6(x) & = \max \left[ f_6(x_6) + \phi_5(x - x_6) \right], \quad 0 \leq x_6 \leq x \\
        \phi_6(0) & = 0, \quad x_6^0 = 0                                                      \\
        \phi_6(1) & = \max \begin{cases}
                               f_6(1) + \phi_5(0) \\
                               f_6(0) + \phi_5(1)
                           \end{cases} = \max \begin{cases}
                                                  1 + 0 \\
                                                  0 + 7
                                              \end{cases} = 7, \quad x_6^1 = 0                \\
        \phi_6(2) & = \max \begin{cases}
                               f_6(2) + \phi_5(0) \\
                               f_6(1) + \phi_5(1) \\
                               f_6(0) + \phi_5(2)
                           \end{cases} = \max \begin{cases}
                                                  8 + 0 \\
                                                  1 + 7 \\
                                                  0 + 12
                                              \end{cases} = 12, \quad x_6^1 = 0               \\
        \phi_6(3) & = \max \begin{cases}
                               f_6(3) + \phi_5(0) \\
                               f_6(2) + \phi_5(1) \\
                               f_6(1) + \phi_5(2) \\
                               f_6(0) + \phi_5(3)
                           \end{cases} = \max \begin{cases}
                                                  5 + 0  \\
                                                  8 + 7  \\
                                                  1 + 12 \\
                                                  0 + 16
                                              \end{cases} = 16, \quad x_6^3 = 0               \\
        \phi_6(4) & = \max \begin{cases}
                               f_6(4) + \phi_5(0) \\
                               f_6(3) + \phi_5(1) \\
                               f_6(2) + \phi_5(2) \\
                               f_6(1) + \phi_5(3) \\
                               f_6(0) + \phi_5(4)
                           \end{cases} = \max \begin{cases}
                                                  7 + 0  \\
                                                  5 + 7  \\
                                                  8 + 12 \\
                                                  1 + 16 \\
                                                  0 + 19
                                              \end{cases} = 20, \quad x_6^4 = 2               \\
        \phi_6(5) & = \max \begin{cases}
                               f_6(5) + \phi_5(0) \\
                               f_6(4) + \phi_5(1) \\
                               f_6(3) + \phi_5(2) \\
                               f_6(2) + \phi_5(3) \\
                               f_6(1) + \phi_5(4) \\
                               f_6(0) + \phi_5(5)
                           \end{cases} = \max \begin{cases}
                                                  3 + 0  \\
                                                  7 + 7  \\
                                                  5 + 12 \\
                                                  8 + 16 \\
                                                  1 + 19 \\
                                                  0 + 21
                                              \end{cases} = 24, \quad x_6^4 = 2               \\
        \phi_6(6) & = \max \begin{cases}
                               f_6(6) + \phi_5(0) \\
                               f_6(5) + \phi_5(1) \\
                               f_6(4) + \phi_5(2) \\
                               f_6(3) + \phi_5(3) \\
                               f_6(2) + \phi_5(4) \\
                               f_6(1) + \phi_5(5) \\
                               f_6(0) + \phi_5(6)
                           \end{cases} = \max \begin{cases}
                                                  4 + 0  \\
                                                  3 + 7  \\
                                                  7 + 12 \\
                                                  5 + 16 \\
                                                  8 + 19 \\
                                                  1 + 21 \\
                                                  0 + 23
                                              \end{cases} = 27, \quad x_6^6 = 2               \\
    \end{aligned}
\]

\[
    \begin{aligned}
         & X^0 = (1, 0, 1, 1, 1, 2), \quad \text{Прибыль} = 27.
    \end{aligned}
\]

Используя код на Go:

\begin{verbatim}
    package main

    import (
        "fmt"
    )
    
    // Размеры задачи
    const (
        Resources = 6
        Firms     = 6
    )
    
    func main() {
        // Матрица доходности
        F := [Resources + 1][Firms]int{
            {0, 0, 0, 0, 0, 0},
            {4, 2, 5, 3, 7, 1},
            {6, 3, 2, 5, 4, 8},
            {3, 7, 4, 2, 6, 5},
            {5, 1, 3, 4, 2, 7},
            {2, 4, 6, 1, 5, 3},
            {7, 5, 1, 6, 3, 4},
        }
    
        // Таблица для хранения максимального дохода
        maxProfit := make([][]int, Firms+1)
        for i := range maxProfit {
            maxProfit[i] = make([]int, Resources+1)
        }
    
        // Таблица для хранения распределения ресурсов
        allocation := make([][]int, Firms+1)
        for i := range allocation {
            allocation[i] = make([]int, Resources+1)
        }
    
        // Динамическое программирование
        for firm := 1; firm <= Firms; firm++ {
            for resource := 0; resource <= Resources; resource++ {
                for used := 0; used <= resource; used++ {
                    profit := F[used][firm-1] + maxProfit[firm-1][resource-used]
                    if profit > maxProfit[firm][resource] {
                        maxProfit[firm][resource] = profit
                        allocation[firm][resource] = used
                    }
                }
            }
        }
    
        // Вывод результата
        fmt.Println("Максимальный доход:", maxProfit[Firms][Resources])
    
        // Восстановление распределения ресурсов
        resourceLeft := Resources
        allocResult := make([]int, Firms)
        for firm := Firms; firm > 0; firm-- {
            allocResult[firm-1] = allocation[firm][resourceLeft]
            resourceLeft -= allocResult[firm-1]
        }
    
        fmt.Println("Распределение ресурсов:", allocResult)
    }
\end{verbatim}

Получаем такие же значения:
\begin{verbatim}
    Максимальный доход: 27
    Распределение ресурсов: [1 0 1 1 1 2]
\end{verbatim}

\text{Результаты совапали.}

\newpage
\section{Задание №10}
\subsection{Условие}

Имеется рюкзак грузоподъемностью $W$. \\
$P_i$ -- вес одного предмета $i$-ого типа. \\
$V_i$ -- стоимость (ценность) одного предмета $i$-ого типа. \\
$X_i$ -- число предметов $i$-ого типа, которые будут загружаться на транспортировочное средство. \\

Требуется заполнить рюкзак грузом, состоящим из предметов $N$ различных типов, таким образом, чтобы стоимость (ценность) всего груза была максимальной:
\[
    \sum_{i=1}^N V_i x_i \to \max,
\]
при условиях:
\[
    \sum_{i=1}^N P_i x_i \leq W, \quad x_i \in \{0, 1, \dots\},
\]
где $x_i$ -- целое неотрицательное число.

\subsection*{Рекуррентное решение}
Решение задачи разбивается на $n$ этапов, на каждом из которых определяется максимальная стоимость груза, состоящего из предметов:
\begin{itemize}
    \item 1-го типа (1-ый этап),
    \item 1-го и 2-го типов (2-ой этап),
    \item $\dots$
    \item всех $n$ типов (последний этап).
\end{itemize}

Рекуррентное уравнение Беллмана для задачи:
\[
    W_i(C) = \max_{0 \leq x_i \leq \lfloor C / P_i \rfloor} \{ V_i x_i + W_{i-1}(C - P_i x_i) \},
\]
где $W_i(C)$ -- максимальная стоимость груза, состоящего из предметов типов $1, \dots, i$, при суммарном весе $C$. Начальные условия:
\[
    W_0(C) = 0 \quad \text{при $0 \leq C \leq S$}.
\]

\subsection*{Пример решения задачи}
Пусть $S = 20$ -- грузоподъемность рюкзака. Данные:
\[
    \begin{array}{|c|c|c|c|c|c|c|}
        \hline
        C & 1  & 2  & 3  & 4  & 5  & 6  \\
        \hline
        P & 5  & 4  & 7  & 3  & 6  & 2  \\
        \hline
        V & 50 & 40 & 70 & 30 & 60 & 20 \\
        \hline
    \end{array}
\]

\paragraph*{Шаг 1. Расчет $W_1(C)$:}
\[
    W_1(C) = \max_{0 \leq x_1 \leq \frac{20}{5}} \{ 50 \cdot x_1 \}, \quad \text{при } x_1 \in \{0, 1, 2, 3\}.
\]
Результат:
\[
    \begin{array}{|c|c|c|c|c|c|c|c|c|c|c|c|c|c|c|c|c|c|c|c|c|c|}
        \hline
        C      & 0 & 1 & 2 & 3 & 4 & 5  & 6  & 7  & 8  & 9  & 10  & 11  & 12  & 13  & 14  & 15  & 16  & 17  & 18  & 19  & 20  \\
        \hline
        W_1(C) & 0 & 0 & 0 & 0 & 0 & 50 & 50 & 50 & 50 & 50 & 100 & 100 & 100 & 100 & 100 & 150 & 150 & 150 & 150 & 150 & 200 \\
        \hline
        x_1    & 0 & 0 & 0 & 0 & 0 & 1  & 1  & 1  & 1  & 1  & 2   & 2   & 2   & 2   & 2   & 3   & 3   & 3   & 3   & 3   & 4   \\
        \hline
    \end{array}
\]

\paragraph*{Шаг 2. Расчет $W_2(C)$:}
\[
    W_2(C) = \max_{0 \leq x_2 \leq \frac{20}{4}} \{ 40 \cdot x_2 + W_1(C - 4 x_2) \}, \quad \text{при } x_2 \in \{0, 1, 2, 3, 4, 5\}.
\]
Результат:
\[
    \begin{array}{|c|c|c|c|c|c|c|c|c|c|c|c|c|c|c|c|c|c|c|c|c|c|}
        \hline
        C      & 0 & 1 & 2 & 3 & 4  & 5  & 6  & 7  & 8  & 9  & 10  & 11  & 12  & 13  & 14  & 15  & 16  & 17  & 18  & 19  & 20  \\
        \hline
        W_2(C) & 0 & 0 & 0 & 0 & 40 & 50 & 50 & 50 & 80 & 90 & 100 & 100 & 120 & 130 & 140 & 150 & 160 & 170 & 180 & 190 & 200 \\
        \hline
        x_2    & 0 & 0 & 0 & 0 & 1  & 0  & 0  & 0  & 2  & 1  & 0   & 0   & 3   & 2   & 1   & 0   & 4   & 3   & 2   & 1   & 0   \\
        \hline
    \end{array}
\]

\paragraph*{Шаг 3. Расчет $W_3(C)$:}
\[
    W_3(C) = \max_{0 \leq x_3 \leq \frac{20}{7}} \{ 70 \cdot x_3 + W_2(C - 7 x_3) \}, \quad \text{при } x_3 \in \{0, 1, 2\}.
\]
Результат:
\[
    \begin{array}{|c|c|c|c|c|c|c|c|c|c|c|c|c|c|c|c|c|c|c|c|c|c|}
        \hline
        C      & 0 & 1 & 2 & 3 & 4  & 5  & 6  & 7  & 8  & 9  & 10  & 11  & 12  & 13  & 14  & 15  & 16  & 17  & 18  & 19  & 20  \\
        \hline
        W_3(C) & 0 & 0 & 0 & 0 & 40 & 50 & 50 & 70 & 80 & 90 & 100 & 110 & 120 & 130 & 140 & 150 & 160 & 170 & 180 & 190 & 200 \\
        \hline
        x_3    & 0 & 0 & 0 & 0 & 0  & 0  & 0  & 1  & 0  & 0  & 0   & 1   & 0   & 0   & 0   & 0   & 0   & 0   & 0   & 0   & 0   \\
        \hline
    \end{array}
\]

\paragraph*{Шаг 4. Расчет $W_4(C)$:}
\[
    W_4(C) = \max_{0 \leq x_4 \leq \frac{20}{3}} \{ 30 \cdot x_4 + W_3(C - 3 x_4) \}, \quad \text{при } x_4 \in \{0, 1, 2, 3, 4, 5, 6\}.
\]
Результат:
\[
    \begin{array}{|c|c|c|c|c|c|c|c|c|c|c|c|c|c|c|c|c|c|c|c|c|c|}
        \hline
        C      & 0 & 1 & 2 & 3  & 4  & 5  & 6  & 7  & 8  & 9  & 10  & 11  & 12  & 13  & 14  & 15  & 16  & 17  & 18  & 19  & 20  \\
        \hline
        W_4(C) & 0 & 0 & 0 & 30 & 40 & 50 & 60 & 70 & 80 & 90 & 100 & 110 & 120 & 130 & 140 & 150 & 160 & 170 & 180 & 190 & 200 \\
        \hline
        x_4    & 0 & 0 & 0 & 1  & 0  & 0  & 2  & 0  & 0  & 0  & 0   & 0   & 0   & 0   & 0   & 0   & 0   & 0   & 0   & 0   & 0   \\
        \hline
    \end{array}
\]

\paragraph*{Шаг 5. Расчет $W_5(C)$:}
\[
    W_5(C) = \max_{0 \leq x_5 \leq \frac{20}{6}} \{ 60 \cdot x_5 + W_4(C - 6 x_5) \}, \quad \text{при } x_5 \in \{0, 1, 2, 3\}.
\]
Результат:
\[
    \begin{array}{|c|c|c|c|c|c|c|c|c|c|c|c|c|c|c|c|c|c|c|c|c|c|}
        \hline
        C      & 0 & 1 & 2 & 3  & 4  & 5  & 6  & 7  & 8  & 9  & 10  & 11  & 12  & 13  & 14  & 15  & 16  & 17  & 18  & 19  & 20  \\
        \hline
        W_5(C) & 0 & 0 & 0 & 30 & 40 & 50 & 60 & 70 & 80 & 90 & 100 & 110 & 120 & 130 & 140 & 150 & 160 & 170 & 180 & 190 & 200 \\
        \hline
        x_5    & 0 & 0 & 0 & 0  & 0  & 0  & 0  & 0  & 0  & 0  & 0   & 0   & 0   & 0   & 0   & 0   & 0   & 0   & 0   & 0   & 0   \\
        \hline
    \end{array}
\]

\paragraph*{Шаг 6. Расчет $W_6(C)$:}
\[
    W_6(C) = \max_{0 \leq x_6 \leq \frac{20}{2}} \{ 20 \cdot x_6 + W_5(C - 2 x_6) \}, \quad \text{при } x_6 \in \{0, 1, 2, \dots, 10\}.
\]
Результат:
\[
    \begin{array}{|c|c|c|c|c|c|c|c|c|c|c|c|c|c|c|c|c|c|c|c|c|c|}
        \hline
        C      & 0 & 1 & 2  & 3  & 4  & 5  & 6  & 7  & 8  & 9  & 10  & 11  & 12  & 13  & 14  & 15  & 16  & 17  & 18  & 19  & 20  \\
        \hline
        W_6(C) & 0 & 0 & 20 & 30 & 40 & 50 & 60 & 70 & 80 & 90 & 100 & 110 & 120 & 130 & 140 & 150 & 160 & 170 & 180 & 190 & 200 \\
        \hline
        x_6    & 0 & 0 & 1  & 0  & 0  & 0  & 0  & 0  & 0  & 0  & 0   & 0   & 0   & 0   & 0   & 0   & 0   & 0   & 0   & 0   & 0   \\
        \hline
    \end{array}
\]

Таким образом, максимальная стоимость груза $W_6(20)$ равна 200 денежным единицам.
При этом $x_6 = 0$, так как $W_6(20) = 200$ достигается при $x_6=0$.
Предметы остальных типов распределяются следующим образом:
\[
    C = 20 - 2 \cdot 0 = 20
\]
$W_5(20) = 200$ достигается при $x_5 = 0$.
\[
    C = 20 - 6 \cdot 0 = 20
\]
$W_4(20) = 200$ достигается при $x_4 = 0$.
\[
    C = 20 - 3 \cdot 0 = 20
\]
$W_3(20) = 200$ достигается при $x_3 = 0$.
\[
    C = 20 - 7 \cdot 0 = 20
\]
$W_2(20) = 200$ достигается при $x_2 = 0$.
\[
    C = 20 - 4 \cdot 0 = 20
\]
$W_1(20) = 200$ достигается при $x_1 = 4$.
\[
    C = 20 - 5 \cdot 4 = 0
\]\\
Итоговое решение: $X = (4, 0, 0, 0, 0, 0)$, максимальная ценность $200$.

Используя код на Go:

\begin{verbatim}
package main

import (
"fmt"
)

func main() {
S := 20
P := []int{5, 4, 7, 3, 6, 2}
V := []int{50, 40, 70, 30, 60, 20}
n := len(P)

W := make([][]int, n+1)
X := make([][]int, n+1)
for i := range W {
        W[i] = make([]int, S+1)
        X[i] = make([]int, S+1)
    }

for i := 1; i <= n; i++ {
for C := 0; C <= S; C++ {
maxValue := 0
maxX := 0
for x := 0; x <= C/P[i-1]; x++ {
value := V[i-1]*x + W[i-1][C-P[i-1]*x]
if value > maxValue {
        maxValue = value
        maxX = x
    }
}
W[i][C] = maxValue
X[i][C] = maxX
}
}

fmt.Printf("Максимальная стоимость груза W_%d(%d) равна %d денежным единицам.\n", n, S, W[n][S])
fmt.Printf("Итоговое решение: X = (")

C := S
for i := n; i > 0; i-- {
fmt.Printf("%d", X[i][C])
if i > 1 {
        fmt.Printf(", ")
    }
C -= P[i-1] * X[i][C]
}
fmt.Printf("), максимальная ценность %d.\n", W[n][S])
}
\end{verbatim}

Получаем такие же значения:
\begin{verbatim}
Максимальная стоимость груза W_6(20) равна 200 денежным единицам.
Итоговое решение: X = (0, 0, 0, 0, 0, 4), максимальная ценность 200.
\end{verbatim}

\text{Результаты совапали.}
\end{document}
