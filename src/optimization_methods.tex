\documentclass{article}
\usepackage{amsmath}
\usepackage{pgfplots}
\usepackage{geometry}
\usepackage{hyperref}
\usepackage{verbatim}
\geometry{a4paper, margin=1in}
\usepackage[utf8]{inputenc}
\usepackage[T2A]{fontenc}
\usepackage[russian]{babel}
\usepackage{colortbl}
\usepackage{xcolor}
\usetikzlibrary{intersections}
\usepgfplotslibrary{fillbetween}
\pgfplotsset{compat=1.18}
\usepackage{graphicx}

\hypersetup{
    colorlinks=true,   
    linkcolor=black,     
    urlcolor=blue,       
    citecolor=blue,      
    filecolor=blue,        
    pdfborder={0 0 0}     
}


\begin{document}

% Титульный лист
\begin{titlepage}
    \centering
    {\large Министерство науки и высшего образования Российской Федерации\\[0.5cm]}
    {\Large ФЕДЕРАЛЬНОЕ ГОСУДАРСТВЕННОЕ АВТОНОМНОЕ ОБРАЗОВАТЕЛЬНОЕ УЧРЕЖДЕНИЕ\\
        ВЫСШЕГО ОБРАЗОВАНИЯ}\\[0.5cm]
    {\Large Национальный исследовательский ядерный университет «МИФИ»\\
    (НИЯУ МИФИ)}\\[1cm]
    {\large ИНСТИТУТ ИНТЕЛЛЕКТУАЛЬНЫХ КИБЕРНЕТИЧЕСКИХ СИСТЕМ\\
    КАФЕДРА КИБЕРНЕТИКИ}\\[1cm]
    \includegraphics[width=0.3\textwidth]{IIKS.png}\\[1.5cm] % Логотип университета
    {\huge Отчет по курсу «Методы оптимизации»}\\[5cm]

    \begin{flushright}
        \textbf{Выполнил:}\\
        Студент группы Б22-534\\
        Зацепилин А.В.\\[0.5cm]
        \textbf{Вариант №55}
    \end{flushright}

    \vfill
    Москва, осень 2024
\end{titlepage}

\tableofcontents
\newpage

\section{Задание №1}

\subsection{Задача (a)}

Оптимизационная задача:
\[
    F = 3x_1 + 4x_2 \to \max, \ \min
\]
Ограничения:
\[
    \begin{cases}
        x_1 - 3 \cdot x_2 \leq 3  \\
        x_1 + x_2 \leq 10         \\
        -x_1 + 4 \cdot x_2 \leq 4 \\
        x_1 \geq 0, \ x_2 \geq 0
    \end{cases}
\]

\begin{figure}[h]
    \centering
    \begin{tikzpicture}
        \begin{axis}[
                xlabel={$x_1$},
                ylabel={$x_2$},
                axis lines=middle,
                xmin=0, xmax=10,
                ymin=0, ymax=10,
                grid=both,
                width=14cm,
                height=12cm,
                legend pos=north east
            ]
            % Линии ограничений с разными цветами
            \addplot[name path=A, domain=0:10, thick, blue] { (x - 3)/3 };
            \addplot[name path=B, domain=0:10, thick, red] { 10 - x };
            \addplot[name path=C, domain=0:10, thick, green] { (4 + x)/4 };

            % Область допустимых решений
            \addplot[fill opacity=0.2, color=gray] fill between[
                    of=A and B,
                    soft clip={domain=6:10}
                ];

            % Вектор {3, 4}
            \addplot[->, thick, black] coordinates {(0,0) (3,4)};
            \node at (axis cs: 3,4) [anchor=west] {$\vec{v} = \{3,4\}$};

            % Подписи
            \legend{
                $x_1 - 3x_2 = 3$,
                $x_1 + x_2 = 10$,
                $-x_1 + 4x_2 = 4$
            }

        \end{axis}
    \end{tikzpicture}

    \caption{Графическое решение задачи (a)}
\end{figure}

Вычисление точек пересечения

1. Минимум \(x_1 - 3x_2 = 3\) и \(x_2 = 0\):
\[
    \begin{cases}
        x_1 = 3 + 3x_2 \\
        x_2 = 0 \implies x_1 =  3
    \end{cases}
\]
Точка пересечения: \((3,0)\).

2. Максимум \(x_1 + x_2 = 10\) и \(-x_1 + 4x_2 = 4\):
\[
    \begin{cases}
        x_1 + x_2 = 10 \\
        -x_1 + 4x_2 = 4
    \end{cases}
    \implies
    \begin{cases}
        x_1 = \frac{36}{5} \\
        x_2 = \frac{14}{5}
    \end{cases}
\]
Точка пересечения: \(\left(\frac{36}{5}, \frac{14}{5}\right)\).

Вычисление значений целевой функции \(F = 3x_1 + 4x_2\) в найденных точках:

- В точке \((3, 0)\):
\[
    F(3, 0) = 9
\]

- В точке \(\left(\frac{36}{5}, \frac{14}{5}\right)\):
\[
    F\left(\frac{36}{5}, \frac{14}{5}\right) = \frac{164}{5}
\]

Ответ:
\begin{tabular}{l l}
    Максимум функции \(F = 3x_1 + 4x_2\) & достигается в точке \(\left(\frac{36}{5}, \frac{14}{5}\right)\), где \(F =\frac{164}{5}\). \\
    Минимум функции \(F = 3x_1 + 4x_2\)  & достигается в точке \((3, 0)\), где \(F = 9\).
\end{tabular}

\newpage

\subsection{Задача (b)}

Оптимизационная задача:
\[
    F = x_1 + 7x_2 \to \max, \ \min
\]
Ограничения:
\[
    \begin{cases}
        3 \cdot x_1 + 4 \cdot x_2 \geq 12 \\
        2 \cdot x_1 - x_2 \leq 6          \\
        x_1 \geq 0, \ x_2 \geq 0
    \end{cases}
\]

Построение графиков

\begin{figure}[h]
    \centering
    \begin{tikzpicture}
        \begin{axis}[
                xlabel={$x_1$},
                ylabel={$x_2$},
                axis lines=middle,
                xmin=0, xmax=10,
                ymin=0, ymax=10,
                grid=both,
                width=14cm,
                height=12cm,
                legend pos=north east
            ]
            % Линии ограничений с разными цветами
            \addplot[name path=A, domain=0:10, thick, blue] { (12 - 3*x)/4 };
            \addplot[name path=B, domain=0:10, thick, red] { 2*x - 6 };

            % Область допустимых решений
            \addplot[fill opacity=0.2, color=gray] fill between[
                    of=A and B,
                    soft clip={domain=0:10}
                ];

            \addplot[->, thick, black] coordinates {(0,0) (1,7)};
            \node at (axis cs: 1,7) [anchor=west] {$\vec{v} = \{1,7\}$};

            % Подписи
            \legend{
                $3x_1 + 4x_2 = 12$,
                $2x_1 - x_2 = 6$
            }
        \end{axis}
    \end{tikzpicture}
    \caption{Графическое решение задачи (b)}
\end{figure}

Вычисление точек пересечения

1. Минимум \(3x_1 + 4x_2 = 12\) и \(2x_1 - x_2 = 6\):
\[
    \begin{cases}
        3x_1 + 4x_2 = 12 \\
        2x_1 - x_2 = 6
    \end{cases} \implies
    \begin{cases}
        x_1 = \frac{36}{11} \\
        x_2 = \frac{6}{11}
    \end{cases}
\]
Точка пересечения: \(\left(\frac{36}{11}, \frac{6}{11}\right)\).

Вычисление значений целевой функции \(F = x_1 + 7x_2\) в найденной точке:

В точке \(\left(\frac{36}{11}, \frac{6}{11}\right)\):
\[
    F\left(\frac{36}{11}, \frac{6}{11}\right) = \frac{78}{11}
\]

Ответ:
\begin{tabular}{l l}
    Максимума функции не существует.                                                                                                  \\
    Минимум функции \(F = x_1 + 7x_2\) & достигается в точке \(\left(\frac{36}{11}, \frac{6}{11}\right)\), где \(F = \frac{78}{11}\).
\end{tabular}

\newpage

\subsection{Задача (c)}

Оптимизационная задача:
\[
    F = 2x_1 + x_2 \to \max, \ \min
\]
Ограничения:
\[
    \begin{cases}
        3 \cdot x_1 - x_2 \geq 9 \\
        x_1 + x_2 \leq 2         \\
        x_1 \geq 0, \ x_2 \geq 0
    \end{cases}
\]

Построение графиков

\begin{figure}[h]
    \centering
    \begin{tikzpicture}
        \begin{axis}[
                xlabel={$x_1$},
                ylabel={$x_2$},
                axis lines=middle,
                xmin=-5, xmax=10,
                ymin=-5, ymax=10,
                grid=both,
                width=14cm,
                height=12cm,
                legend pos=north east
            ]
            % Линии ограничений с разными цветами
            \addplot[name path=A, domain=0:10, thick, blue] { 3*x - 9 };
            \addplot[name path=B, domain=0:10, thick, red] { 2 - x };

            % Область допустимых решений
            \addplot[fill opacity=0.2, color=gray] fill between[
                    of=A and B,
                    soft clip={domain=0:10}
                ];

            \addplot[->, thick, black] coordinates {(0,0) (2,1)};
            \node at (axis cs: 2,1) [anchor=west] {$\vec{v} = \{2,1\}$};

            % Подписи
            \legend{
                $3x_1 - x_2 = 9$,
                $x_1 + x_2 = 2$
            }
        \end{axis}
    \end{tikzpicture}
    \caption{Графическое решение задачи (c)}
\end{figure}

Ответ:
\begin{tabular}{l l}
    Максимума функции не существует. \\
    Минимума функции не существует.
\end{tabular}

\section{Задание №2}

\subsection{Задача (a)}

Оптимизационная задача (из задачи 1.1):
\[
    F = 3x_1 + 4x_2 \to \max
\]
Ограничения:
\[
    \begin{cases}
        x_1 - 3 \cdot x_2 \leq 3  \\
        x_1 + x_2 \leq 10         \\
        -x_1 + 4 \cdot x_2 \leq 4 \\
        x_1 \geq 0, \ x_2 \geq 0
    \end{cases}
\]

Решение задачи с помощью симплекс-метода:
\[
    F - 3x_1 - 4x_2 = 0
\]

\[
    \begin{cases}
        x_1 - 3 \cdot x_2 + x_3 = 3  \\
        x_1 + x_2 + x_4 = 10         \\
        -x_1 + 4 \cdot x_2 + x_5 = 4 \\
        x_1 \geq 0, \ x_2 \geq 0, \ x_3 \geq 0, \ x_4 \geq 0, \ x_5 \geq 0
    \end{cases}
\]

\vspace{25pt}

\[
    \begin{array}{|c|c|>{\columncolor{green}}c|c|c|c|c|c|}
        \hline
             & x_1 & x_2 & x_3 & x_4 & x_5 & b_i & b_i/p.c. \geq 0 \\
        \hline
        x_3  & 1   & -3  & 1   & 0   & 0   & 3   & -1              \\
        x_4  & 1   & 1   & 0   & 1   & 0   & 10  & 10              \\
        \rowcolor{yellow}
        x_5  & -1  & 4   & 0   & 0   & 1   & 4   & 1               \\
        F(x) & -3  & -4  & 0   & 0   & 0   & 0   & 0               \\
        \hline
    \end{array}
\]

В последней строке есть элементы \(\leq 0\). Занулим элементы выше и ниже стоящие от разрешающего элемента.

\[
    \begin{array}{|c|>{\columncolor{green}}c|c|c|c|c|c|c|}
        \hline
             & x_1          & x_2 & x_3 & x_4 & x_5          & b_i & b_i/p.c. \geq 0 \\
        \hline
        x_2  & -\frac{1}{4} & 1   & 0   & 0   & \frac{1}{4}  & 1   & -4              \\
        x_3  & \frac{1}{4}  & 0   & 1   & 0   & \frac{3}{4}  & 6   & 24              \\
        \rowcolor{yellow}
        x_4  & \frac{5}{4}  & 0   & 0   & 1   & -\frac{1}{4} & 9   & \frac{36}{5}    \\
        F(x) & -4           & 0   & 0   & 0   & 1            & 4   &                 \\
        \hline
    \end{array}
\]

В последней строке есть элементы \(\leq 0\). Занулим элементы выше и ниже стоящие от разрешающего элемента.

\[
    \begin{array}{|c|c|c|c|c|c|c|c|}
        \hline
             & x_1 & x_2 & x_3 & x_4          & x_5          & b_i           & b_i/p.c. \geq 0 \\
        \hline
        x_1  & 1   & 0   & 0   & \frac{4}{5}  & -\frac{1}{5} & \frac{36}{5}  & \frac{36}{5}    \\
        x_2  & 0   & 1   & 0   & \frac{1}{5}  & \frac{4}{20} & \frac{14}{5}  & -               \\
        x_3  & 0   & 0   & 1   & -\frac{1}{5} & \frac{4}{5}  & \frac{21}{5}  & 24              \\
        F(x) & 0   & 0   & 0   & \frac{16}{5} & \frac{1}{5}  & \frac{164}{5} &                 \\
        \hline
    \end{array}
\]

В последней строке не осталось элементов \(\leq 0\). Мы пришли к конечной таблице.\\
Максимум функции достигается при \(x_1 = \frac{36}{5}, x_2 = \frac{14}{5}\), и значение целевой функции равно \(F(x) = \frac{164}{5}\).

\newpage

\subsection{Задача (b)}

Оптимизационная задача (из задачи 1.2):
\[
    F = x_1 + 7x_2 \to \max
\]
Ограничения:
\[
    \begin{cases}
        3 \cdot x_1 + 4 \cdot x_2 \geq 12 \\
        2 \cdot x_1 - x_2 \leq 6          \\
        x_1 \geq 0, \ x_2 \geq 0
    \end{cases}
\]

Решение задачи с помощью симплекс-метода:
\[
    F - x_1 - 7x_2 = 0
\]

\[
    \begin{cases}
        -3 \cdot x_1 - 4 \cdot x_2 + x_3 =  -12 \\
        2 \cdot x_1 - x_2 + x_4 = 6             \\
        x_1 \geq 0, \ x_2 \geq 0
    \end{cases}
\]

\vspace{25pt}

\[
    \begin{array}{|c|c|>{\columncolor{pink}}c|c|c|c|c|}
        \hline
             & x_1 & x_2 & x_3 & x_4 & b_i & b_i/p.c. \geq 0 \\
        \hline
        x_3  & -3  & -4  & 1   & 0   & -12 & -               \\
        x_4  & 2   & -1  & 0   & 1   & 6   & -               \\
        F(x) & -1  & -7  & 0   & 0   & 0   & -               \\
        \hline
    \end{array}
\]

В последней строке есть элементы \(\leq 0\). Минимальный из них -7, но т.к. все элементы этого столбца отрицательные, то область допустимых решений неограниченна.


Оптимизационная задача (из задачи 1.2):
\[
    F = x_1 + 7x_2 \to \min
\]
Переведём эту задачу в поиск максимума взяв обратную функцию от изначальной.
\[
    G = -x_1 - 7x_2 \to \max
\]
Ограничения:
\[
    \begin{cases}
        3 \cdot x_1 + 4 \cdot x_2 \geq 12 \\
        2 \cdot x_1 - x_2 \leq 6          \\
        x_1 \geq 0, \ x_2 \geq 0
    \end{cases}
\]

Решение задачи с помощью симплекс-метода:
\[
    G + x_1 + 7x_2 = 0
\]

\[
    \begin{cases}
        -3 \cdot x_1 - 4 \cdot x_2 + x_3 =  -12 \\
        2 \cdot x_1 - x_2 + x_4 = 6             \\
        x_1 \geq 0, \ x_2 \geq 0
    \end{cases}
\]

\vspace{25pt}

\[
    \begin{array}{|c|>{\columncolor{green}}c|c|c|c|c|c|}
        \hline
             & x_1 & x_2 & x_3 & x_4 & b_i & b_i/p.c. \geq 0 \\
        \hline
        x_3  & -3  & -4  & 1   & 0   & -12 & 4               \\
        \rowcolor{yellow}
        x_4  & 2   & -1  & 0   & 1   & 6   & 3               \\
        G(x) & 1   & 7   & 0   & 0   & 0   & 0               \\
        \hline
    \end{array}
\]

В последней строке есть элементы \(\leq 0\). Занулим элементы выше и ниже стоящие от разрешающего элемента.

\[
    \begin{array}{|c|c|>{\columncolor{green}}c|c|c|c|c|}
        \hline
             & x_1 & x_2           & x_3 & x_4          & b_i & b_i/p.c. \geq 0 \\
        \hline
        x_1  & 1   & -\frac{1}{2}  & 0   & \frac{1}{2}  & 3   & -6              \\
        \rowcolor{yellow}
        x_3  & 0   & -\frac{11}{2} & 1   & \frac{3}{2}  & -3  & \frac{6}{11}    \\
        G(x) & 0   & \frac{15}{2}  & 0   & -\frac{1}{2} & -3  & -\frac{6}{15}   \\
        \hline
    \end{array}
\]

В последней строке есть элементы \(\leq 0\). Занулим элементы выше и ниже стоящие от разрешающего элемента.

\[
    \begin{array}{|c|c|c|c|c|c|c|}
        \hline
             & x_1 & x_2 & x_3           & x_4           & b_i           & b_i/p.c. \geq 0 \\
        \hline
        x_2  & 0   & 1   & -\frac{2}{11} & -\frac{3}{11} & \frac{6}{11}  & -               \\
        x_1  & 1   & 0   & -\frac{1}{11} & \frac{4}{11}  & \frac{36}{11} & -               \\
        G(x) & 0   & 0   & \frac{4}{165} & \frac{17}{11} & \frac{78}{11} & -               \\
        \hline
    \end{array}
\]

В последней строке не осталось элементов \(\leq 0\). Мы пришли к конечной таблице.\\
Максимум функции достигается при \(x_1 = \frac{36}{11}, x_2 = \frac{6}{11}\), и значение целевой функции равно \(F(x) = \frac{78}{11}\).

\newpage

\subsection{Задача (с)}

Оптимизационная задача (из задачи 1.3):
\[
    F = 2x_1 + x_2 \to \max
\]
Ограничения:
\[
    \begin{cases}
        3 \cdot x_1 - x_2 \geq 9 \\
        x_1 + x_2 \leq 2         \\
        x_1 \geq 0, \ x_2 \geq 0
    \end{cases}
\]

Решение задачи с помощью симплекс-метода:
\[
    F - 2x_1 - x_2 = 0
\]

\[
    \begin{cases}
        -3 \cdot x_1 + x_2 + x_3 =  -9 \\
        x_1 + x_2 + x_4 = 2            \\
        x_1 \geq 0, \ x_2 \geq 0
    \end{cases}
\]

\vspace{25pt}

\[
    \begin{array}{|c|>{\columncolor{green}}c|c|c|c|c|c|}
        \hline
             & x_1 & x_2 & x_3 & x_4 & b_i & b_i/p.c. \geq 0 \\
        \hline
        x_3  & -3  & 1   & 1   & 0   & -9  & -               \\
        \rowcolor{yellow}
        x_4  & 1   & 1   & 0   & 1   & 2   & 2               \\
        F(x) & -2  & -1  & 0   & 0   & 0   & -               \\
        \hline
    \end{array}
\]
Первую и последнюю строки не вычисляем для последнего столбца т.к. элементы р.с. \(\leq 0\).
В последней строке есть элементы \(\leq 0\). Занулим элементы выше и ниже стоящие от разрешающего элемента.

\[
    \begin{array}{|c|c|c|c|c|>{\columncolor{pink}}c|c|}
        \hline
             & x_1 & x_2 & x_3 & x_4 & b_i & b_i/p.c. \geq 0 \\
        \hline
        x_1  & 1   & 1   & 0   & 1   & 2   & -               \\
        x_3  & 0   & 4   & 1   & 3   & -3  & -               \\
        F(x) & 0   & 1   & 0   & 2   & 4   & -               \\
        \hline
    \end{array}
\]

В последней строке не осталось элементов \(\leq 0\).\\
Мы пришли к конечной таблице. Т.к. не все \(b_i\geq 0\) \(\implies\) решения не существует.


\newpage

\section{Задание №3}
\subsection{Задача (a)}

Оптимизационная задача (из задачи 1.1):
\[
    F = 3x_1 + 4x_2 \to \max
\]
Ограничения:
\[
    \begin{cases}
        x_1 - 3 \cdot x_2 \leq 3  \\
        x_1 + x_2 \leq 10         \\
        -x_1 + 4 \cdot x_2 \leq 4 \\
        x_1 \geq 0, \ x_2 \geq 0
    \end{cases}
\]

Решение задачи с помощью симплекс-метода:

\[
    \begin{array}{|c|c|c|c|c|c|c|c|}
        \hline
             & x_1 & x_2 & x_3 & x_4          & x_5          & b_i           & b_i/p.c. \geq 0 \\
        \hline
        x_1  & 1   & 0   & 0   & \frac{4}{5}  & -\frac{1}{5} & \frac{36}{5}  & \frac{36}{5}    \\
        x_2  & 0   & 1   & 0   & \frac{1}{5}  & \frac{4}{20} & \frac{14}{5}  & -               \\
        x_3  & 0   & 0   & 1   & -\frac{1}{5} & \frac{4}{5}  & \frac{21}{5}  & 24              \\
        F(x) & 0   & 0   & 0   & \frac{16}{5} & \frac{1}{5}  & \frac{164}{5} &                 \\
        \hline
    \end{array}
\]

Максимум функции достигается при \(x_1 = \frac{36}{5}, x_2 = \frac{14}{5}\), и значение целевой функции равно \(F(x) = \frac{164}{5}\).\\

Составим двойственную задачу:

\[
    F^* = 3y_1 + 10x_2 + 4y_3 \to \min
\]

Ограничения:
\[
    \begin{cases}
        y_1 + y_2 - y_3 \geq 3                  \\
        -3 \cdot y_1 + y_2 + 4 \cdot y_3 \geq 4 \\
        y_1 \geq 0, \ y_2 \geq 0, \ y_3 \geq 0
    \end{cases}
\]\\

Решим задачу 1 способом для этого составим систему для нахождения \(y_1^*, y_2^*, y_3^*\):\\

\[
    \begin{cases}
        (\frac{36}{5} - 3\frac{14}{5} - 3) \cdot y_1^* = 0 \\
        (\frac{36}{5} + \frac{14}{5} - 10) \cdot y_2^* = 0 \\
        (-\frac{36}{5} + 4\frac{14}{5} - 4) \cdot y_3^* = 0
    \end{cases} \implies
    \begin{cases}
        -\frac{8}{5} \cdot y_1^* = 0 \implies y_1^* = 0 \\
        0 \cdot y_2^* = 0 \implies y_2^* \geq 0         \\
        0 \cdot y_3^* = 0 \implies y_3^* \geq 0
    \end{cases}
\]\\

Вычислим \(y_2^*, y_3^*\) с учётом что \(y_1^* = 0\)\\

\[
    \begin{cases}
        (y_1^* + y_2^* - y_3^* - 3) \cdot \frac{36}{5} = 0 \\
        (-3y_1^* + y_2^* + 4y_3^* - 4) \cdot \frac{14}{5} = 0
    \end{cases} \implies
    \begin{cases}
        y_2^* - y_3^* - 3 = 0 \implies y_2^* = \frac{15}{5} \\
        y_2^* + 4y_3^* - 4 = 0 \implies y_3^* = \frac{1}{5}
    \end{cases}
\]\\

Вектор решения: \[y^* = (0;\frac{15}{5};\frac{1}{5})\]\\
Подставим решение в \(F^*\) и сравним с тем что получалось в  \(F\):\\
\[F^* = 10 \cdot \frac{16}{5} + 4 \cdot \frac{1}{5} = \frac{164}{5}\]\\

Правильное решение найдено.\\

\vspace{25pt}

Решим задачу 2 способом для этого возьмём конечную симплекс-таблицу для базовой задачи:\\

\[
    \begin{array}{|c|c|c|c|c|c|c|c|}
        \hline
        \text{Базис} & A_1 & A_2 & A_3 & A_4          & A_5          & B_i          & C \\
        \hline
        A_1          & 1   & 0   & 0   & \frac{4}{5}  & -\frac{1}{5} & \frac{36}{5} & 3 \\
        A_2          & 0   & 1   & 0   & \frac{1}{5}  & \frac{4}{20} & \frac{14}{5} & 4 \\
        A_3          & 0   & 0   & 1   & -\frac{1}{5} & \frac{4}{5}  & \frac{21}{5} & 0 \\
        \hline
    \end{array}
\]\\

Считаем по формуле: \(y^* = C \cdot A^{-1}\)\\

Посчитаем значение \(y^*\):

\[
    y^* =\begin{pmatrix} 3 & 4 & 0 \end{pmatrix} \cdot \begin{pmatrix}
        0 & \frac{4}{5}  & -\frac{1}{5} \\
        0 & \frac{1}{5}  & \frac{4}{20} \\
        1 & -\frac{1}{5} & \frac{4}{5}
    \end{pmatrix} = \begin{pmatrix} 0 & \frac{16}{5} & \frac{1}{5} \end{pmatrix}
\]

Теперь посчитаем значение функции \(F^*\):

\[
    F^* = 10 \cdot \frac{16}{5} + 4 \cdot \frac{1}{5} = \frac{164}{5}
\]

Правильное решение найдено.



\newpage

\section{Задание №4(6)}

Условие задачи:

\[
    \begin{array}{|c|c|c|c|c|c|}
        \hline
        \text{Сырьё}      & A & B & C & D  & \text{Запасы} \\
        \hline
        \text{Металл}     & 1 & 6 & 4 & 5  & 800           \\
        \hline
        \text{Пластмасса} & 5 & 9 & 8 & 10 & 2500          \\
        \hline
        \text{Резина}     & 0 & 3 & 1 & 5  & 600           \\
        \hline
        \text{Прибыль}    & 2 & 7 & 8 & 4  & -             \\
        \hline
    \end{array}
\]\\
Математическая интерпретация задачи:\\
\[
    F = 2x_1 + 7x_2 + 8x_3 + 4x_4 \to \max
\]

\[
    \begin{cases}
        x_1 + 6 \cdot x_2 + 4 \cdot x_3 + 5 \cdot x_4 \leq 800           \\
        5 \cdot x_1 + 9 \cdot x_2 + 8 \cdot x_3 + 10 \cdot x_4 \leq 2500 \\
        3 \cdot x_2 + x_3 + 5 \cdot x_4 \leq 600                         \\
    \end{cases}
\]\\
Составим условие задачи для решения симплекс методом:\\

\[
    F - 2x_1 - 7x_2 - 8x_3 - 4x_4 = 0
\]

\[
    \begin{cases}
        x_1 + 6 \cdot x_2 + 4 \cdot x_3 + 5 \cdot x_4 + x_5 = 800           \\
        5 \cdot x_1 + 9 \cdot x_2 + 8 \cdot x_3 + 10 \cdot x_4 + x_6 = 2500 \\
        3 \cdot x_2 + x_3 + 5 \cdot x_4 \leq 600 + x_7 = 600                \\
    \end{cases}
\]

Составим начальную симплекс-таблицу:\\

\[
    \begin{array}{|c|c|c|>{\columncolor{green}}c|c|c|c|c|c|c|}
        \hline
             & x_1 & x_2 & x_3 & x_4 & x_5 & x_6 & x_7 & b_i  & b_i/p.c. \geq 0 \\
        \hline
        \rowcolor{yellow}
        x_5  & 1   & 6   & 4   & 5   & 1   & 0   & 0   & 800  & 200             \\
        x_6  & 5   & 9   & 8   & 10  & 0   & 1   & 0   & 2500 & \frac{625}{2}   \\
        x_7  & 0   & 3   & 1   & 5   & 0   & 0   & 1   & 600  & 600             \\
        F(x) & -2  & -7  & -8  & -4  & 0   & 0   & 0   & 0    & 0               \\
        \hline
    \end{array}
\]

Т.к. в последней строке есть элементы  \(\leq 0\) выбираем минимальный отрицательный элемент в последнем столбце и считаем последний столбец после чего выбираем разрешающий элемент.

Занулим все элементы выше и ниже разрешающего элемента:

\[
    \begin{array}{|c|c|c|c|c|c|c|c|c|c|}
        \hline
             & x_1          & x_2         & x_3 & x_4          & x_5          & x_6 & x_7 & b_i  & b_i/p.c. \geq 0 \\
        \hline
        x_3  & \frac{1}{4}  & \frac{3}{2} & 1   & \frac{5}{4}  & \frac{1}{4}  & 0   & 0   & 200  & 200             \\
        x_6  & 3            & -3          & 0   & 0            & -2           & 1   & 0   & 900  & \frac{625}{2}   \\
        x_7  & -\frac{1}{4} & \frac{3}{2} & 0   & \frac{15}{4} & -\frac{1}{4} & 0   & 1   & 400  & 600             \\
        F(x) & 0            & 5           & 0   & 6            & 2            & 0   & 0   & 1600 & -               \\
        \hline
    \end{array}
\]

В последней строке все элементы \(\geq 0 \implies\) оптимальный план найден.

Максимум функции достигается при \(x_1 = 0, x_2 = 0, x_3 = 200, x_4 = 0\), и значение целевой функции равно \(F(x) =1600\).

Составим двойственную задачу:

\[
    F^* = 800y_1 + 2500y_2 + 600y_3 \to \min
\]

Конечная симлекс-таблица с добавлением столбца C:

\[
    \begin{array}{|c|c|c|c|c|c|c|c|c|c|c|}
        \hline
             & x_1          & x_2         & x_3 & x_4          & x_5          & x_6 & x_7 & b_i  & b_i/p.c. \geq 0 & C \\
        \hline
        x_3  & \frac{1}{4}  & \frac{3}{2} & 1   & \frac{5}{4}  & \frac{1}{4}  & 0   & 0   & 200  & 200             & 8 \\
        x_6  & 3            & -3          & 0   & 0            & -2           & 1   & 0   & 900  & \frac{625}{2}   & 0 \\
        x_7  & -\frac{1}{4} & \frac{3}{2} & 0   & \frac{15}{4} & -\frac{1}{4} & 0   & 1   & 400  & 600             & 0 \\
        F(x) & 0            & 5           & 0   & 6            & 2            & 0   & 0   & 1600 & -               & - \\
        \hline
    \end{array}
\]

Считаем по формуле: \(y^* = C \cdot A^{-1}\)

Посчитаем значение \(y^*\):

\[
    y^* =\begin{pmatrix} 8 & 0 & 0 \end{pmatrix} \cdot \begin{pmatrix}
        \frac{1}{4}  & 0 & 0 \\
        -2           & 1 & 0 \\
        -\frac{1}{4} & 0 & 1
    \end{pmatrix} = \begin{pmatrix} 2 & 0 & 0 \end{pmatrix}
\]

Минимум функции достигается при \(y_1 = 2, y_2 = 0, y_3 = 0\).

Теперь посчитаем значение функции \(F^*\):

\[
    F^* = 800 \cdot 2 + 2500 \cdot 0 + 600 \cdot 0 = 1600
\]

Значения \(F \text{ и } F^*\) совпадают \(\implies\) задача решена правильно.

\textbf{Анализ результатов}

Подставим $ \mathbf{x^*} = \left(0; 0; 200; 0\right)$ в условия прямой задачи:

\[
    \begin{cases}
        0 + 6 \cdot 0 + 4 \cdot 200 + 5 \cdot 0 = 800                     \\
        5 \cdot 0 + 9 \cdot 0 + 8 \cdot 200 + 10 \cdot 0 = 1600 \leq 2500 \\
        3 \cdot 0 + 200 + 5 \cdot 0 = 200 \leq 600                        \\
    \end{cases}
\]

Второе и третье условия имеют строгий знак $<$, значит второй и третий ресурсы (пластмасса и резина) не являются дефицитными (остатки 900 и 400 соответственно).

Первое условие образует равенство $=$, значит первый ресурс (металл) дефицитен.

Подставим $ \mathbf{y^*} = \begin{pmatrix} 2 & 0 & 0 \end{pmatrix}$ в условия двойственной задачи:

\[
    \begin{cases}
        6 > 2  \\
        12 > 7 \\
        8 = 8  \\
        10 > 4
    \end{cases}
\]

Первое, второе и четвёртое условия имеют строгий знак $>$, следовательно, производить эти изделия экономически невыгодно.

Третье условие имеет равенство $=$, следовательно, двойственная оценка ресурса, используемого для изготовления продукта в точности равна доходам, а значит продукт выгодно производить.

Величина двойственных оценок показывает, насколько возрастает целевая функция при увеличении запасов дефицитного ресурса на единицу.
Увеличение запасов ресурса Р1 (металл) на единицу приведет к новому оптимальному плану.
Коэффициенты $A_B^{-1}$ показывают, что увеличение прибыли достигается засчет увеличения выпуска продукции $C$, при этом запасы пластмассы сократятся на $2$ единиц и запасы резины сократятся на $\frac{1}{2}$ единицы.


Ответ: $\mathbf{x^*} = \begin{pmatrix} 0 & 0 & 200 & 0 \end{pmatrix}, \mathbf{y^*} = \begin{pmatrix} 2 & 0 & 0 \end{pmatrix}$


\textbf{Анализ устойчивости двойственных оценок}

Определим интервалы устойчивости:
\[x^*_{B\text{нов}} = x_B + A_B^{-1} \cdot (b + \Delta b)\]
\[ A_B^{-1} \cdot (b + \Delta b) \geq 0 \]

\[
    A_B^{-1} \cdot (b + \Delta b) = \begin{pmatrix}
        \frac{1}{4}  & 0 & 0 \\
        -2           & 1 & 0 \\
        -\frac{1}{4} & 0 & 1
    \end{pmatrix} \cdot \begin{pmatrix}
        800 + \Delta b_1  \\
        2500 + \Delta b_2 \\
        600 + \Delta b_3
    \end{pmatrix} = \begin{pmatrix}
        200 + \frac{1}{4} \Delta b_1    \\
        900 - 2 \Delta b_1 + \Delta b_2 \\
        400 - \frac{1}{4} \Delta b_1 + \Delta b_3
    \end{pmatrix} \geq 0
\]

\newpage
Рассмотрим частные случаи:
\begin{enumerate}
    \item $\Delta b_1 \geq 0, \Delta b_2 = 0, \Delta b_3 = 0$:
          \[
              \begin{pmatrix}
                  200 + \frac{1}{4} \Delta b_1 \\
                  900 - 2 \Delta b_1           \\
                  400 - \frac{1}{4} \Delta b_1
              \end{pmatrix} \geq 0 \Leftrightarrow \begin{cases}
                  200 + \frac{1}{4} \Delta b_1 \geq 0 \\
                  900 - 2 \Delta b_1 \geq 0           \\
                  400 - \frac{1}{4} \Delta b_1 \geq 0
              \end{cases} \Leftrightarrow \begin{cases}
                  \Delta b_1 \geq -800 \\
                  \Delta b_1 \leq 450  \\
                  \Delta b_1 \leq 1600
              \end{cases} \Leftrightarrow -800 \leq \Delta b_1 \leq 450
          \]
          При увеличении запасов 1-го ресурса не более чем на 450 единиц и уменьшении его запасов не более чем на 800 единиц значение целевой функции не изменится.
    \item $\Delta b_1 = 0, \Delta b_2 \geq 0, \Delta b_3 = 0$:
          \[
              \begin{pmatrix}
                  200              \\
                  900 + \Delta b_2 \\
                  400
              \end{pmatrix} \geq 0
              \Leftrightarrow 900 + \Delta b_2 \geq 0 \Leftrightarrow \Delta b_2 \geq -900
          \]
          При уменьшении запасов 2-го ресурса не более чем на 900 единиц, при этом оптимальный план двойственной задачи не изменится.
    \item $\Delta b_1 = 0, \Delta b_2 = 0, \Delta b_3 \geq 0$:
          \[
              \begin{pmatrix}
                  200 \\
                  900 \\
                  400 + \Delta b_3
              \end{pmatrix} \geq 0
              \Leftrightarrow 400 + \Delta b_3 \geq 0 \Leftrightarrow \Delta b_3 \geq -400
          \]
          При уменьшении запасов 3-го ресурса не более чем на 400 единиц, при этом оптимальный план двойственной задачи не изменится.
\end{enumerate}

Предположим: $\Delta b_1 = 450, \Delta b_2 = -900, \Delta b_3 = 400$:

\[
    \begin{pmatrix}
        x_3^{\text{нов}} \\
        x_6^{\text{нов}} \\
        x_7^{\text{нов}}
    \end{pmatrix}
    =
    \begin{pmatrix}
        200 + \frac{1}{4} \cdot 450 \\
        900 - 2 \cdot 450 - 900     \\
        400 - \frac{1}{4} \cdot 450 + 400
    \end{pmatrix} = \begin{pmatrix}
        \frac{625}{2} \\
        0             \\
        \frac{1375}{2}
    \end{pmatrix} \geq 0
\]

Посчитаем новое значение целевой функции:

\[
    F = 8 \cdot \frac{625}{2} = 2500
\]

\newpage
\section{Задание №5}
\textbf{Условие задачи:}

Рассмотрим закрытую транспортную задачу размером \(5 \times 4\) с пятью поставщиками и четырьмя потребителями. Общий запас равен общему спросу.

\textbf{Данные задачи:}

\begin{itemize}
    \item \textbf{Запасы поставщиков} (в единицах товара):
          \[
              S_1 = 55, \quad S_2 = 75, \quad S_3 = 100, \quad S_4 = 60, \quad S_5 = 110
          \]
    \item \textbf{Потребности потребителей} (в единицах товара):
          \[
              D_1 = 90, \quad D_2 = 110, \quad D_3 = 80, \quad D_4 = 120
          \]
\end{itemize}

Проверим общий баланс:
\[
    S = S_1 + S_2 + S_3 + S_4 + S_5 = 400
\]
\[
    D = D_1 + D_2 + D_3 + D_4 = 400
\]

Так как общий запас равен общему спросу, задача является закрытой.

\textbf{Матрица стоимости транспортировки} (в таблице указана стоимость транспортировки единицы товара от поставщика \(S_i\) к потребителю \(D_j\)):

\[
    \begin{array}{c|ccccc|c}
                      & S_1 & S_2 & S_3 & S_4 & S_5 & \text{Потребности} \\
        \hline
        D_1           & 4   & 5   & 6   & 7   & 3   & 90                 \\
        D_2           & 8   & 1   & 3   & 4   & 6   & 110                \\
        D_3           & 6   & 4   & 9   & 3   & 5   & 80                 \\
        D_4           & 3   & 7   & 2   & 8   & 1   & 120                \\
        \hline
        \text{Запасы} & 55  & 75  & 100 & 60  & 110 &                    \\
    \end{array}
\]

Целевая функция:
\[
    F = 4x_{11} + 5x_{12} + 6x_{13} + ... + 2x_{44} + 8x_{45} + x_{46}
\]

Ограничения:

\[
    \begin{cases}
        x_{11} + x_{12} + x_{13} + x_{14} + x_{15} = 90  \\
        x_{21} + x_{22} + x_{23} + x_{24} + x_{25} = 110 \\
        x_{31} + x_{32} + x_{33} + x_{34} + x_{35} = 80  \\
        x_{41} + x_{42} + x_{43} + x_{44} + x_{45} = 120 \\
        x_{11} + x_{21} + x_{31} + x_{41} = 55           \\
        x_{12} + x_{22} + x_{32} + x_{42} = 75           \\
        x_{13} + x_{23} + x_{33} + x_{43} = 100          \\
        x_{14} + x_{24} + x_{34} + x_{44} = 60           \\
        x_{15} + x_{25} + x_{35} + x_{45} = 110
    \end{cases}
\]\\

Задача состоит в том, чтобы минимизировать общую стоимость (целевую функцию) транспортировки при соблюдении ограничений на запасы и потребности.

\textbf{Решение задачи}

\textbf{Метод северо-западного угла}

Метод северо-западного угла предполагает заполнение транспортной таблицы, начиная с левой верхней ячейки и двигаясь по строкам и столбцам. На каждом шаге распределяем максимум возможного количества товара в текущую ячейку, обновляя остатки.

\textbf{Шаги метода северо-западного угла:}

\begin{enumerate}
    \item Ячейка \((S_1, D_1)\): минимальное значение между \(55\) и \(90\) — это \(55\). Заполняем \(55\), обновляем \(S_1 = 0\), \(D_1 = 35\).
    \item Ячейка \((S_2, D_1)\): минимальное значение между \(75\) и \(35\) — это \(35\). Заполняем \(35\), обновляем \(S_2 = 40\), \(D_1 = 0\).
    \item Ячейка \((S_2, D_2)\): минимальное значение между \(40\) и \(110\) — это \(40\). Заполняем \(40\), обновляем \(S_2 = 0\), \(D_2 = 70\).
    \item Ячейка \((S_3, D_2)\): минимальное значение между \(100\) и \(70\) — это \(70\). Заполняем \(70\), обновляем \(S_3 = 30\), \(D_2 = 0\).
    \item Ячейка \((S_3, D_3)\): минимальное значение между \(30\) и \(80\) — это \(30\). Заполняем \(30\), обновляем \(S_3 = 0\), \(D_3 = 50\).
    \item Ячейка \((S_4, D_3)\): минимальное значение между \(60\) и \(50\) — это \(50\). Заполняем \(50\), обновляем \(S_4 = 10\), \(D_3 = 0\).
    \item Ячейка \((S_4, D_4)\): минимальное значение между \(10\) и \(120\) — это \(10\). Заполняем \(10\), обновляем \(S_4 = 0\), \(D_4 = 110\).
    \item Ячейка \((S_5, D_4)\): минимальное значение между \(110\) и \(110\) — это \(110\). Заполняем \(110\), обновляем \(S_5 = 0\), \(D_4 = 0\).
\end{enumerate}

\textbf{Итоговое распределение методом северо-западного угла:}

\[
    \begin{array}{c|ccccc|c}
                      & S_1  & S_2  & S_3  & S_4  & S_5   & \text{Потребности} \\
        \hline
        D_1           & 55^4 & 35^5 & 0^6  & 0^7  & 0^3   & 90                 \\
        D_2           & 0^8  & 40^1 & 70^3 & 0^4  & 0^6   & 110                \\
        D_3           & 0^6  & 0^4  & 30^9 & 50^3 & 0^5   & 80                 \\
        D_4           & 0^3  & 0^7  & 0^2  & 10^8 & 110^1 & 120                \\
        \hline
        \text{Запасы} & 55   & 75   & 100  & 60   & 110   &                    \\
    \end{array}
\]

\textbf{Вычисление общей стоимости}

Теперь рассчитаем общую стоимость транспортировки \(F\), используя полученное распределение:

\[
    F = 55 \cdot 4 + 35 \cdot 5 + 40 \cdot 1 + 70 \cdot 3 + 30 \cdot 9 + 50 \cdot 3 + 10 \cdot 8 + 110 \cdot 1 = 1255
\]

Итак, общая стоимость транспортировки составляет \(F = 1255\).

\textbf{Итоговое распределение \(X\)}

Итоговая матрица распределения \(X\):

\[
    X = \begin{pmatrix}
        55 & 35 & 0  & 0  & 0   \\
        0  & 40 & 70 & 0  & 0   \\
        0  & 0  & 30 & 50 & 0   \\
        0  & 0  & 0  & 10 & 110 \\
    \end{pmatrix}
\]

\textbf{Алгоритм метода минимального элемента}

Метод минимального элемента включает следующие шаги:

\begin{enumerate}
    \item Найти ячейку с наименьшей стоимостью в матрице \(C_{ij}\).

    \item Заполнить ячейку \((i, j)\) максимальным возможным количеством: \(\min(S_i, D_j)\).

    \item Обновить запасы и потребности, вычитая заполненное количество из соответствующих значений \(S_i\) и \(D_j\).

    \item Если потребность или запас равен нулю, вычеркнуть соответствующую строку или столбец.

    \item Повторить шаги 1-4, пока все потребности и запасы не будут удовлетворены.
\end{enumerate}

\textbf{Решение методом минимального элемента}

\[
    \begin{array}{c|ccccc|c}
                      & S_1  & S_2  & S_3  & S_4  & S_5  & \text{Потребности} \\
        \hline
        D_1           & 0^4  & 0^5  & 0^6  & 0^7  & 90^3 & 90                 \\
        D_2           & 0^8  & 75^1 & 35^3 & 0^4  & 0^6  & 110                \\
        D_3           & 0^6  & 0^4  & 0^9  & 60^3 & 20^5 & 80                 \\
        D_4           & 55^3 & 0^7  & 65^2 & 0^8  & 0^1  & 120                \\
        \hline
        \text{Запасы} & 55   & 75   & 100  & 60   & 110  &                    \\
    \end{array}
\]

\textbf{Вычисление общей стоимости}

Теперь рассчитаем общую стоимость транспортировки \(F\), используя полученное распределение:

\[
    F = 90 \cdot 3 + 75 + 35 \cdot 3 + 60 \cdot 3 + 20 \cdot 5 + 55 \cdot 3 + 65 \cdot 2 = 1025
\]

Итак, общая стоимость транспортировки составляет \(F = 1025\).

\textbf{Итоговое распределение \(X\)}

Итоговая матрица распределения \(X\):

\[
    X = \begin{pmatrix}
        0  & 0  & 0  & 0  & 90 \\
        0  & 75 & 35 & 0  & 0  \\
        0  & 0  & 0  & 60 & 20 \\
        55 & 0  & 65 & 0  & 0  \\
    \end{pmatrix}
\]

\textbf{Метод потенциалов}

Метод потенциалов используется для проверки оптимальности текущего распределения и нахождения улучшенного решения, если оно не оптимально.

Для базисных клеток используем условие \( U_i + V_j = C_{ij} \). Примем \( U_1 = 0 \) и вычислим остальные потенциалы.

Обозначение: \( C^x_y \), где С - количество поставляемого груза, x - цена за единицу, y - потенциал.

Составим таблицу:
\[
    \begin{array}{c|ccccc|c}
        \text{Потребности/Запасы} & 55_4     & 75_5     & 100_7                                 & 60_1                                  & 110_{-6} &     \\
        \hline
        90_0                      & 55^4_-   & 35^5_-   & 0^6_{-1}                              & 0^7_6                                 & 0^3_9    & D_1 \\
        110_{-4}                  & 0^8_8    & 40^1_-   & 70^3_-                                & 0^4_7                                 & 0^6_{16} & D_2 \\
        80_2                      & 0^6_0    & 0^4_{-3} & \cellcolor[RGB]{102, 205, 170} 30^9_- & \cellcolor[RGB]{102, 205, 170} 50^3_- & 0^5_9    & D_3 \\
        120_7                     & 0^3_{-8} & 0^7_{-5} & \cellcolor{orange} 0^2_{-14}          & \cellcolor[RGB]{102, 205, 170} 10^8_- & 110^1_-  & D_4 \\
        \hline
                                  & S_1      & S_2      & S_3                                   & S_4                                   & S_5      &     \\
    \end{array}
\]

Есть потенциалы \( (< 0) \).\\
Найдем элемент с наименьшим потенциалом: \( (S_3, D_3) \).\\
Построим цикл зелёным цветом.\\
Проделаем перераспределение товаров и построим новую таблицу:
\[
    \begin{array}{c|ccccc|c}
        \text{Потребности/Запасы} & 55_4   & 75_5     & 100_7                                 & 60_1     & 110_6                                  &     \\
        \hline
        90_0                      & 55^4_- & 35^5_-   & 0^6_{-1}                              & 0^7_6    & 0^3_{-3}                               & D_1 \\
        110_{-4}                  & 0^8_8  & 40^1_-   & 70^3_-                                & 0^4_7    & 0^6_4                                  & D_2 \\
        80_2                      & 0^6_0  & 0^4_{-3} & \cellcolor[RGB]{102, 205, 170} 20^9_- & 60^3_-   & \cellcolor{orange} 0^5_{-3}            & D_3 \\
        120_{-5}                  & 0^3_4  & 0^7_0    & \cellcolor[RGB]{102, 205, 170} 10^2_- & 0^8_{12} & \cellcolor[RGB]{102, 205, 170} 110^1_- & D_4 \\
        \hline
                                  & S_1    & S_2      & S_3                                   & S_4      & S_5                                    &     \\
    \end{array}
\]
Есть потенциалы \( (< 0) \).\\
Найдем элемент с наименьшим потенциалом: \( (S_5, D_3) \).\\
Построим цикл зелёным цветом.\\
Проделаем перераспределение товаров и построим новую таблицу:
\[
    \begin{array}{c|ccccc|c}
        \text{Потребности/Запасы} & 55_4   & 75_5                                  & 100_7                                 & 60_4   & 110_6                                 &     \\
        \hline
        90_0                      & 55^4_- & \cellcolor[RGB]{102, 205, 170} 35^5_- & 0^6_{-1}                              & 0^7_3  & \cellcolor{orange} 0^3_{-3}           & D_1 \\
        110_{-4}                  & 0^8_8  & \cellcolor[RGB]{102, 205, 170} 40^1_- & \cellcolor[RGB]{102, 205, 170} 70^3_- & 0^4_0  & 0^6_4                                 & D_2 \\
        80_{-1}                   & 0^6_3  & 0^4_0                                 & 0^9_3                                 & 60^3_- & 20^5_-                                & D_3 \\
        120_{-5}                  & 0^3_4  & 0^7_7                                 & \cellcolor[RGB]{102, 205, 170} 30^2_- & 0^8_9  & \cellcolor[RGB]{102, 205, 170} 90^1_- & D_4 \\
        \hline
                                  & S_1    & S_2                                   & S_3                                   & S_4    & S_5                                   &     \\
    \end{array}
\]
Есть потенциалы \( (< 0) \).\\
Найдем элемент с наименьшим потенциалом: \( (S_5, D_1) \).\\
Построим цикл зелёным цветом.\\
Проделаем перераспределение товаров и построим новую таблицу:

\[
    \begin{array}{c|ccccc|c}
        \text{Потребности/Запасы} & 55_4   & 75_5                                  & 100_7                                 & 60_1   & 110_3                                 &     \\
        \hline
        90_0                      & 55^4_- & 0^5_0                                 & 0^6_{-1}                              & 0^7_6  & 35^3_-                                & D_1 \\
        110_{-4}                  & 0^8_8  & \cellcolor[RGB]{102, 205, 170} 75^1_- & \cellcolor[RGB]{102, 205, 170} 35^3_- & 0^4_7  & 0^6_7                                 & D_2 \\
        80_2                      & 0^6_0  & \cellcolor{orange} 0^4_{-3}           & 0^9_0                                 & 60^3_- & \cellcolor[RGB]{102, 205, 170} 20^5_- & D_3 \\
        120_{-2}                  & 0^3_1  & 0^7_4                                 & \cellcolor[RGB]{102, 205, 170} 65^2_- & 0^8_9  & \cellcolor[RGB]{102, 205, 170} 55^1_0 & D_4 \\
        \hline
                                  & S_1    & S_2                                   & S_3                                   & S_4    & S_5                                   &     \\
    \end{array}
\]
Есть потенциалы \( (< 0) \).\\
Найдем элемент с наименьшим потенциалом: \( (S_5, D_1) \).\\
Построим цикл зелёным цветом.\\
Проделаем перераспределение товаров и построим новую таблицу:

\[
    \begin{array}{c|ccccc|c}
        \text{Потребности/Запасы} & 55_4   & 75_2   & 100_4  & 60_1   & 110_3  &     \\
        \hline
        90_0                      & 55^4_- & 0^5_3  & 0^6_2  & 0^7_6  & 35^3_- & D_1 \\
        110_{-1}                  & 0^8_5  & 55^1_- & 55^3_- & 0^4_4  & 0^6_4  & D_2 \\
        80_2                      & 0^6_0  & 20^4_- & 0^9_3  & 60^3_- & 0^5_0  & D_3 \\
        120_{-2}                  & 0^3_1  & 0^7_7  & 45^2_- & 0^8_9  & 75^1_- & D_4 \\
        \hline
                                  & S_1    & S_2    & S_3    & S_4    & S_5    &     \\
    \end{array}
\]

Все потенциалы \( (\geq 0) \) оптимальный план найден.\\

\[
    F = 55 \cdot 4 + 35 \cdot 3 + 55 + 55 \cdot 3 + 20 \cdot 4 + 60 \cdot 3 + 45 \cdot 2 + 75 = 970
\]

Итак, общая стоимость транспортировки составляет \(F = 970\).

Итоговая матрица распределения \(X\):

\[
    X = \begin{pmatrix}
        55 & 0  & 0  & 0  & 35 \\
        0  & 55 & 55 & 0  & 0  \\
        0  & 20 & 0  & 60 & 0  \\
        0  & 0  & 45 & 0  & 75 \\
    \end{pmatrix}
\]

Используя код на Python:

\begin{verbatim}
from cvxopt.modeling import variable, op
import time

start = time.time()

# Переменные
x = variable(20, 'x')

# Стоимости
c = [4, 8, 6, 3, 5, 1, 4, 7, 6, 3, 9, 2, 7, 4, 3, 8, 3, 6, 5, 1]

# Целевая функция
z = sum(c[i] * x[i] for i in range(20))

# Ограничения
supply = [55, 75, 100, 60, 110]
demand = [90, 110, 80, 120]

constraints = []
for i in range(5):
    constraints.append(sum(x[i * 4 + j] for j in range(4)) <= supply[i])

for j in range(4):
    constraints.append(sum(x[i * 4 + j] for i in range(5)) == demand[j])

x_non_negative = (x >= 0)
constraints.append(x_non_negative)

# Постановка задачи
problem = op(z, constraints)

# Решение задачи
problem.solve(solver='glpk')

# Вывод результатов
print("Результат Xopt:")
for i in x.value:
    print(i)

print("Стоимость доставки:")
print(problem.objective.value()[0])

stop = time.time()
print("Время:")
print(stop - start)
\end{verbatim}

Получаем такие же значения:
\begin{verbatim}
GLPK Simplex Optimizer 5.0
29 rows, 20 columns, 60 non-zeros
      0: obj =   0.000000000e+00 inf =   4.000e+02 (4)
      8: obj =   1.255000000e+03 inf =   0.000e+00 (0)
*    20: obj =   9.700000000e+02 inf =   0.000e+00 (0)
OPTIMAL LP SOLUTION FOUND
Result Xopt:
 55   0   0   0 
  0  55  20   0 
  0  55   0  45 
  0   0  60   0 
 35   0   0  75 
Cost: 970.0
Time:0.01
\end{verbatim}

\text{Результаты совапали.}

\newpage
\section{Задание №6}

Оптимизационная задача:
\[
    F = 3x_1 + 4x_2 \to \max
\]
Ограничения:
\[
    \begin{cases}
        x_1 - 3 \cdot x_2 \leq 3  \\
        x_1 + x_2 \leq 10         \\
        -x_1 + 4 \cdot x_2 \leq 4 \\
        x_1 \geq 0, \ x_2 \geq 0
    \end{cases}
\]
\subsection{Графическое решение задачи.}
\begin{figure}[h]
    \centering
    \begin{tikzpicture}
        \begin{axis}[
                xlabel={$x_1$},
                ylabel={$x_2$},
                axis lines=middle,
                xmin=0, xmax=10,
                ymin=0, ymax=10,
                grid=both,
                width=14cm,
                height=12cm,
                legend pos=north east
            ]
            % Линии ограничений с разными цветами
            \addplot[name path=A, domain=0:10, thick, blue] { (x - 3)/3 };
            \addplot[name path=B, domain=0:10, thick, red] { 10 - x };
            \addplot[name path=C, domain=0:10, thick, green] { (4 + x)/4 };

            % Область допустимых решений
            \addplot[fill opacity=0.2, color=gray] fill between[
                    of=A and B,
                    soft clip={domain=6:10}
                ];

            % Вектор {3, 4}
            \addplot[->, thick, black] coordinates {(0,0) (3,4)};
            \node at (axis cs: 3,4) [anchor=west] {$\vec{v} = \{3,4\}$};

            % Подписи
            \legend{
                $x_1 - 3x_2 = 3$,
                $x_1 + x_2 = 10$,
                $-x_1 + 4x_2 = 4$
            }

        \end{axis}
    \end{tikzpicture}

    \caption{Графическое решение задачи}
\end{figure}

Вычисление точек пересечения

\(x_1 + x_2 = 10\) и \(-x_1 + 4x_2 = 4\):
\[
    \begin{cases}
        x_1 + x_2 = 10 \\
        -x_1 + 4x_2 = 4
    \end{cases}
    \implies
    \begin{cases}
        x_1 = \frac{36}{5} \\
        x_2 = \frac{14}{5}
    \end{cases}
\]
Точка пересечения: \(\left(\frac{36}{5}, \frac{14}{5}\right)\).

Вычисление значений целевой функции \(F = 3x_1 + 4x_2\):

\(\left(\frac{36}{5}, \frac{14}{5}\right)\):
\[
    F\left(\frac{36}{5}, \frac{14}{5}\right) = \frac{164}{5}
\]

Ответ:
\begin{tabular}{l l}
    Максимум функции \(F = 3x_1 + 4x_2\) & достигается в точке \(\left(\frac{36}{5}, \frac{14}{5}\right)\), где \(F = \frac{164}{5}\).
\end{tabular}\\

\subsection{Решение задачи с помощью симплекс-метода:}

Оптимизационная задача:
\[
    F = 3x_1 + 4x_2 \to \max
\]
Ограничения:
\[
    \begin{cases}
        x_1 - 3 \cdot x_2 \leq 3  \\
        x_1 + x_2 \leq 10         \\
        -x_1 + 4 \cdot x_2 \leq 4 \\
        x_1 \geq 0, \ x_2 \geq 0
    \end{cases}
\]

Решение задачи с помощью симплекс-метода:
\[
    F - 3x_1 - 4x_2 = 0
\]

\[
    \begin{cases}
        x_1 - 3 \cdot x_2 + x_3 = 3  \\
        x_1 + x_2 + x_4 = 10         \\
        -x_1 + 4 \cdot x_2 + x_5 = 4 \\
        x_1 \geq 0, \ x_2 \geq 0, \ x_3 \geq 0, \ x_4 \geq 0, \ x_5 \geq 0
    \end{cases}
\]

\vspace{25pt}

\[
    \begin{array}{|c|c|>{\columncolor{green}}c|c|c|c|c|c|}
        \hline
             & x_1 & x_2 & x_3 & x_4 & x_5 & b_i & b_i/p.c. \geq 0 \\
        \hline
        x_3  & 1   & -3  & 1   & 0   & 0   & 3   & -1              \\
        x_4  & 1   & 1   & 0   & 1   & 0   & 10  & 10              \\
        \rowcolor{yellow}
        x_5  & -1  & 4   & 0   & 0   & 1   & 4   & 1               \\
        F(x) & -3  & -4  & 0   & 0   & 0   & 0   & 0               \\
        \hline
    \end{array}
\]

В последней строке есть элементы \(\leq 0\). Занулим элементы выше и ниже стоящие от разрешающего элемента.

\[
    \begin{array}{|c|>{\columncolor{green}}c|c|c|c|c|c|c|}
        \hline
             & x_1          & x_2 & x_3 & x_4 & x_5          & b_i & b_i/p.c. \geq 0 \\
        \hline
        x_2  & -\frac{1}{4} & 1   & 0   & 0   & \frac{1}{4}  & 1   & -4              \\
        x_3  & \frac{1}{4}  & 0   & 1   & 0   & \frac{3}{4}  & 6   & 24              \\
        \rowcolor{yellow}
        x_4  & \frac{5}{4}  & 0   & 0   & 1   & -\frac{1}{4} & 9   & \frac{36}{5}    \\
        F(x) & -4           & 0   & 0   & 0   & 1            & 4   &                 \\
        \hline
    \end{array}
\]

В последней строке есть элементы \(\leq 0\). Занулим элементы выше и ниже стоящие от разрешающего элемента.

\[
    \begin{array}{|c|c|c|c|c|c|c|c|}
        \hline
             & x_1 & x_2 & x_3 & x_4          & x_5          & b_i           & b_i/p.c. \geq 0 \\
        \hline
        x_1  & 1   & 0   & 0   & \frac{4}{5}  & -\frac{1}{5} & \frac{36}{5}  & \frac{36}{5}    \\
        x_2  & 0   & 1   & 0   & \frac{1}{5}  & \frac{4}{20} & \frac{14}{5}  & -               \\
        x_3  & 0   & 0   & 1   & -\frac{1}{5} & \frac{4}{5}  & \frac{21}{5}  & 24              \\
        F(x) & 0   & 0   & 0   & \frac{16}{5} & \frac{1}{5}  & \frac{164}{5} &                 \\
        \hline
    \end{array}
\]

В последней строке не осталось элементов \(\leq 0\). Мы пришли к конечной таблице.\\
Максимум функции достигается при \(x_1 = \frac{36}{5}, x_2 = \frac{14}{5}\), и значение целевой функции равно \(F(x) = \frac{164}{5}\).

\newpage
\subsection{Решение задачи методом отсечения Гомори:}

\subsubsection{Геометрическим методом:}

\begin{figure}[h]
    \centering
    \begin{tikzpicture}
        \begin{axis}[
                xlabel={$x_1$},
                ylabel={$x_2$},
                axis lines=middle,
                xmin=0, xmax=10,
                ymin=0, ymax=10,
                grid=both,
                width=14cm,
                height=12cm,
                legend pos=north east
            ]
            % Линии ограничений с разными цветами
            \addplot[name path=A, domain=0:10, thick, blue] { (x - 3)/3 };
            \addplot[name path=B, domain=0:10, thick, red] { 10 - x };
            \addplot[name path=C, domain=0:10, thick, green] { (4 + x)/4 };

            % Область допустимых решений
            \addplot[fill opacity=0.2, color=gray] fill between[
                    of=A and B,
                    soft clip={domain=6:10}
                ];

            % Вектор {3, 4}
            \addplot[->, thick, black] coordinates {(0,0) (3,4)};
            \node at (axis cs: 3,4) [anchor=west] {$\vec{v} = \{3,4\}$};

            % Подписи
            \legend{
                $x_1 - 3x_2 = 3$,
                $x_1 + x_2 = 10$,
                $-x_1 + 4x_2 = 4$
            }

            \addplot[mark=*, red] coordinates {(8, 2)};
            \node at (axis cs:8,2) [anchor=west] {$(8, 2)$};
            \addplot[mark=*, red] coordinates {(7, 2)};
            \addplot[mark=*, red] coordinates {(6, 2)};
            \addplot[mark=*, red] coordinates {(6, 1)};
            \addplot[mark=*, red] coordinates {(5, 2)};
            \addplot[mark=*, red] coordinates {(5, 1)};
            \addplot[mark=*, red] coordinates {(4, 2)};
            \addplot[mark=*, red] coordinates {(4, 1)};
            \addplot[mark=*, red] coordinates {(3, 0)};
            \addplot[mark=*, red] coordinates {(3, 1)};
            \addplot[mark=*, red] coordinates {(2, 0)};
            \addplot[mark=*, red] coordinates {(2, 1)};
            \addplot[mark=*, red] coordinates {(1, 0)};
            \addplot[mark=*, red] coordinates {(1, 1)};

        \end{axis}
    \end{tikzpicture}

    \caption{Графическое решение задачи}
\end{figure}

Решение:
\[
    \begin{cases}
        x_1 = \frac{36}{5}, \\
        x_2 = \frac{14}{5}.
    \end{cases}
\]

Целевая функция:
\[
    F\left(\frac{36}{5}, \frac{14}{5}\right) = 3 \cdot \frac{36}{5} + 4 \cdot \frac{14}{5} = \frac{164}{5}.
\]

Решение:
\[
    \begin{cases}
        x_1 = 8, \\
        x_2 = 2.
    \end{cases}
\]

Целевая функция:
\[
    F(8, 2) = 3 \cdot 8 + 4 \cdot 2 = 32.
\]

\textbf{Ответ:} Максимум функции \(F = 3x_1 + 4x_2\) с учетом целочисленных ограничений достигается в точке \((8, 2)\), где \(F = 32\).

\subsubsection{Симплекс-методом:}

Добавляем дополнительные переменные \(x_3, x_4, x_5\) для приведения ограничений к равенствам:
\[
    \begin{cases}
        x_1 - 3x_2 + x_3        & = 3    \\
        x_1 + x_2 + x_4         & = 10   \\
        -x_1 + 4x_2 + x_5       & = 4    \\
        x_1, x_2, x_3, x_4, x_5 & \geq 0
    \end{cases}
\]

Целевая функция:
\[
    F = -3x_1 - 4x_2 \to \min.
\]


\textbf{Конечная симплекс-таблица:}
\[
    \begin{array}{|c|c|c|c|c|c|c|}
        \hline
            & x_1 & x_2 & x_3 & x_4          & x_5          & b             \\ \hline
        x_1 & 1   & 0   & 0   & \frac{4}{5}  & -\frac{1}{5} & \frac{36}{5}  \\ \hline
        x_2 & 0   & 1   & 0   & \frac{1}{5}  & \frac{1}{5}  & \frac{14}{5}  \\ \hline
        x_3 & 0   & 0   & 1   & -\frac{1}{5} & \frac{4}{5}  & \frac{21}{5}  \\ \hline
        F   & 0   & 0   & 0   & \frac{16}{5} & \frac{1}{5}  & \frac{164}{5} \\ \hline
    \end{array}
\]

Найдено нецелочисленное решение: \(x_1 = \frac{36}{5}\), \(x_2 = \frac{14}{5}\), \(F = \frac{164}{5}\).

Найдено оптимальное нецелочисленное решение. Среди свободных членов находим переменную с максимальным дробным числом:

\[
    x_1 = \frac{36}{5} = 1\frac{1}{5}, \quad x_2 = \frac{14}{5} = 2\frac{4}{5}
\]

Переменная \(x_2\) имеет максимальное дробное значение. Поэтому вводим дополнительное ограничение по 2 строке:

\[
    \begin{array}{|c|c|c|c|c|c|c|}
        \hline
            & x_1 & x_2 & x_3 & x_4          & x_5          & b             \\ \hline
        x_1 & 1   & 0   & 0   & \frac{4}{5}  & -\frac{1}{5} & \frac{36}{5}  \\ \hline
        \rowcolor{yellow}
        x_2 & 0   & 1   & 0   & \frac{1}{5}  & \frac{1}{5}  & \frac{14}{5}  \\ \hline
        x_3 & 0   & 0   & 1   & -\frac{1}{5} & \frac{4}{5}  & \frac{21}{5}  \\ \hline
        F   & 0   & 0   & 0   & \frac{16}{5} & \frac{1}{5}  & \frac{164}{5} \\ \hline
    \end{array}
\]

Записываем новое ограничение:

\[
    -\frac{4}{5} = -0x_1 - 0x_2 - 0x_3 - \frac{1}{5}x_4 - \frac{1}{5}x_5 + x_6
\]

\textbf{Обновлённая таблица:}
\[
    \begin{array}{|c|c|c|c|c|c|c|c|}
        \hline
                 & b             & x_1 & x_2 & x_3 & x_4          & x_5          & x_6 \\ \hline
        x_1      & \frac{36}{5}  & 1   & 0   & 0   & \frac{4}{5}  & -\frac{1}{5} & 0   \\ \hline
        x_2      & \frac{14}{5}  & 0   & 1   & 0   & \frac{1}{5}  & \frac{1}{5}  & 0   \\ \hline
        x_3      & \frac{21}{5}  & 0   & 0   & 1   & -\frac{1}{5} & \frac{4}{5}  & 0   \\ \hline
        x_1      & -\frac{4}{5}  & 0   & 0   & 0   & -\frac{1}{5} & -\frac{1}{5} & 1   \\ \hline
        F_{\max} & \frac{164}{5} & 0   & 0   & 0   & \frac{16}{5} & \frac{1}{5}  & 0   \\ \hline
    \end{array}
\]



Т.к. среди свободных членов есть отрицательные значения, то решение недопустимое, и сначала нужно перейти к допустимому решению. Для этого находим среди свободных членов максимальное отрицательное число по модулю. Это число будет задавать разрешающую (ведущую) строку.

В этой строке так же находим максимальный по модулю отрицательный элемент, который будет разрешающим (ведущим) столбцом.\\

\textbf{Разрешающий столбец:} \(x_4\)

\textbf{Разрешающая строка:} \(x_1\)\\


\textbf{Пересчитываем таблицу:}

\[
    \begin{array}{|c|c|c|c|c|>{\columncolor{green}}c|c|c|c|}
        \hline
                 & b             & x_1 & x_2 & x_3 & x_4          & x_5          & x_6 & \frac{b}{x_4} \\ \hline
        x_1      & \frac{36}{5}  & 1   & 0   & 0   & \frac{4}{5}  & -\frac{1}{5} & 0   & 9             \\ \hline
        x_2      & \frac{14}{5}  & 0   & 1   & 0   & \frac{1}{5}  & \frac{1}{5}  & 0   & 14            \\ \hline
        x_3      & \frac{21}{5}  & 0   & 0   & 1   & -\frac{1}{5} & \frac{4}{5}  & 0   & -21           \\ \hline
        \rowcolor{yellow}
        x_1      & -\frac{4}{5}  & 0   & 0   & 0   & -\frac{1}{5} & -\frac{1}{5} & 1   & 4             \\ \hline
        F_{\max} & \frac{164}{5} & 0   & 0   & 0   & \frac{16}{5} & \frac{1}{5}  & 0   &               \\ \hline
    \end{array}
\]

\textbf{Пересчитываем таблицу:}

\[
    \begin{array}{|c|c|c|c|c|c|c|c|}
        \hline
                 & b  & x_1 & x_2 & x_3 & x_4 & x_5 & x_6 \\ \hline
        x_1      & 4  & 1   & 0   & 0   & 0   & -1  & 4   \\ \hline
        x_2      & 2  & 0   & 1   & 0   & 0   & 0   & 1   \\ \hline
        x_3      & 5  & 0   & 0   & 1   & 0   & 1   & -1  \\ \hline
        x_4      & 4  & 0   & 0   & 0   & 1   & 1   & -5  \\ \hline
        F_{\max} & 20 & 0   & 0   & 0   & 0   & -3  & 16  \\ \hline
    \end{array}
\]

\textbf{Правило выбора разрешающего элемента:}

Среди коэффициентов целевой функции выбираем максимальный по модулю отрицательный элемент. Этот элемент определяет разрешающий столбец.

Разрешающая строка выбирается так, чтобы отношение свободного члена к элементу, находящемуся на пересечении разрешающего столбца и строки, было минимальным и неотрицательным.\\
Разрешающий столбец: \( x_5 \)\\
Разрешающая строка: \( x_4 \)\\

\[
    \begin{array}{|c|c|c|c|c|c|>{\columncolor{green}}c|c|c|}
        \hline
                 & b  & x_1 & x_2 & x_3 & x_4 & x_5 & x_6 & \frac{b}{x_5} \\ \hline
        x_1      & 4  & 1   & 0   & 0   & 0   & -1  & 4   & -4            \\ \hline
        x_2      & 2  & 0   & 1   & 0   & 0   & 0   & 1   & -             \\ \hline
        x_3      & 5  & 0   & 0   & 1   & 0   & 1   & -1  & 5             \\ \hline
        \rowcolor{yellow}
        x_4      & 4  & 0   & 0   & 0   & 1   & 1   & -5  & 4             \\ \hline
        F_{\max} & 20 & 0   & 0   & 0   & 0   & -3  & 16  &               \\ \hline
    \end{array}
\]

\textbf{Пересчитываем таблицу:}

\[
    \begin{array}{|c|c|c|c|c|c|c|c|}
        \hline
                 & b  & x_1 & x_2 & x_3 & x_4 & x_5 & x_6 \\ \hline
        x_1      & 8  & 1   & 0   & 0   & 1   & 0   & -1  \\ \hline
        x_2      & 2  & 0   & 1   & 0   & 0   & 0   & 1   \\ \hline
        x_3      & 1  & 0   & 0   & 1   & -1  & 0   & 4   \\ \hline
        x_5      & 4  & 0   & 0   & 0   & 1   & 1   & -5  \\ \hline
        F_{\max} & 32 & 0   & 0   & 0   & 3   & 0   & 1   \\ \hline
    \end{array}
\]

Так как все коэффициенты при целевой функции неотрицательны, решение оптимально.

\textbf{Значения переменных:}
\[
    x_1 = 8, \quad x_2 = 2
\]

\textbf{Значение целевой функции:}
\[
    F_{\max}(x) = 32
\]

\section{Задание №7}

Придумать задачу коммивояжера размерности $10 \times 10$. Значения в матрице расстояний должны быть любыми целыми числами от 1 до 100.
Решить задачу методом ветвей и границ. Полный перебор не использовать.
После выполнения задания добавить в отчёт граф решения, добавить решение задачи с помощью программных средств.

\section*{Постановка задачи}

Рассмотрим задачу коммивояжера для $10$ городов. Пусть города обозначены номерами от $1$ до $10$. Задана матрица расстояний $C = (c_{ij})$, где $c_{ij}$ — расстояние между городами $i$ и $j$. Требуется найти минимальный замкнутый путь, проходящий через каждый город ровно один раз.

\textbf{Матрица расстояний:}
\[
    C = \begin{pmatrix}
        \infty & 29     & 20     & 21     & 16     & 31     & 100    & 12     & 4      & 31     \\
        29     & \infty & 15     & 29     & 28     & 40     & 72     & 21     & 29     & 41     \\
        20     & 15     & \infty & 15     & 14     & 25     & 81     & 9      & 23     & 27     \\
        21     & 29     & 15     & \infty & 4      & 12     & 92     & 12     & 25     & 13     \\
        16     & 28     & 14     & 4      & \infty & 16     & 94     & 9      & 20     & 16     \\
        31     & 40     & 25     & 12     & 16     & \infty & 95     & 24     & 36     & 3      \\
        100    & 72     & 81     & 92     & 94     & 95     & \infty & 90     & 101    & 99     \\
        12     & 21     & 9      & 12     & 9      & 24     & 90     & \infty & 15     & 25     \\
        4      & 29     & 23     & 25     & 20     & 36     & 101    & 15     & \infty & 35     \\
        31     & 41     & 27     & 13     & 16     & 3      & 99     & 25     & 35     & \infty
    \end{pmatrix}.
\]

Здесь $\infty$ обозначает отсутствие дуги между городом $i$ и самим собой.

\section*{Метод решения: ветви и границы}

Метод ветвей и границ используется для эффективного решения задач дискретной оптимизации. Основная идея заключается в построении дерева решений, где каждая ветвь представляет собой подзадачу, а границы (оценки) позволяют исключить невыгодные подзадачи.

\subsection*{Шаг 1. Исходная оценка задачи}

1. Для исходной матрицы $C$ выполните \textbf{редукцию строк и столбцов}:
- Для каждой строки вычтите минимальный элемент этой строки из всех её элементов.
- Для каждого столбца вычтите минимальный элемент этого столбца из всех его элементов.

Пример редукции:
\begin{enumerate}
    \item Минимальные элементы строк: $[4, 15, 9, 4, 4, 3, 72, 9, 4, 3]$.
    \item Вычитаем из строк минимумы:
          \[
              C' = \begin{pmatrix}
                  \infty & 25     & 16     & 17     & 12     & 27     & 96     & 8      & 0      & 27     \\
                  14     & \infty & 0      & 14     & 13     & 25     & 57     & 6      & 14     & 26     \\
                  11     & 6      & \infty & 6      & 5      & 16     & 72     & 0      & 14     & 18     \\
                  17     & 25     & 11     & \infty & 0      & 8      & 88     & 8      & 21     & 9      \\
                  12     & 24     & 10     & 0      & \infty & 12     & 90     & 5      & 16     & 12     \\
                  28     & 37     & 22     & 9      & 13     & \infty & 92     & 21     & 33     & 0      \\
                  28     & 0      & 9      & 20     & 22     & 23     & \infty & 18     & 29     & 27     \\
                  3      & 12     & 0      & 3      & 0      & 15     & 81     & \infty & 6      & 16     \\
                  0      & 25     & 19     & 21     & 16     & 32     & 97     & 11     & \infty & 31     \\
                  28     & 38     & 24     & 10     & 13     & 0      & 96     & 22     & 32     & \infty
              \end{pmatrix}.
          \]
\end{enumerate}

\subsection*{Шаг 2. Построение дерева решений}

\begin{enumerate}
    \item Выберите путь с минимальной оценкой.
    \item Разделите задачу на две подзадачи:
          \begin{itemize}
              \item Включить ребро $(i, j)$ в путь.
              \item Исключить ребро $(i, j)$ из пути.
          \end{itemize}
    \item Оцените каждую подзадачу (границы).
    \item Повторяйте до тех пор, пока не найдётся оптимальный путь.
\end{enumerate}

\subsection*{Шаг 3. Окончательное решение}

Оптимальный путь: $1 \to 8 \to 3 \to 7 \to 9 \to 5 \to 4 \to 6 \to 10 \to 2 \to 1$. Длина пути равна $87$.


\end{document}
